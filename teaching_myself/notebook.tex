
% Default to the notebook output style

    


% Inherit from the specified cell style.




    
\documentclass[11pt]{article}

    
    
    \usepackage[T1]{fontenc}
    % Nicer default font (+ math font) than Computer Modern for most use cases
    \usepackage{mathpazo}

    % Basic figure setup, for now with no caption control since it's done
    % automatically by Pandoc (which extracts ![](path) syntax from Markdown).
    \usepackage{graphicx}
    % We will generate all images so they have a width \maxwidth. This means
    % that they will get their normal width if they fit onto the page, but
    % are scaled down if they would overflow the margins.
    \makeatletter
    \def\maxwidth{\ifdim\Gin@nat@width>\linewidth\linewidth
    \else\Gin@nat@width\fi}
    \makeatother
    \let\Oldincludegraphics\includegraphics
    % Set max figure width to be 80% of text width, for now hardcoded.
    \renewcommand{\includegraphics}[1]{\Oldincludegraphics[width=.8\maxwidth]{#1}}
    % Ensure that by default, figures have no caption (until we provide a
    % proper Figure object with a Caption API and a way to capture that
    % in the conversion process - todo).
    \usepackage{caption}
    \DeclareCaptionLabelFormat{nolabel}{}
    \captionsetup{labelformat=nolabel}

    \usepackage{adjustbox} % Used to constrain images to a maximum size 
    \usepackage{xcolor} % Allow colors to be defined
    \usepackage{enumerate} % Needed for markdown enumerations to work
    \usepackage{geometry} % Used to adjust the document margins
    \usepackage{amsmath} % Equations
    \usepackage{amssymb} % Equations
    \usepackage{textcomp} % defines textquotesingle
    % Hack from http://tex.stackexchange.com/a/47451/13684:
    \AtBeginDocument{%
        \def\PYZsq{\textquotesingle}% Upright quotes in Pygmentized code
    }
    \usepackage{upquote} % Upright quotes for verbatim code
    \usepackage{eurosym} % defines \euro
    \usepackage[mathletters]{ucs} % Extended unicode (utf-8) support
    \usepackage[utf8x]{inputenc} % Allow utf-8 characters in the tex document
    \usepackage{fancyvrb} % verbatim replacement that allows latex
    \usepackage{grffile} % extends the file name processing of package graphics 
                         % to support a larger range 
    % The hyperref package gives us a pdf with properly built
    % internal navigation ('pdf bookmarks' for the table of contents,
    % internal cross-reference links, web links for URLs, etc.)
    \usepackage{hyperref}
    \usepackage{longtable} % longtable support required by pandoc >1.10
    \usepackage{booktabs}  % table support for pandoc > 1.12.2
    \usepackage[inline]{enumitem} % IRkernel/repr support (it uses the enumerate* environment)
    \usepackage[normalem]{ulem} % ulem is needed to support strikethroughs (\sout)
                                % normalem makes italics be italics, not underlines
    

    
    
    % Colors for the hyperref package
    \definecolor{urlcolor}{rgb}{0,.145,.698}
    \definecolor{linkcolor}{rgb}{.71,0.21,0.01}
    \definecolor{citecolor}{rgb}{.12,.54,.11}

    % ANSI colors
    \definecolor{ansi-black}{HTML}{3E424D}
    \definecolor{ansi-black-intense}{HTML}{282C36}
    \definecolor{ansi-red}{HTML}{E75C58}
    \definecolor{ansi-red-intense}{HTML}{B22B31}
    \definecolor{ansi-green}{HTML}{00A250}
    \definecolor{ansi-green-intense}{HTML}{007427}
    \definecolor{ansi-yellow}{HTML}{DDB62B}
    \definecolor{ansi-yellow-intense}{HTML}{B27D12}
    \definecolor{ansi-blue}{HTML}{208FFB}
    \definecolor{ansi-blue-intense}{HTML}{0065CA}
    \definecolor{ansi-magenta}{HTML}{D160C4}
    \definecolor{ansi-magenta-intense}{HTML}{A03196}
    \definecolor{ansi-cyan}{HTML}{60C6C8}
    \definecolor{ansi-cyan-intense}{HTML}{258F8F}
    \definecolor{ansi-white}{HTML}{C5C1B4}
    \definecolor{ansi-white-intense}{HTML}{A1A6B2}

    % commands and environments needed by pandoc snippets
    % extracted from the output of `pandoc -s`
    \providecommand{\tightlist}{%
      \setlength{\itemsep}{0pt}\setlength{\parskip}{0pt}}
    \DefineVerbatimEnvironment{Highlighting}{Verbatim}{commandchars=\\\{\}}
    % Add ',fontsize=\small' for more characters per line
    \newenvironment{Shaded}{}{}
    \newcommand{\KeywordTok}[1]{\textcolor[rgb]{0.00,0.44,0.13}{\textbf{{#1}}}}
    \newcommand{\DataTypeTok}[1]{\textcolor[rgb]{0.56,0.13,0.00}{{#1}}}
    \newcommand{\DecValTok}[1]{\textcolor[rgb]{0.25,0.63,0.44}{{#1}}}
    \newcommand{\BaseNTok}[1]{\textcolor[rgb]{0.25,0.63,0.44}{{#1}}}
    \newcommand{\FloatTok}[1]{\textcolor[rgb]{0.25,0.63,0.44}{{#1}}}
    \newcommand{\CharTok}[1]{\textcolor[rgb]{0.25,0.44,0.63}{{#1}}}
    \newcommand{\StringTok}[1]{\textcolor[rgb]{0.25,0.44,0.63}{{#1}}}
    \newcommand{\CommentTok}[1]{\textcolor[rgb]{0.38,0.63,0.69}{\textit{{#1}}}}
    \newcommand{\OtherTok}[1]{\textcolor[rgb]{0.00,0.44,0.13}{{#1}}}
    \newcommand{\AlertTok}[1]{\textcolor[rgb]{1.00,0.00,0.00}{\textbf{{#1}}}}
    \newcommand{\FunctionTok}[1]{\textcolor[rgb]{0.02,0.16,0.49}{{#1}}}
    \newcommand{\RegionMarkerTok}[1]{{#1}}
    \newcommand{\ErrorTok}[1]{\textcolor[rgb]{1.00,0.00,0.00}{\textbf{{#1}}}}
    \newcommand{\NormalTok}[1]{{#1}}
    
    % Additional commands for more recent versions of Pandoc
    \newcommand{\ConstantTok}[1]{\textcolor[rgb]{0.53,0.00,0.00}{{#1}}}
    \newcommand{\SpecialCharTok}[1]{\textcolor[rgb]{0.25,0.44,0.63}{{#1}}}
    \newcommand{\VerbatimStringTok}[1]{\textcolor[rgb]{0.25,0.44,0.63}{{#1}}}
    \newcommand{\SpecialStringTok}[1]{\textcolor[rgb]{0.73,0.40,0.53}{{#1}}}
    \newcommand{\ImportTok}[1]{{#1}}
    \newcommand{\DocumentationTok}[1]{\textcolor[rgb]{0.73,0.13,0.13}{\textit{{#1}}}}
    \newcommand{\AnnotationTok}[1]{\textcolor[rgb]{0.38,0.63,0.69}{\textbf{\textit{{#1}}}}}
    \newcommand{\CommentVarTok}[1]{\textcolor[rgb]{0.38,0.63,0.69}{\textbf{\textit{{#1}}}}}
    \newcommand{\VariableTok}[1]{\textcolor[rgb]{0.10,0.09,0.49}{{#1}}}
    \newcommand{\ControlFlowTok}[1]{\textcolor[rgb]{0.00,0.44,0.13}{\textbf{{#1}}}}
    \newcommand{\OperatorTok}[1]{\textcolor[rgb]{0.40,0.40,0.40}{{#1}}}
    \newcommand{\BuiltInTok}[1]{{#1}}
    \newcommand{\ExtensionTok}[1]{{#1}}
    \newcommand{\PreprocessorTok}[1]{\textcolor[rgb]{0.74,0.48,0.00}{{#1}}}
    \newcommand{\AttributeTok}[1]{\textcolor[rgb]{0.49,0.56,0.16}{{#1}}}
    \newcommand{\InformationTok}[1]{\textcolor[rgb]{0.38,0.63,0.69}{\textbf{\textit{{#1}}}}}
    \newcommand{\WarningTok}[1]{\textcolor[rgb]{0.38,0.63,0.69}{\textbf{\textit{{#1}}}}}
    
    
    % Define a nice break command that doesn't care if a line doesn't already
    % exist.
    \def\br{\hspace*{\fill} \\* }
    % Math Jax compatability definitions
    \def\gt{>}
    \def\lt{<}
    % Document parameters
    \title{Learin\_Python-Round2}
    
    
    

    % Pygments definitions
    
\makeatletter
\def\PY@reset{\let\PY@it=\relax \let\PY@bf=\relax%
    \let\PY@ul=\relax \let\PY@tc=\relax%
    \let\PY@bc=\relax \let\PY@ff=\relax}
\def\PY@tok#1{\csname PY@tok@#1\endcsname}
\def\PY@toks#1+{\ifx\relax#1\empty\else%
    \PY@tok{#1}\expandafter\PY@toks\fi}
\def\PY@do#1{\PY@bc{\PY@tc{\PY@ul{%
    \PY@it{\PY@bf{\PY@ff{#1}}}}}}}
\def\PY#1#2{\PY@reset\PY@toks#1+\relax+\PY@do{#2}}

\expandafter\def\csname PY@tok@w\endcsname{\def\PY@tc##1{\textcolor[rgb]{0.73,0.73,0.73}{##1}}}
\expandafter\def\csname PY@tok@c\endcsname{\let\PY@it=\textit\def\PY@tc##1{\textcolor[rgb]{0.25,0.50,0.50}{##1}}}
\expandafter\def\csname PY@tok@cp\endcsname{\def\PY@tc##1{\textcolor[rgb]{0.74,0.48,0.00}{##1}}}
\expandafter\def\csname PY@tok@k\endcsname{\let\PY@bf=\textbf\def\PY@tc##1{\textcolor[rgb]{0.00,0.50,0.00}{##1}}}
\expandafter\def\csname PY@tok@kp\endcsname{\def\PY@tc##1{\textcolor[rgb]{0.00,0.50,0.00}{##1}}}
\expandafter\def\csname PY@tok@kt\endcsname{\def\PY@tc##1{\textcolor[rgb]{0.69,0.00,0.25}{##1}}}
\expandafter\def\csname PY@tok@o\endcsname{\def\PY@tc##1{\textcolor[rgb]{0.40,0.40,0.40}{##1}}}
\expandafter\def\csname PY@tok@ow\endcsname{\let\PY@bf=\textbf\def\PY@tc##1{\textcolor[rgb]{0.67,0.13,1.00}{##1}}}
\expandafter\def\csname PY@tok@nb\endcsname{\def\PY@tc##1{\textcolor[rgb]{0.00,0.50,0.00}{##1}}}
\expandafter\def\csname PY@tok@nf\endcsname{\def\PY@tc##1{\textcolor[rgb]{0.00,0.00,1.00}{##1}}}
\expandafter\def\csname PY@tok@nc\endcsname{\let\PY@bf=\textbf\def\PY@tc##1{\textcolor[rgb]{0.00,0.00,1.00}{##1}}}
\expandafter\def\csname PY@tok@nn\endcsname{\let\PY@bf=\textbf\def\PY@tc##1{\textcolor[rgb]{0.00,0.00,1.00}{##1}}}
\expandafter\def\csname PY@tok@ne\endcsname{\let\PY@bf=\textbf\def\PY@tc##1{\textcolor[rgb]{0.82,0.25,0.23}{##1}}}
\expandafter\def\csname PY@tok@nv\endcsname{\def\PY@tc##1{\textcolor[rgb]{0.10,0.09,0.49}{##1}}}
\expandafter\def\csname PY@tok@no\endcsname{\def\PY@tc##1{\textcolor[rgb]{0.53,0.00,0.00}{##1}}}
\expandafter\def\csname PY@tok@nl\endcsname{\def\PY@tc##1{\textcolor[rgb]{0.63,0.63,0.00}{##1}}}
\expandafter\def\csname PY@tok@ni\endcsname{\let\PY@bf=\textbf\def\PY@tc##1{\textcolor[rgb]{0.60,0.60,0.60}{##1}}}
\expandafter\def\csname PY@tok@na\endcsname{\def\PY@tc##1{\textcolor[rgb]{0.49,0.56,0.16}{##1}}}
\expandafter\def\csname PY@tok@nt\endcsname{\let\PY@bf=\textbf\def\PY@tc##1{\textcolor[rgb]{0.00,0.50,0.00}{##1}}}
\expandafter\def\csname PY@tok@nd\endcsname{\def\PY@tc##1{\textcolor[rgb]{0.67,0.13,1.00}{##1}}}
\expandafter\def\csname PY@tok@s\endcsname{\def\PY@tc##1{\textcolor[rgb]{0.73,0.13,0.13}{##1}}}
\expandafter\def\csname PY@tok@sd\endcsname{\let\PY@it=\textit\def\PY@tc##1{\textcolor[rgb]{0.73,0.13,0.13}{##1}}}
\expandafter\def\csname PY@tok@si\endcsname{\let\PY@bf=\textbf\def\PY@tc##1{\textcolor[rgb]{0.73,0.40,0.53}{##1}}}
\expandafter\def\csname PY@tok@se\endcsname{\let\PY@bf=\textbf\def\PY@tc##1{\textcolor[rgb]{0.73,0.40,0.13}{##1}}}
\expandafter\def\csname PY@tok@sr\endcsname{\def\PY@tc##1{\textcolor[rgb]{0.73,0.40,0.53}{##1}}}
\expandafter\def\csname PY@tok@ss\endcsname{\def\PY@tc##1{\textcolor[rgb]{0.10,0.09,0.49}{##1}}}
\expandafter\def\csname PY@tok@sx\endcsname{\def\PY@tc##1{\textcolor[rgb]{0.00,0.50,0.00}{##1}}}
\expandafter\def\csname PY@tok@m\endcsname{\def\PY@tc##1{\textcolor[rgb]{0.40,0.40,0.40}{##1}}}
\expandafter\def\csname PY@tok@gh\endcsname{\let\PY@bf=\textbf\def\PY@tc##1{\textcolor[rgb]{0.00,0.00,0.50}{##1}}}
\expandafter\def\csname PY@tok@gu\endcsname{\let\PY@bf=\textbf\def\PY@tc##1{\textcolor[rgb]{0.50,0.00,0.50}{##1}}}
\expandafter\def\csname PY@tok@gd\endcsname{\def\PY@tc##1{\textcolor[rgb]{0.63,0.00,0.00}{##1}}}
\expandafter\def\csname PY@tok@gi\endcsname{\def\PY@tc##1{\textcolor[rgb]{0.00,0.63,0.00}{##1}}}
\expandafter\def\csname PY@tok@gr\endcsname{\def\PY@tc##1{\textcolor[rgb]{1.00,0.00,0.00}{##1}}}
\expandafter\def\csname PY@tok@ge\endcsname{\let\PY@it=\textit}
\expandafter\def\csname PY@tok@gs\endcsname{\let\PY@bf=\textbf}
\expandafter\def\csname PY@tok@gp\endcsname{\let\PY@bf=\textbf\def\PY@tc##1{\textcolor[rgb]{0.00,0.00,0.50}{##1}}}
\expandafter\def\csname PY@tok@go\endcsname{\def\PY@tc##1{\textcolor[rgb]{0.53,0.53,0.53}{##1}}}
\expandafter\def\csname PY@tok@gt\endcsname{\def\PY@tc##1{\textcolor[rgb]{0.00,0.27,0.87}{##1}}}
\expandafter\def\csname PY@tok@err\endcsname{\def\PY@bc##1{\setlength{\fboxsep}{0pt}\fcolorbox[rgb]{1.00,0.00,0.00}{1,1,1}{\strut ##1}}}
\expandafter\def\csname PY@tok@kc\endcsname{\let\PY@bf=\textbf\def\PY@tc##1{\textcolor[rgb]{0.00,0.50,0.00}{##1}}}
\expandafter\def\csname PY@tok@kd\endcsname{\let\PY@bf=\textbf\def\PY@tc##1{\textcolor[rgb]{0.00,0.50,0.00}{##1}}}
\expandafter\def\csname PY@tok@kn\endcsname{\let\PY@bf=\textbf\def\PY@tc##1{\textcolor[rgb]{0.00,0.50,0.00}{##1}}}
\expandafter\def\csname PY@tok@kr\endcsname{\let\PY@bf=\textbf\def\PY@tc##1{\textcolor[rgb]{0.00,0.50,0.00}{##1}}}
\expandafter\def\csname PY@tok@bp\endcsname{\def\PY@tc##1{\textcolor[rgb]{0.00,0.50,0.00}{##1}}}
\expandafter\def\csname PY@tok@fm\endcsname{\def\PY@tc##1{\textcolor[rgb]{0.00,0.00,1.00}{##1}}}
\expandafter\def\csname PY@tok@vc\endcsname{\def\PY@tc##1{\textcolor[rgb]{0.10,0.09,0.49}{##1}}}
\expandafter\def\csname PY@tok@vg\endcsname{\def\PY@tc##1{\textcolor[rgb]{0.10,0.09,0.49}{##1}}}
\expandafter\def\csname PY@tok@vi\endcsname{\def\PY@tc##1{\textcolor[rgb]{0.10,0.09,0.49}{##1}}}
\expandafter\def\csname PY@tok@vm\endcsname{\def\PY@tc##1{\textcolor[rgb]{0.10,0.09,0.49}{##1}}}
\expandafter\def\csname PY@tok@sa\endcsname{\def\PY@tc##1{\textcolor[rgb]{0.73,0.13,0.13}{##1}}}
\expandafter\def\csname PY@tok@sb\endcsname{\def\PY@tc##1{\textcolor[rgb]{0.73,0.13,0.13}{##1}}}
\expandafter\def\csname PY@tok@sc\endcsname{\def\PY@tc##1{\textcolor[rgb]{0.73,0.13,0.13}{##1}}}
\expandafter\def\csname PY@tok@dl\endcsname{\def\PY@tc##1{\textcolor[rgb]{0.73,0.13,0.13}{##1}}}
\expandafter\def\csname PY@tok@s2\endcsname{\def\PY@tc##1{\textcolor[rgb]{0.73,0.13,0.13}{##1}}}
\expandafter\def\csname PY@tok@sh\endcsname{\def\PY@tc##1{\textcolor[rgb]{0.73,0.13,0.13}{##1}}}
\expandafter\def\csname PY@tok@s1\endcsname{\def\PY@tc##1{\textcolor[rgb]{0.73,0.13,0.13}{##1}}}
\expandafter\def\csname PY@tok@mb\endcsname{\def\PY@tc##1{\textcolor[rgb]{0.40,0.40,0.40}{##1}}}
\expandafter\def\csname PY@tok@mf\endcsname{\def\PY@tc##1{\textcolor[rgb]{0.40,0.40,0.40}{##1}}}
\expandafter\def\csname PY@tok@mh\endcsname{\def\PY@tc##1{\textcolor[rgb]{0.40,0.40,0.40}{##1}}}
\expandafter\def\csname PY@tok@mi\endcsname{\def\PY@tc##1{\textcolor[rgb]{0.40,0.40,0.40}{##1}}}
\expandafter\def\csname PY@tok@il\endcsname{\def\PY@tc##1{\textcolor[rgb]{0.40,0.40,0.40}{##1}}}
\expandafter\def\csname PY@tok@mo\endcsname{\def\PY@tc##1{\textcolor[rgb]{0.40,0.40,0.40}{##1}}}
\expandafter\def\csname PY@tok@ch\endcsname{\let\PY@it=\textit\def\PY@tc##1{\textcolor[rgb]{0.25,0.50,0.50}{##1}}}
\expandafter\def\csname PY@tok@cm\endcsname{\let\PY@it=\textit\def\PY@tc##1{\textcolor[rgb]{0.25,0.50,0.50}{##1}}}
\expandafter\def\csname PY@tok@cpf\endcsname{\let\PY@it=\textit\def\PY@tc##1{\textcolor[rgb]{0.25,0.50,0.50}{##1}}}
\expandafter\def\csname PY@tok@c1\endcsname{\let\PY@it=\textit\def\PY@tc##1{\textcolor[rgb]{0.25,0.50,0.50}{##1}}}
\expandafter\def\csname PY@tok@cs\endcsname{\let\PY@it=\textit\def\PY@tc##1{\textcolor[rgb]{0.25,0.50,0.50}{##1}}}

\def\PYZbs{\char`\\}
\def\PYZus{\char`\_}
\def\PYZob{\char`\{}
\def\PYZcb{\char`\}}
\def\PYZca{\char`\^}
\def\PYZam{\char`\&}
\def\PYZlt{\char`\<}
\def\PYZgt{\char`\>}
\def\PYZsh{\char`\#}
\def\PYZpc{\char`\%}
\def\PYZdl{\char`\$}
\def\PYZhy{\char`\-}
\def\PYZsq{\char`\'}
\def\PYZdq{\char`\"}
\def\PYZti{\char`\~}
% for compatibility with earlier versions
\def\PYZat{@}
\def\PYZlb{[}
\def\PYZrb{]}
\makeatother


    % Exact colors from NB
    \definecolor{incolor}{rgb}{0.0, 0.0, 0.5}
    \definecolor{outcolor}{rgb}{0.545, 0.0, 0.0}



    
    % Prevent overflowing lines due to hard-to-break entities
    \sloppy 
    % Setup hyperref package
    \hypersetup{
      breaklinks=true,  % so long urls are correctly broken across lines
      colorlinks=true,
      urlcolor=urlcolor,
      linkcolor=linkcolor,
      citecolor=citecolor,
      }
    % Slightly bigger margins than the latex defaults
    
    \geometry{verbose,tmargin=1in,bmargin=1in,lmargin=1in,rmargin=1in}
    
    

    \begin{document}
    
    
    \maketitle
    
    

    
    \section{Teaching Yourself Python
Basics}\label{teaching-yourself-python-basics}

\subsection{Intro}\label{intro}

The Coursera course taught by the University of Michigan wasn't really
doing it for me. So I decided to start from scratch with this handy
notebook, where I will lay down the Python basics to remind myself (and
whoever else may be interested) how things work. Ideally this will help
in the longrun when taking the upper level courses for the University of
Michigan's Data Science with Python (especially because they don't go
into too much detail in the course). My work here will be based off of
the lessons on the website www.learnpython.org. So without further ado,
let's get started!

\textbf{Aside}:

For new users checking this notebook out, if you would like to play with
it in dark mode (rather than the bright default Jupyter offers us), run
the below cell and reboot Jupyter.

    \begin{Verbatim}[commandchars=\\\{\}]
{\color{incolor}In [{\color{incolor} }]:} \PY{o}{!}pip install jupyterthemes
        \PY{o}{!}jt \PYZhy{}t chesterish
\end{Verbatim}


    \subsection{Learning the Basics}\label{learning-the-basics}

\subsubsection{Hello World!}\label{hello-world}

Let's start from the absolute bottom up so that absolutely no stone is
left unturned. We'll do this by opening up our window, looking outside,
and giving a hearty "Hello World!"

    \begin{Verbatim}[commandchars=\\\{\}]
{\color{incolor}In [{\color{incolor}13}]:} \PY{n+nb}{print}\PY{p}{(}\PY{l+s+s2}{\PYZdq{}}\PY{l+s+s2}{Hello World!}\PY{l+s+s2}{\PYZdq{}}\PY{p}{)}
\end{Verbatim}


    \begin{Verbatim}[commandchars=\\\{\}]
Hello World!

    \end{Verbatim}

    Easy peasy. That print statement will do exactly what it says; print out
what you put inside. Now unlike R, we don't need braces or anything
around things like if statements. Instead they just need to be indented:

    \begin{Verbatim}[commandchars=\\\{\}]
{\color{incolor}In [{\color{incolor}14}]:} \PY{n}{x} \PY{o}{=} \PY{l+m+mi}{1}
         \PY{k}{if} \PY{n}{x} \PY{o}{==} \PY{l+m+mi}{1}\PY{p}{:}
             \PY{c+c1}{\PYZsh{} indented four spaces}
             \PY{n+nb}{print}\PY{p}{(}\PY{l+s+s2}{\PYZdq{}}\PY{l+s+s2}{x is 1.}\PY{l+s+s2}{\PYZdq{}}\PY{p}{)}
\end{Verbatim}


    \begin{Verbatim}[commandchars=\\\{\}]
x is 1.

    \end{Verbatim}

    \subsubsection{Variables and Types}\label{variables-and-types}

Python is object oriented, so luckily this part is pretty straight
forward. I'll try to speed through this part.

    \begin{Verbatim}[commandchars=\\\{\}]
{\color{incolor}In [{\color{incolor}15}]:} \PY{c+c1}{\PYZsh{}\PYZsh{}\PYZsh{} Numbers:}
         
         \PY{n}{myint} \PY{o}{=} \PY{l+m+mi}{5}
         \PY{n}{myfloat} \PY{o}{=} \PY{l+m+mf}{5.0} \PY{c+c1}{\PYZsh{}or}
         \PY{n}{myfloat2\PYZus{}thefloatening} \PY{o}{=} \PY{n+nb}{float}\PY{p}{(}\PY{l+m+mi}{5}\PY{p}{)}
         
         \PY{n+nb}{print}\PY{p}{(}\PY{n+nb}{type}\PY{p}{(}\PY{n}{myint}\PY{p}{)}\PY{p}{)}
         \PY{n+nb}{print}\PY{p}{(}\PY{n+nb}{type}\PY{p}{(}\PY{n}{myfloat}\PY{p}{)}\PY{p}{)}
         \PY{n+nb}{print}\PY{p}{(}\PY{n+nb}{type}\PY{p}{(}\PY{n}{myfloat2\PYZus{}thefloatening}\PY{p}{)}\PY{p}{)}
\end{Verbatim}


    \begin{Verbatim}[commandchars=\\\{\}]
<class 'int'>
<class 'float'>
<class 'float'>

    \end{Verbatim}

    \begin{Verbatim}[commandchars=\\\{\}]
{\color{incolor}In [{\color{incolor}16}]:} \PY{c+c1}{\PYZsh{}\PYZsh{}\PYZsh{} Strings:}
         
         \PY{n}{howdy} \PY{o}{=} \PY{l+s+s1}{\PYZsq{}}\PY{l+s+s1}{hello!}\PY{l+s+s1}{\PYZsq{}}
         \PY{n}{nihao} \PY{o}{=} \PY{l+s+s2}{\PYZdq{}}\PY{l+s+s2}{hello!}\PY{l+s+s2}{\PYZdq{}}
         
         \PY{c+c1}{\PYZsh{} Notice both \PYZsq{}\PYZsq{} and \PYZdq{}\PYZdq{} will work when making strings. }
         \PY{c+c1}{\PYZsh{} Just be aware to use \PYZdq{}\PYZdq{} if you have apostrophes.}
         
         \PY{n+nb}{print}\PY{p}{(}\PY{n}{howdy}\PY{p}{)}
         \PY{n+nb}{print}\PY{p}{(}\PY{n}{nihao}\PY{p}{)} 
\end{Verbatim}


    \begin{Verbatim}[commandchars=\\\{\}]
hello!
hello!

    \end{Verbatim}

    \begin{Verbatim}[commandchars=\\\{\}]
{\color{incolor}In [{\color{incolor}17}]:} \PY{c+c1}{\PYZsh{}\PYZsh{}\PYZsh{} None:}
         \PY{n}{depression} \PY{o}{=} \PY{k+kc}{None}
         \PY{c+c1}{\PYZsh{} Explains itself}
         \PY{n+nb}{type}\PY{p}{(}\PY{n}{depression}\PY{p}{)}
\end{Verbatim}


\begin{Verbatim}[commandchars=\\\{\}]
{\color{outcolor}Out[{\color{outcolor}17}]:} NoneType
\end{Verbatim}
            
    \begin{Verbatim}[commandchars=\\\{\}]
{\color{incolor}In [{\color{incolor}18}]:} \PY{c+c1}{\PYZsh{}\PYZsh{}\PYZsh{} Variable Operations:}
         
         \PY{n}{one} \PY{o}{=} \PY{l+m+mi}{1}
         \PY{n}{two} \PY{o}{=} \PY{l+m+mi}{2}
         \PY{n}{three} \PY{o}{=} \PY{n}{one} \PY{o}{+} \PY{n}{two}
         \PY{n+nb}{print}\PY{p}{(}\PY{n}{three}\PY{p}{)}
         
         \PY{n}{hello} \PY{o}{=} \PY{l+s+s2}{\PYZdq{}}\PY{l+s+s2}{hello}\PY{l+s+s2}{\PYZdq{}}
         \PY{n}{world} \PY{o}{=} \PY{l+s+s2}{\PYZdq{}}\PY{l+s+s2}{world}\PY{l+s+s2}{\PYZdq{}}
         \PY{n}{helloworld} \PY{o}{=} \PY{n}{hello} \PY{o}{+} \PY{l+s+s2}{\PYZdq{}}\PY{l+s+s2}{ }\PY{l+s+s2}{\PYZdq{}} \PY{o}{+} \PY{n}{world}
         \PY{n+nb}{print}\PY{p}{(}\PY{n}{helloworld}\PY{p}{)}
         
         \PY{c+c1}{\PYZsh{} We can even assign multiple variables at once}
         
         \PY{n}{a}\PY{p}{,} \PY{n}{b} \PY{o}{=} \PY{l+m+mi}{3}\PY{p}{,} \PY{l+m+mi}{4}
         \PY{n+nb}{print}\PY{p}{(}\PY{n}{a}\PY{p}{,}\PY{n}{b}\PY{p}{)}
\end{Verbatim}


    \begin{Verbatim}[commandchars=\\\{\}]
3
hello world
3 4

    \end{Verbatim}

    It's important to note that mixing operations between numbers and
strings won't work.

    \begin{Verbatim}[commandchars=\\\{\}]
{\color{incolor}In [{\color{incolor}19}]:} \PY{c+c1}{\PYZsh{} This will not work!}
         \PY{n}{one} \PY{o}{=} \PY{l+m+mi}{1}
         \PY{n}{two} \PY{o}{=} \PY{l+m+mi}{2}
         \PY{n}{hello} \PY{o}{=} \PY{l+s+s2}{\PYZdq{}}\PY{l+s+s2}{hello}\PY{l+s+s2}{\PYZdq{}}
         
         \PY{n+nb}{print}\PY{p}{(}\PY{n}{one} \PY{o}{+} \PY{n}{two} \PY{o}{+} \PY{n}{hello}\PY{p}{)}
\end{Verbatim}


    \begin{Verbatim}[commandchars=\\\{\}]

        ---------------------------------------------------------------------------

        TypeError                                 Traceback (most recent call last)

        <ipython-input-19-fa035bb013ee> in <module>()
          4 hello = "hello"
          5 
    ----> 6 print(one + two + hello)
    

        TypeError: unsupported operand type(s) for +: 'int' and 'str'

    \end{Verbatim}

    That being said, we can convert numbers into strings to accomplish this
task!

    \begin{Verbatim}[commandchars=\\\{\}]
{\color{incolor}In [{\color{incolor}20}]:} \PY{c+c1}{\PYZsh{} This will work!}
         \PY{n}{one} \PY{o}{=} \PY{l+m+mi}{1}
         \PY{n}{two} \PY{o}{=} \PY{l+m+mi}{2}
         \PY{n}{three} \PY{o}{=} \PY{n+nb}{str}\PY{p}{(}\PY{n}{one} \PY{o}{+} \PY{n}{two}\PY{p}{)}
         
         \PY{n+nb}{print}\PY{p}{(}\PY{n}{three}\PY{p}{)}
         \PY{n+nb}{print}\PY{p}{(}\PY{l+s+s1}{\PYZsq{}}\PY{l+s+s1}{h}\PY{l+s+s1}{\PYZsq{}} \PY{o}{+} \PY{n}{three} \PY{o}{+} \PY{l+s+s1}{\PYZsq{}}\PY{l+s+s1}{llo}\PY{l+s+s1}{\PYZsq{}}\PY{p}{)}
\end{Verbatim}


    \begin{Verbatim}[commandchars=\\\{\}]
3
h3llo

    \end{Verbatim}

    \paragraph{Exercise}\label{exercise}

The target of this exercise is to create a string, an integer, and a
floating point number. The string should be named \texttt{mystring} and
should contain the word "hello". The floating point number should be
named \texttt{myfloat} and should contain the number 10.0, and the
integer should be named \texttt{myint} and should contain the number 20.
Easy right? Solution is below:

    \begin{Verbatim}[commandchars=\\\{\}]
{\color{incolor}In [{\color{incolor}21}]:} \PY{n}{mystring} \PY{o}{=} \PY{l+s+s1}{\PYZsq{}}\PY{l+s+s1}{hello}\PY{l+s+s1}{\PYZsq{}}
         \PY{n}{myfloat} \PY{o}{=} \PY{n+nb}{float}\PY{p}{(}\PY{l+m+mi}{10}\PY{p}{)}
         \PY{n}{myint} \PY{o}{=} \PY{l+m+mi}{20}
         
         \PY{c+c1}{\PYZsh{} testing code}
         \PY{k}{if} \PY{n}{mystring} \PY{o}{==} \PY{l+s+s2}{\PYZdq{}}\PY{l+s+s2}{hello}\PY{l+s+s2}{\PYZdq{}}\PY{p}{:}
             \PY{n+nb}{print}\PY{p}{(}\PY{l+s+s2}{\PYZdq{}}\PY{l+s+s2}{String: }\PY{l+s+si}{\PYZpc{}s}\PY{l+s+s2}{\PYZdq{}} \PY{o}{\PYZpc{}} \PY{n}{mystring}\PY{p}{)}
         \PY{k}{if} \PY{n+nb}{isinstance}\PY{p}{(}\PY{n}{myfloat}\PY{p}{,} \PY{n+nb}{float}\PY{p}{)} \PY{o+ow}{and} \PY{n}{myfloat} \PY{o}{==} \PY{l+m+mf}{10.0}\PY{p}{:}
             \PY{n+nb}{print}\PY{p}{(}\PY{l+s+s2}{\PYZdq{}}\PY{l+s+s2}{Float: }\PY{l+s+si}{\PYZpc{}f}\PY{l+s+s2}{\PYZdq{}} \PY{o}{\PYZpc{}} \PY{n}{myfloat}\PY{p}{)}
         \PY{k}{if} \PY{n+nb}{isinstance}\PY{p}{(}\PY{n}{myint}\PY{p}{,} \PY{n+nb}{int}\PY{p}{)} \PY{o+ow}{and} \PY{n}{myint} \PY{o}{==} \PY{l+m+mi}{20}\PY{p}{:}
             \PY{n+nb}{print}\PY{p}{(}\PY{l+s+s2}{\PYZdq{}}\PY{l+s+s2}{Integer: }\PY{l+s+si}{\PYZpc{}d}\PY{l+s+s2}{\PYZdq{}} \PY{o}{\PYZpc{}} \PY{n}{myint}\PY{p}{)}
\end{Verbatim}


    \begin{Verbatim}[commandchars=\\\{\}]
String: hello
Float: 10.000000
Integer: 20

    \end{Verbatim}

    \subsubsection{Lists}\label{lists}

Lists are sort of like vectors in R or arrays in other languages. They
contain any type of variable, and can contain as many variables as your
PC can handle. Here's how to build an easy starter list.

    \begin{Verbatim}[commandchars=\\\{\}]
{\color{incolor}In [{\color{incolor}22}]:} \PY{n}{mylist} \PY{o}{=} \PY{p}{[}\PY{p}{]}
         \PY{n}{mylist}\PY{o}{.}\PY{n}{append}\PY{p}{(}\PY{l+m+mi}{1}\PY{p}{)}
         \PY{n}{mylist}\PY{o}{.}\PY{n}{append}\PY{p}{(}\PY{l+m+mi}{2}\PY{p}{)}
         \PY{n}{mylist}\PY{o}{.}\PY{n}{append}\PY{p}{(}\PY{l+m+mi}{3}\PY{p}{)}
         \PY{n+nb}{print}\PY{p}{(}\PY{n}{mylist}\PY{p}{[}\PY{l+m+mi}{0}\PY{p}{]}\PY{p}{)} \PY{c+c1}{\PYZsh{} prints 1}
         \PY{n+nb}{print}\PY{p}{(}\PY{n}{mylist}\PY{p}{[}\PY{l+m+mi}{1}\PY{p}{]}\PY{p}{)} \PY{c+c1}{\PYZsh{} prints 2}
         \PY{n+nb}{print}\PY{p}{(}\PY{n}{mylist}\PY{p}{[}\PY{l+m+mi}{2}\PY{p}{]}\PY{p}{)} \PY{c+c1}{\PYZsh{} prints 3}
\end{Verbatim}


    \begin{Verbatim}[commandchars=\\\{\}]
1
2
3

    \end{Verbatim}

    \begin{Verbatim}[commandchars=\\\{\}]
{\color{incolor}In [{\color{incolor}23}]:} \PY{c+c1}{\PYZsh{} prints out 1,2,3}
         \PY{k}{for} \PY{n}{x} \PY{o+ow}{in} \PY{n}{mylist}\PY{p}{:}
             \PY{n+nb}{print}\PY{p}{(}\PY{n}{x}\PY{p}{)}
\end{Verbatim}


    \begin{Verbatim}[commandchars=\\\{\}]
1
2
3

    \end{Verbatim}

    You can also make a list in one single lined statement such as the
following.

    \begin{Verbatim}[commandchars=\\\{\}]
{\color{incolor}In [{\color{incolor}24}]:} \PY{n}{mylist} \PY{o}{=} \PY{p}{[}\PY{l+m+mi}{1}\PY{p}{,}\PY{l+m+mi}{2}\PY{p}{,}\PY{l+m+mi}{3}\PY{p}{]}
\end{Verbatim}


    Accessing an index which does not exist generates an exception (an
error).

    \begin{Verbatim}[commandchars=\\\{\}]
{\color{incolor}In [{\color{incolor}25}]:} \PY{n}{mylist} \PY{o}{=} \PY{p}{[}\PY{l+m+mi}{1}\PY{p}{,}\PY{l+m+mi}{2}\PY{p}{,}\PY{l+m+mi}{3}\PY{p}{]}
         \PY{n+nb}{print}\PY{p}{(}\PY{n}{mylist}\PY{p}{[}\PY{l+m+mi}{10}\PY{p}{]}\PY{p}{)}
\end{Verbatim}


    \begin{Verbatim}[commandchars=\\\{\}]

        ---------------------------------------------------------------------------

        IndexError                                Traceback (most recent call last)

        <ipython-input-25-ac9eeac6db06> in <module>()
          1 mylist = [1,2,3]
    ----> 2 print(mylist[10])
    

        IndexError: list index out of range

    \end{Verbatim}

    \paragraph{Exercise}\label{exercise}

In this exercise, you will need to add numbers and strings to the
correct lists using the "append" list method. You must add the number 3
to the "numbers" list, and the word 'world' to the strings variable.

You will also have to fill in the variable second\_name with the second
name in the names list, using the brackets operator \texttt{{[}{]}}.
\textbf{Note that the index is zero-based, so if you want to access the
second item in the list, its index will be 1}.

    \begin{Verbatim}[commandchars=\\\{\}]
{\color{incolor}In [{\color{incolor}26}]:} \PY{n}{numbers} \PY{o}{=} \PY{p}{[}\PY{l+m+mi}{1}\PY{p}{,}\PY{l+m+mi}{2}\PY{p}{]}
         \PY{n}{strings} \PY{o}{=} \PY{p}{[}\PY{l+s+s1}{\PYZsq{}}\PY{l+s+s1}{hello}\PY{l+s+s1}{\PYZsq{}}\PY{p}{]}
         \PY{n}{names} \PY{o}{=} \PY{p}{[}\PY{l+s+s2}{\PYZdq{}}\PY{l+s+s2}{John}\PY{l+s+s2}{\PYZdq{}}\PY{p}{,} \PY{l+s+s2}{\PYZdq{}}\PY{l+s+s2}{Eric}\PY{l+s+s2}{\PYZdq{}}\PY{p}{,} \PY{l+s+s2}{\PYZdq{}}\PY{l+s+s2}{Jessica}\PY{l+s+s2}{\PYZdq{}}\PY{p}{]}
         
         \PY{c+c1}{\PYZsh{} write your code here}
         \PY{n}{second\PYZus{}name} \PY{o}{=} \PY{n}{names}\PY{p}{[}\PY{l+m+mi}{1}\PY{p}{]}
         \PY{n}{numbers}\PY{o}{.}\PY{n}{append}\PY{p}{(}\PY{l+m+mi}{3}\PY{p}{)}
         \PY{n}{strings}\PY{o}{.}\PY{n}{append}\PY{p}{(}\PY{l+s+s1}{\PYZsq{}}\PY{l+s+s1}{world}\PY{l+s+s1}{\PYZsq{}}\PY{p}{)}
         
         \PY{c+c1}{\PYZsh{} this code should write out the filled arrays }
         \PY{c+c1}{\PYZsh{} and the second name in the names list (Eric).}
         \PY{n+nb}{print}\PY{p}{(}\PY{n}{numbers}\PY{p}{)}
         \PY{n+nb}{print}\PY{p}{(}\PY{n}{strings}\PY{p}{)}
         \PY{n+nb}{print}\PY{p}{(}\PY{l+s+s2}{\PYZdq{}}\PY{l+s+s2}{The second name on the names list is }\PY{l+s+si}{\PYZpc{}s}\PY{l+s+s2}{\PYZdq{}} \PY{o}{\PYZpc{}} \PY{n}{second\PYZus{}name}\PY{p}{)}
\end{Verbatim}


    \begin{Verbatim}[commandchars=\\\{\}]
[1, 2, 3]
['hello', 'world']
The second name on the names list is Eric

    \end{Verbatim}

    \subsubsection{Basic Operators}\label{basic-operators}

We've touched on a few simple operations so far, so let's dive a little
further in now.

    \paragraph{Arithmetic Operators}\label{arithmetic-operators}

These are the ones we should all be familiar with, the mathematical
operators of addition, subtraction, multiplication, and division. Don't
forget when using these to keep PEMDAS in mind! That is, keep in mind
your order of operations, as Python will follow it.

    \begin{Verbatim}[commandchars=\\\{\}]
{\color{incolor}In [{\color{incolor}15}]:} \PY{n}{number} \PY{o}{=} \PY{l+m+mi}{1} \PY{o}{+} \PY{p}{(}\PY{l+m+mi}{2} \PY{o}{*} \PY{l+m+mi}{3} \PY{o}{/} \PY{l+m+mf}{4.0}\PY{p}{)}
         \PY{n+nb}{print}\PY{p}{(}\PY{n}{number}\PY{p}{)}
\end{Verbatim}


    \begin{Verbatim}[commandchars=\\\{\}]
2.5

    \end{Verbatim}

    A more complicated operation is the modulo operator (\texttt{\%}) which
returns the integer remainder of the division of two numbers: dividend
\% divisor = remainder.

    \begin{Verbatim}[commandchars=\\\{\}]
{\color{incolor}In [{\color{incolor}16}]:} \PY{n}{remainder} \PY{o}{=} \PY{l+m+mi}{11} \PY{o}{\PYZpc{}} \PY{l+m+mi}{3}
         \PY{n+nb}{print}\PY{p}{(}\PY{n}{remainder}\PY{p}{)}
\end{Verbatim}


    \begin{Verbatim}[commandchars=\\\{\}]
2

    \end{Verbatim}

    Unlike languages like R that you know, are beautiful, Python doesn't
always play nice with the human eye. Just like how Jupyter Notebook is a
lesser version of RMarkdown...I'm getting off topic. So unlike what you
might expect by saying \texttt{3\^{}2} is \(3^2\) or "three squared",
Python handles this with two multiplication symbols instead.

    \begin{Verbatim}[commandchars=\\\{\}]
{\color{incolor}In [{\color{incolor}17}]:} \PY{n}{squared} \PY{o}{=} \PY{l+m+mi}{3} \PY{o}{*}\PY{o}{*} \PY{l+m+mi}{2}
         \PY{n}{cubed} \PY{o}{=} \PY{l+m+mi}{2} \PY{o}{*}\PY{o}{*} \PY{l+m+mi}{3}
         \PY{n+nb}{print}\PY{p}{(}\PY{n}{squared}\PY{p}{)}
         \PY{n+nb}{print}\PY{p}{(}\PY{n}{cubed}\PY{p}{)}
\end{Verbatim}


    \begin{Verbatim}[commandchars=\\\{\}]
9
8

    \end{Verbatim}

    \paragraph{Using Operators with Lists}\label{using-operators-with-lists}

Lists can be handles with operators as well. For example, you can
combine lists by using the addition operator.

    \begin{Verbatim}[commandchars=\\\{\}]
{\color{incolor}In [{\color{incolor}18}]:} \PY{n}{even\PYZus{}numbers} \PY{o}{=} \PY{p}{[}\PY{l+m+mi}{2}\PY{p}{,}\PY{l+m+mi}{4}\PY{p}{,}\PY{l+m+mi}{6}\PY{p}{,}\PY{l+m+mi}{8}\PY{p}{]}
         \PY{n}{odd\PYZus{}numbers} \PY{o}{=} \PY{p}{[}\PY{l+m+mi}{1}\PY{p}{,}\PY{l+m+mi}{3}\PY{p}{,}\PY{l+m+mi}{5}\PY{p}{,}\PY{l+m+mi}{7}\PY{p}{]}
         \PY{n}{all\PYZus{}numbers} \PY{o}{=} \PY{n}{odd\PYZus{}numbers} \PY{o}{+} \PY{n}{even\PYZus{}numbers}
         \PY{n+nb}{print}\PY{p}{(}\PY{n}{all\PYZus{}numbers}\PY{p}{)}
\end{Verbatim}


    \begin{Verbatim}[commandchars=\\\{\}]
[1, 3, 5, 7, 2, 4, 6, 8]

    \end{Verbatim}

    Keep in mind that also unlike vectors in R, when multiplying a list by a
scalar value, Python does \textbf{not} do vector algebra. Hence we get
the following.

    \begin{Verbatim}[commandchars=\\\{\}]
{\color{incolor}In [{\color{incolor}19}]:} \PY{n}{list1} \PY{o}{=} \PY{p}{[}\PY{l+m+mi}{1}\PY{p}{,}\PY{l+m+mi}{2}\PY{p}{,}\PY{l+m+mi}{3}\PY{p}{]}
         \PY{n}{list2} \PY{o}{=} \PY{n}{list1} \PY{o}{*} \PY{l+m+mi}{3}
         
         \PY{n+nb}{print}\PY{p}{(}\PY{n}{list1}\PY{p}{)}
         \PY{n+nb}{print}\PY{p}{(}\PY{n}{list2}\PY{p}{)}
\end{Verbatim}


    \begin{Verbatim}[commandchars=\\\{\}]
[1, 2, 3]
[1, 2, 3, 1, 2, 3, 1, 2, 3]

    \end{Verbatim}

    \paragraph{Exercise}\label{exercise}

The target of this exercise is to create two lists called
\texttt{x\_list} and \texttt{y\_list}, which contain 10 instances of the
variables \texttt{x} and \texttt{y}, respectively. You are also required
to create a list called \texttt{big\_list}, which contains the variables
\texttt{x} and \texttt{y}, 10 times each, by concatenating the two lists
you have created.

    \begin{Verbatim}[commandchars=\\\{\}]
{\color{incolor}In [{\color{incolor}20}]:} \PY{n}{x} \PY{o}{=} \PY{n+nb}{object}\PY{p}{(}\PY{p}{)}
         \PY{n}{y} \PY{o}{=} \PY{n+nb}{object}\PY{p}{(}\PY{p}{)}
         
         \PY{c+c1}{\PYZsh{} TODO: change this code}
         \PY{n}{x\PYZus{}list} \PY{o}{=} \PY{p}{[}\PY{n}{x}\PY{p}{]}
         \PY{n}{x\PYZus{}list} \PY{o}{=} \PY{n}{x\PYZus{}list} \PY{o}{*} \PY{l+m+mi}{10}
         \PY{n}{y\PYZus{}list} \PY{o}{=} \PY{p}{[}\PY{n}{y}\PY{p}{]}
         \PY{n}{y\PYZus{}list} \PY{o}{=} \PY{n}{y\PYZus{}list} \PY{o}{*} \PY{l+m+mi}{10}
         \PY{n}{big\PYZus{}list} \PY{o}{=} \PY{n}{x\PYZus{}list} \PY{o}{+} \PY{n}{y\PYZus{}list}
         
         \PY{n+nb}{print}\PY{p}{(}\PY{l+s+s2}{\PYZdq{}}\PY{l+s+s2}{x\PYZus{}list contains }\PY{l+s+si}{\PYZpc{}d}\PY{l+s+s2}{ objects}\PY{l+s+s2}{\PYZdq{}} \PY{o}{\PYZpc{}} \PY{n+nb}{len}\PY{p}{(}\PY{n}{x\PYZus{}list}\PY{p}{)}\PY{p}{)}
         \PY{n+nb}{print}\PY{p}{(}\PY{l+s+s2}{\PYZdq{}}\PY{l+s+s2}{y\PYZus{}list contains }\PY{l+s+si}{\PYZpc{}d}\PY{l+s+s2}{ objects}\PY{l+s+s2}{\PYZdq{}} \PY{o}{\PYZpc{}} \PY{n+nb}{len}\PY{p}{(}\PY{n}{y\PYZus{}list}\PY{p}{)}\PY{p}{)}
         \PY{n+nb}{print}\PY{p}{(}\PY{l+s+s2}{\PYZdq{}}\PY{l+s+s2}{big\PYZus{}list contains }\PY{l+s+si}{\PYZpc{}d}\PY{l+s+s2}{ objects}\PY{l+s+s2}{\PYZdq{}} \PY{o}{\PYZpc{}} \PY{n+nb}{len}\PY{p}{(}\PY{n}{big\PYZus{}list}\PY{p}{)}\PY{p}{)}
         
         \PY{c+c1}{\PYZsh{} testing code}
         \PY{k}{if} \PY{n}{x\PYZus{}list}\PY{o}{.}\PY{n}{count}\PY{p}{(}\PY{n}{x}\PY{p}{)} \PY{o}{==} \PY{l+m+mi}{10} \PY{o+ow}{and} \PY{n}{y\PYZus{}list}\PY{o}{.}\PY{n}{count}\PY{p}{(}\PY{n}{y}\PY{p}{)} \PY{o}{==} \PY{l+m+mi}{10}\PY{p}{:}
             \PY{n+nb}{print}\PY{p}{(}\PY{l+s+s2}{\PYZdq{}}\PY{l+s+s2}{Almost there...}\PY{l+s+s2}{\PYZdq{}}\PY{p}{)}
         \PY{k}{if} \PY{n}{big\PYZus{}list}\PY{o}{.}\PY{n}{count}\PY{p}{(}\PY{n}{x}\PY{p}{)} \PY{o}{==} \PY{l+m+mi}{10} \PY{o+ow}{and} \PY{n}{big\PYZus{}list}\PY{o}{.}\PY{n}{count}\PY{p}{(}\PY{n}{y}\PY{p}{)} \PY{o}{==} \PY{l+m+mi}{10}\PY{p}{:}
             \PY{n+nb}{print}\PY{p}{(}\PY{l+s+s2}{\PYZdq{}}\PY{l+s+s2}{Great!}\PY{l+s+s2}{\PYZdq{}}\PY{p}{)}
\end{Verbatim}


    \begin{Verbatim}[commandchars=\\\{\}]
x\_list contains 10 objects
y\_list contains 10 objects
big\_list contains 20 objects
Almost there{\ldots}
Great!

    \end{Verbatim}

    \subsubsection{String Formatting}\label{string-formatting}

If you're familiar with C, you're in luck! Python uses C-style string
formatting to create new, formatted strings. It's also similar to the
\texttt{sprintf()} function in R, that allows the user to use C-style
string formatting commands. (Have you noticed I sprinkle a lot of R in
here? It's my baby.) The \texttt{\%} operator is used to format a set of
variables enclosed in a "tuple" (a fixed size list), together with a
format string, which contains normal text together with "argument
specifiers", special symbols like \texttt{\%s} and \texttt{\%d}. Here's
an example.

    \begin{Verbatim}[commandchars=\\\{\}]
{\color{incolor}In [{\color{incolor}21}]:} \PY{c+c1}{\PYZsh{} This prints out \PYZdq{}Hello, John!\PYZdq{}}
         \PY{n}{name} \PY{o}{=} \PY{l+s+s2}{\PYZdq{}}\PY{l+s+s2}{John}\PY{l+s+s2}{\PYZdq{}}
         \PY{n+nb}{print}\PY{p}{(}\PY{l+s+s2}{\PYZdq{}}\PY{l+s+s2}{Hello, }\PY{l+s+si}{\PYZpc{}s}\PY{l+s+s2}{!}\PY{l+s+s2}{\PYZdq{}} \PY{o}{\PYZpc{}} \PY{n}{name}\PY{p}{)}
\end{Verbatim}


    \begin{Verbatim}[commandchars=\\\{\}]
Hello, John!

    \end{Verbatim}

    To use two or more argument specifiers, use a tuple (parentheses).

    \begin{Verbatim}[commandchars=\\\{\}]
{\color{incolor}In [{\color{incolor}22}]:} \PY{c+c1}{\PYZsh{} This prints out \PYZdq{}John is 23 years old.\PYZdq{}}
         \PY{n}{name} \PY{o}{=} \PY{l+s+s2}{\PYZdq{}}\PY{l+s+s2}{John}\PY{l+s+s2}{\PYZdq{}}
         \PY{n}{age} \PY{o}{=} \PY{l+m+mi}{23}
         \PY{n+nb}{print}\PY{p}{(}\PY{l+s+s2}{\PYZdq{}}\PY{l+s+si}{\PYZpc{}s}\PY{l+s+s2}{ is }\PY{l+s+si}{\PYZpc{}d}\PY{l+s+s2}{ years old.}\PY{l+s+s2}{\PYZdq{}} \PY{o}{\PYZpc{}} \PY{p}{(}\PY{n}{name}\PY{p}{,} \PY{n}{age}\PY{p}{)}\PY{p}{)}
\end{Verbatim}


    \begin{Verbatim}[commandchars=\\\{\}]
John is 23 years old.

    \end{Verbatim}

    Any object which is not a string can be formatted using the \texttt{\%s}
operator as well. The string which returns from the "repr" method of
that object is formatted as the string. For example:

    \begin{Verbatim}[commandchars=\\\{\}]
{\color{incolor}In [{\color{incolor}23}]:} \PY{c+c1}{\PYZsh{} This prints out: A list: [1, 2, 3]}
         \PY{n}{mylist} \PY{o}{=} \PY{p}{[}\PY{l+m+mi}{1}\PY{p}{,}\PY{l+m+mi}{2}\PY{p}{,}\PY{l+m+mi}{3}\PY{p}{]}
         \PY{n+nb}{print}\PY{p}{(}\PY{l+s+s2}{\PYZdq{}}\PY{l+s+s2}{A list: }\PY{l+s+si}{\PYZpc{}s}\PY{l+s+s2}{\PYZdq{}} \PY{o}{\PYZpc{}} \PY{n}{mylist}\PY{p}{)}
\end{Verbatim}


    \begin{Verbatim}[commandchars=\\\{\}]
A list: [1, 2, 3]

    \end{Verbatim}

    Here are some basic argument specifiers we should know:

\texttt{\%s\ -\ String\ (or\ any\ object\ with\ a\ string\ representation,\ like\ numbers)}

\texttt{\%d\ -\ Integers}

\texttt{\%f\ -\ Floating\ point\ numbers}

\texttt{\%.\textless{}number\ of\ digits\textgreater{}f\ -\ Floating\ point\ numbers\ with\ a\ fixed\ amount\ of\ digits\ to\ the\ right\ of\ the\ dot.}

\texttt{\%x/\%X\ -\ Integers\ in\ hex\ representation\ (lowercase/uppercase)}

    \paragraph{Exercise}\label{exercise}

You will need to write a format string which prints out the data using
the following syntax:
\texttt{Hello\ John\ Doe.\ Your\ current\ balance\ is\ \$53.44.}

    \begin{Verbatim}[commandchars=\\\{\}]
{\color{incolor}In [{\color{incolor}24}]:} \PY{n}{data} \PY{o}{=} \PY{p}{(}\PY{l+s+s2}{\PYZdq{}}\PY{l+s+s2}{John}\PY{l+s+s2}{\PYZdq{}}\PY{p}{,} \PY{l+s+s2}{\PYZdq{}}\PY{l+s+s2}{Doe}\PY{l+s+s2}{\PYZdq{}}\PY{p}{,} \PY{l+m+mf}{53.44}\PY{p}{)}
         \PY{n}{format\PYZus{}string} \PY{o}{=} \PY{l+s+s2}{\PYZdq{}}\PY{l+s+s2}{Hello }\PY{l+s+si}{\PYZpc{}s}\PY{l+s+s2}{ }\PY{l+s+si}{\PYZpc{}s}\PY{l+s+s2}{. Your current balance is \PYZdl{}}\PY{l+s+si}{\PYZpc{}.2f}\PY{l+s+s2}{\PYZdq{}}
         
         \PY{n+nb}{print}\PY{p}{(}\PY{n}{format\PYZus{}string} \PY{o}{\PYZpc{}} \PY{n}{data}\PY{p}{)}
\end{Verbatim}


    \begin{Verbatim}[commandchars=\\\{\}]
Hello John Doe. Your current balance is \$53.44

    \end{Verbatim}

    \subsubsection{Basic String Operations}\label{basic-string-operations}

By now we aught to know what strings are, but there's quite a big more
we can do with them. To start, check out the \texttt{len()} function.

    \begin{Verbatim}[commandchars=\\\{\}]
{\color{incolor}In [{\color{incolor}25}]:} \PY{n}{astring} \PY{o}{=} \PY{l+s+s1}{\PYZsq{}}\PY{l+s+s1}{Zoinks!}\PY{l+s+s1}{\PYZsq{}}
         \PY{n+nb}{len}\PY{p}{(}\PY{n}{astring}\PY{p}{)}
\end{Verbatim}


\begin{Verbatim}[commandchars=\\\{\}]
{\color{outcolor}Out[{\color{outcolor}25}]:} 7
\end{Verbatim}
            
    As you can see here the \texttt{len()} function returns 7 since that's
how long the \texttt{astring} object is, including the punctuation. If
we had spaces, those would be counted as well. We can also get a bit
more precise with our string operations. What we're about to dive into
can be useful for text mining.

    \begin{Verbatim}[commandchars=\\\{\}]
{\color{incolor}In [{\color{incolor}26}]:} \PY{n}{astring} \PY{o}{=} \PY{l+s+s2}{\PYZdq{}}\PY{l+s+s2}{Zoinko the Clown}\PY{l+s+s2}{\PYZdq{}}
         \PY{n+nb}{print}\PY{p}{(}\PY{n}{astring}\PY{o}{.}\PY{n}{index}\PY{p}{(}\PY{l+s+s2}{\PYZdq{}}\PY{l+s+s2}{o}\PY{l+s+s2}{\PYZdq{}}\PY{p}{)}\PY{p}{)}
\end{Verbatim}


    \begin{Verbatim}[commandchars=\\\{\}]
1

    \end{Verbatim}

    That prints out 1, because the location of the first occurrence of the
letter "o" is 1 characters away from the first character. Notice how
there are actually three o's in the phrase - this method only recognizes
the first.

But why didn't it print out 2? Isn't "o" the second character in the
string? As we've mentioned before, Python (but not R because it's way
cooler) start things at 0 instead of 1. So the index of "o" is 1.

On the flip side of this, if we used \texttt{.count} instead of
\texttt{.index} we get the following.

    \begin{Verbatim}[commandchars=\\\{\}]
{\color{incolor}In [{\color{incolor}27}]:} \PY{n}{astring} \PY{o}{=} \PY{l+s+s2}{\PYZdq{}}\PY{l+s+s2}{Zoinko the Clown}\PY{l+s+s2}{\PYZdq{}}
         \PY{n+nb}{print}\PY{p}{(}\PY{n}{astring}\PY{o}{.}\PY{n}{count}\PY{p}{(}\PY{l+s+s2}{\PYZdq{}}\PY{l+s+s2}{o}\PY{l+s+s2}{\PYZdq{}}\PY{p}{)}\PY{p}{)}
\end{Verbatim}


    \begin{Verbatim}[commandchars=\\\{\}]
3

    \end{Verbatim}

    As we see here, \texttt{.count} returns to us the number of times that
the input character was used in the string. Say we wanted to take a
slice of a string now. By that I mean, consider a situation where we
only want a specific portion of a string. The code cell below shows how
we may do this. Note that in this code cell, since the text is fairly
long, we can make it into a multi-line string by coding as we do below.

    \begin{Verbatim}[commandchars=\\\{\}]
{\color{incolor}In [{\color{incolor}48}]:} \PY{n}{astring} \PY{o}{=} \PY{p}{(}\PY{l+s+s2}{\PYZdq{}}\PY{l+s+s2}{Help me I}\PY{l+s+s2}{\PYZsq{}}\PY{l+s+s2}{m trapped inside this computer!}\PY{l+s+s2}{\PYZdq{}} 
                    \PY{l+s+s2}{\PYZdq{}}\PY{l+s+s2}{ This is not a joke please send help!}\PY{l+s+s2}{\PYZdq{}}\PY{p}{)}
         \PY{n+nb}{print}\PY{p}{(}\PY{n}{astring}\PY{p}{)}
         \PY{n+nb}{print}\PY{p}{(}\PY{n}{astring}\PY{p}{[}\PY{l+m+mi}{42}\PY{p}{:}\PY{l+m+mi}{50}\PY{p}{]} \PY{o}{+} \PY{n}{astring}\PY{p}{[}\PY{l+m+mi}{54}\PY{p}{:}\PY{l+m+mi}{60}\PY{p}{]}\PY{p}{)}
\end{Verbatim}


    \begin{Verbatim}[commandchars=\\\{\}]
Help me I'm trapped inside this computer! This is not a joke please send help!
This is a joke

    \end{Verbatim}

    Note that this uses the standard indexing methods that we should be
getting used to (starting with 0 instead of 1).

We can also slice text with negative numbered index values, ie. if you
were to write \texttt{astring{[}-3{]}} the print statement would return
the 3rd character from the end. Another option we have comes when we put
a 3rd item into the brackets, ie. \texttt{astring{[}x:y:z{]}}. Here the
form is {[}start:stop:step{]}, basically meaning that you start with
index \(x\), stop at index \(y\), and go up by a value of \(z\). Here's
an example.

    \begin{Verbatim}[commandchars=\\\{\}]
{\color{incolor}In [{\color{incolor}29}]:} \PY{n}{astring} \PY{o}{=} \PY{p}{(}\PY{l+s+s2}{\PYZdq{}}\PY{l+s+s2}{Help me I}\PY{l+s+s2}{\PYZsq{}}\PY{l+s+s2}{m trapped inside this computer!}\PY{l+s+s2}{\PYZdq{}} 
                    \PY{l+s+s2}{\PYZdq{}}\PY{l+s+s2}{ This is not a joke please send help!}\PY{l+s+s2}{\PYZdq{}}\PY{p}{)}
         \PY{n+nb}{print}\PY{p}{(}\PY{n}{astring}\PY{p}{[}\PY{l+m+mi}{3}\PY{p}{:}\PY{l+m+mi}{60}\PY{p}{:}\PY{l+m+mi}{2}\PY{p}{]}\PY{p}{)}
\end{Verbatim}


    \begin{Verbatim}[commandchars=\\\{\}]
pm ' rpe nieti optr hsi o  oe

    \end{Verbatim}

    In order to reverse a string we can do the following:

    \begin{Verbatim}[commandchars=\\\{\}]
{\color{incolor}In [{\color{incolor}30}]:} \PY{n}{astring} \PY{o}{=} \PY{l+s+s2}{\PYZdq{}}\PY{l+s+s2}{Coding is cool...if you}\PY{l+s+s2}{\PYZsq{}}\PY{l+s+s2}{re a nerd.}\PY{l+s+s2}{\PYZdq{}}
         \PY{n+nb}{print}\PY{p}{(}\PY{n}{astring}\PY{p}{[}\PY{p}{:}\PY{p}{:}\PY{o}{\PYZhy{}}\PY{l+m+mi}{1}\PY{p}{]}\PY{p}{)}
\end{Verbatim}


    \begin{Verbatim}[commandchars=\\\{\}]
.dren a er'uoy fi{\ldots}looc si gnidoC

    \end{Verbatim}

    Take that bully. We can also do this which gives off two fairly
different impressions:

    \begin{Verbatim}[commandchars=\\\{\}]
{\color{incolor}In [{\color{incolor}31}]:} \PY{n}{astring} \PY{o}{=} \PY{l+s+s2}{\PYZdq{}}\PY{l+s+s2}{Shut Up Dad}\PY{l+s+s2}{\PYZdq{}}
         \PY{n+nb}{print}\PY{p}{(}\PY{n}{astring}\PY{o}{.}\PY{n}{upper}\PY{p}{(}\PY{p}{)}\PY{p}{)}
         \PY{n+nb}{print}\PY{p}{(}\PY{n}{astring}\PY{o}{.}\PY{n}{lower}\PY{p}{(}\PY{p}{)}\PY{p}{)}
\end{Verbatim}


    \begin{Verbatim}[commandchars=\\\{\}]
SHUT UP DAD
shut up dad

    \end{Verbatim}

    And we can test what is contained within string values:

    \begin{Verbatim}[commandchars=\\\{\}]
{\color{incolor}In [{\color{incolor}32}]:} \PY{n}{astring} \PY{o}{=} \PY{l+s+s2}{\PYZdq{}}\PY{l+s+s2}{Hello friends!}\PY{l+s+s2}{\PYZdq{}}
         \PY{n+nb}{print}\PY{p}{(}\PY{n}{astring}\PY{o}{.}\PY{n}{startswith}\PY{p}{(}\PY{l+s+s2}{\PYZdq{}}\PY{l+s+s2}{Hello}\PY{l+s+s2}{\PYZdq{}}\PY{p}{)}\PY{p}{)}
         \PY{n+nb}{print}\PY{p}{(}\PY{n}{astring}\PY{o}{.}\PY{n}{endswith}\PY{p}{(}\PY{l+s+s2}{\PYZdq{}}\PY{l+s+s2}{asdfasdfasdf}\PY{l+s+s2}{\PYZdq{}}\PY{p}{)}\PY{p}{)}
\end{Verbatim}


    \begin{Verbatim}[commandchars=\\\{\}]
True
False

    \end{Verbatim}

    The last thing we'll go over in this section is how to split a string
into multiple strings grouped together in a list. This could be useful
when doing text mining on down the line.

    \begin{Verbatim}[commandchars=\\\{\}]
{\color{incolor}In [{\color{incolor}33}]:} \PY{n}{astring} \PY{o}{=} \PY{l+s+s2}{\PYZdq{}}\PY{l+s+s2}{We}\PY{l+s+s2}{\PYZsq{}}\PY{l+s+s2}{re almost done!}\PY{l+s+s2}{\PYZdq{}}
         \PY{n}{afewwords} \PY{o}{=} \PY{n}{astring}\PY{o}{.}\PY{n}{split}\PY{p}{(}\PY{l+s+s2}{\PYZdq{}}\PY{l+s+s2}{ }\PY{l+s+s2}{\PYZdq{}}\PY{p}{)}
         \PY{n}{afewwords}
\end{Verbatim}


\begin{Verbatim}[commandchars=\\\{\}]
{\color{outcolor}Out[{\color{outcolor}33}]:} ["We're", 'almost', 'done!']
\end{Verbatim}
            
    \paragraph{Exercise}\label{exercise}

Try to fix the code to print out the correct information by changing the
string. The Solution is done below.

    \begin{Verbatim}[commandchars=\\\{\}]
{\color{incolor}In [{\color{incolor}34}]:} \PY{n}{s} \PY{o}{=} \PY{l+s+s2}{\PYZdq{}}\PY{l+s+s2}{Hey there! what should this string be?}\PY{l+s+s2}{\PYZdq{}}
         \PY{c+c1}{\PYZsh{} Length should be 20}
         \PY{n+nb}{print}\PY{p}{(}\PY{l+s+s2}{\PYZdq{}}\PY{l+s+s2}{Length of s = }\PY{l+s+si}{\PYZpc{}d}\PY{l+s+s2}{\PYZdq{}} \PY{o}{\PYZpc{}} \PY{n+nb}{len}\PY{p}{(}\PY{n}{s}\PY{p}{)}\PY{p}{)}
         
         \PY{c+c1}{\PYZsh{} First occurrence of \PYZdq{}a\PYZdq{} should be at index 8}
         \PY{n+nb}{print}\PY{p}{(}\PY{l+s+s2}{\PYZdq{}}\PY{l+s+s2}{The first occurrence of the letter a = }\PY{l+s+si}{\PYZpc{}d}\PY{l+s+s2}{\PYZdq{}} \PY{o}{\PYZpc{}} \PY{n}{s}\PY{o}{.}\PY{n}{index}\PY{p}{(}\PY{l+s+s2}{\PYZdq{}}\PY{l+s+s2}{a}\PY{l+s+s2}{\PYZdq{}}\PY{p}{)}\PY{p}{)}
         
         \PY{c+c1}{\PYZsh{} Number of a\PYZsq{}s should be 2}
         \PY{n+nb}{print}\PY{p}{(}\PY{l+s+s2}{\PYZdq{}}\PY{l+s+s2}{a occurs }\PY{l+s+si}{\PYZpc{}d}\PY{l+s+s2}{ times}\PY{l+s+s2}{\PYZdq{}} \PY{o}{\PYZpc{}} \PY{n}{s}\PY{o}{.}\PY{n}{count}\PY{p}{(}\PY{l+s+s2}{\PYZdq{}}\PY{l+s+s2}{a}\PY{l+s+s2}{\PYZdq{}}\PY{p}{)}\PY{p}{)}
         
         \PY{c+c1}{\PYZsh{} Slicing the string into bits}
         \PY{n+nb}{print}\PY{p}{(}\PY{l+s+s2}{\PYZdq{}}\PY{l+s+s2}{The first five characters are }\PY{l+s+s2}{\PYZsq{}}\PY{l+s+si}{\PYZpc{}s}\PY{l+s+s2}{\PYZsq{}}\PY{l+s+s2}{\PYZdq{}} \PY{o}{\PYZpc{}} \PY{n}{s}\PY{p}{[}\PY{p}{:}\PY{l+m+mi}{5}\PY{p}{]}\PY{p}{)} \PY{c+c1}{\PYZsh{} Start to 5}
         \PY{n+nb}{print}\PY{p}{(}\PY{l+s+s2}{\PYZdq{}}\PY{l+s+s2}{The next five characters are }\PY{l+s+s2}{\PYZsq{}}\PY{l+s+si}{\PYZpc{}s}\PY{l+s+s2}{\PYZsq{}}\PY{l+s+s2}{\PYZdq{}} \PY{o}{\PYZpc{}} \PY{n}{s}\PY{p}{[}\PY{l+m+mi}{5}\PY{p}{:}\PY{l+m+mi}{10}\PY{p}{]}\PY{p}{)} \PY{c+c1}{\PYZsh{} 5 to 10}
         \PY{n+nb}{print}\PY{p}{(}\PY{l+s+s2}{\PYZdq{}}\PY{l+s+s2}{The thirteenth character is }\PY{l+s+s2}{\PYZsq{}}\PY{l+s+si}{\PYZpc{}s}\PY{l+s+s2}{\PYZsq{}}\PY{l+s+s2}{\PYZdq{}} \PY{o}{\PYZpc{}} \PY{n}{s}\PY{p}{[}\PY{l+m+mi}{12}\PY{p}{]}\PY{p}{)} \PY{c+c1}{\PYZsh{} Just number 12}
         \PY{n+nb}{print}\PY{p}{(}\PY{l+s+s2}{\PYZdq{}}\PY{l+s+s2}{The characters with odd index are }\PY{l+s+s2}{\PYZsq{}}\PY{l+s+si}{\PYZpc{}s}\PY{l+s+s2}{\PYZsq{}}\PY{l+s+s2}{\PYZdq{}} \PY{o}{\PYZpc{}}\PY{k}{s}[1::2]) \PYZsh{}(0\PYZhy{}based indexing)
         \PY{n+nb}{print}\PY{p}{(}\PY{l+s+s2}{\PYZdq{}}\PY{l+s+s2}{The last five characters are }\PY{l+s+s2}{\PYZsq{}}\PY{l+s+si}{\PYZpc{}s}\PY{l+s+s2}{\PYZsq{}}\PY{l+s+s2}{\PYZdq{}} \PY{o}{\PYZpc{}} \PY{n}{s}\PY{p}{[}\PY{o}{\PYZhy{}}\PY{l+m+mi}{5}\PY{p}{:}\PY{p}{]}\PY{p}{)} \PY{c+c1}{\PYZsh{} 5th\PYZhy{}from\PYZhy{}last to end}
         
         \PY{c+c1}{\PYZsh{} Convert everything to uppercase}
         \PY{n+nb}{print}\PY{p}{(}\PY{l+s+s2}{\PYZdq{}}\PY{l+s+s2}{String in uppercase: }\PY{l+s+si}{\PYZpc{}s}\PY{l+s+s2}{\PYZdq{}} \PY{o}{\PYZpc{}} \PY{n}{s}\PY{o}{.}\PY{n}{upper}\PY{p}{(}\PY{p}{)}\PY{p}{)}
         
         \PY{c+c1}{\PYZsh{} Convert everything to lowercase}
         \PY{n+nb}{print}\PY{p}{(}\PY{l+s+s2}{\PYZdq{}}\PY{l+s+s2}{String in lowercase: }\PY{l+s+si}{\PYZpc{}s}\PY{l+s+s2}{\PYZdq{}} \PY{o}{\PYZpc{}} \PY{n}{s}\PY{o}{.}\PY{n}{lower}\PY{p}{(}\PY{p}{)}\PY{p}{)}
         
         \PY{c+c1}{\PYZsh{} Check how a string starts}
         \PY{k}{if} \PY{n}{s}\PY{o}{.}\PY{n}{startswith}\PY{p}{(}\PY{l+s+s2}{\PYZdq{}}\PY{l+s+s2}{Str}\PY{l+s+s2}{\PYZdq{}}\PY{p}{)}\PY{p}{:}
             \PY{n+nb}{print}\PY{p}{(}\PY{l+s+s2}{\PYZdq{}}\PY{l+s+s2}{String starts with }\PY{l+s+s2}{\PYZsq{}}\PY{l+s+s2}{Str}\PY{l+s+s2}{\PYZsq{}}\PY{l+s+s2}{. Good!}\PY{l+s+s2}{\PYZdq{}}\PY{p}{)}
         
         \PY{c+c1}{\PYZsh{} Check how a string ends}
         \PY{k}{if} \PY{n}{s}\PY{o}{.}\PY{n}{endswith}\PY{p}{(}\PY{l+s+s2}{\PYZdq{}}\PY{l+s+s2}{ome!}\PY{l+s+s2}{\PYZdq{}}\PY{p}{)}\PY{p}{:}
             \PY{n+nb}{print}\PY{p}{(}\PY{l+s+s2}{\PYZdq{}}\PY{l+s+s2}{String ends with }\PY{l+s+s2}{\PYZsq{}}\PY{l+s+s2}{ome!}\PY{l+s+s2}{\PYZsq{}}\PY{l+s+s2}{. Good!}\PY{l+s+s2}{\PYZdq{}}\PY{p}{)}
         
         \PY{c+c1}{\PYZsh{} Split the string into three separate strings,}
         \PY{c+c1}{\PYZsh{} each containing only a word}
         \PY{n+nb}{print}\PY{p}{(}\PY{l+s+s2}{\PYZdq{}}\PY{l+s+s2}{Split the words of the string: }\PY{l+s+si}{\PYZpc{}s}\PY{l+s+s2}{\PYZdq{}} \PY{o}{\PYZpc{}} \PY{n}{s}\PY{o}{.}\PY{n}{split}\PY{p}{(}\PY{l+s+s2}{\PYZdq{}}\PY{l+s+s2}{ }\PY{l+s+s2}{\PYZdq{}}\PY{p}{)}\PY{p}{)}
\end{Verbatim}


    \begin{Verbatim}[commandchars=\\\{\}]
Length of s = 38
The first occurrence of the letter a = 13
a occurs 1 times
The first five characters are 'Hey t'
The next five characters are 'here!'
The thirteenth character is 'h'
The characters with odd index are 'e hr!wa hudti tigb?'
The last five characters are 'g be?'
String in uppercase: HEY THERE! WHAT SHOULD THIS STRING BE?
String in lowercase: hey there! what should this string be?
Split the words of the string: ['Hey', 'there!', 'what', 'should', 'this', 'string', 'be?']

    \end{Verbatim}

    \begin{Verbatim}[commandchars=\\\{\}]
{\color{incolor}In [{\color{incolor}35}]:} \PY{n}{s} \PY{o}{=} \PY{l+s+s2}{\PYZdq{}}\PY{l+s+s2}{Strings are awesome!}\PY{l+s+s2}{\PYZdq{}}
         \PY{c+c1}{\PYZsh{} Length should be 20}
         \PY{n+nb}{print}\PY{p}{(}\PY{l+s+s2}{\PYZdq{}}\PY{l+s+s2}{Length of s = }\PY{l+s+si}{\PYZpc{}d}\PY{l+s+s2}{\PYZdq{}} \PY{o}{\PYZpc{}} \PY{n+nb}{len}\PY{p}{(}\PY{n}{s}\PY{p}{)}\PY{p}{)}
         
         \PY{c+c1}{\PYZsh{} First occurrence of \PYZdq{}a\PYZdq{} should be at index 8}
         \PY{n+nb}{print}\PY{p}{(}\PY{l+s+s2}{\PYZdq{}}\PY{l+s+s2}{The first occurrence of the letter a = }\PY{l+s+si}{\PYZpc{}d}\PY{l+s+s2}{\PYZdq{}} \PY{o}{\PYZpc{}} \PY{n}{s}\PY{o}{.}\PY{n}{index}\PY{p}{(}\PY{l+s+s2}{\PYZdq{}}\PY{l+s+s2}{a}\PY{l+s+s2}{\PYZdq{}}\PY{p}{)}\PY{p}{)}
         
         \PY{c+c1}{\PYZsh{} Number of a\PYZsq{}s should be 2}
         \PY{n+nb}{print}\PY{p}{(}\PY{l+s+s2}{\PYZdq{}}\PY{l+s+s2}{a occurs }\PY{l+s+si}{\PYZpc{}d}\PY{l+s+s2}{ times}\PY{l+s+s2}{\PYZdq{}} \PY{o}{\PYZpc{}} \PY{n}{s}\PY{o}{.}\PY{n}{count}\PY{p}{(}\PY{l+s+s2}{\PYZdq{}}\PY{l+s+s2}{a}\PY{l+s+s2}{\PYZdq{}}\PY{p}{)}\PY{p}{)}
         
         \PY{c+c1}{\PYZsh{} Slicing the string into bits}
         \PY{n+nb}{print}\PY{p}{(}\PY{l+s+s2}{\PYZdq{}}\PY{l+s+s2}{The first five characters are }\PY{l+s+s2}{\PYZsq{}}\PY{l+s+si}{\PYZpc{}s}\PY{l+s+s2}{\PYZsq{}}\PY{l+s+s2}{\PYZdq{}} \PY{o}{\PYZpc{}} \PY{n}{s}\PY{p}{[}\PY{p}{:}\PY{l+m+mi}{5}\PY{p}{]}\PY{p}{)} \PY{c+c1}{\PYZsh{} Start to 5}
         \PY{n+nb}{print}\PY{p}{(}\PY{l+s+s2}{\PYZdq{}}\PY{l+s+s2}{The next five characters are }\PY{l+s+s2}{\PYZsq{}}\PY{l+s+si}{\PYZpc{}s}\PY{l+s+s2}{\PYZsq{}}\PY{l+s+s2}{\PYZdq{}} \PY{o}{\PYZpc{}} \PY{n}{s}\PY{p}{[}\PY{l+m+mi}{5}\PY{p}{:}\PY{l+m+mi}{10}\PY{p}{]}\PY{p}{)} \PY{c+c1}{\PYZsh{} 5 to 10}
         \PY{n+nb}{print}\PY{p}{(}\PY{l+s+s2}{\PYZdq{}}\PY{l+s+s2}{The thirteenth character is }\PY{l+s+s2}{\PYZsq{}}\PY{l+s+si}{\PYZpc{}s}\PY{l+s+s2}{\PYZsq{}}\PY{l+s+s2}{\PYZdq{}} \PY{o}{\PYZpc{}} \PY{n}{s}\PY{p}{[}\PY{l+m+mi}{12}\PY{p}{]}\PY{p}{)} \PY{c+c1}{\PYZsh{} Just number 12}
         \PY{n+nb}{print}\PY{p}{(}\PY{l+s+s2}{\PYZdq{}}\PY{l+s+s2}{The characters with odd index are }\PY{l+s+s2}{\PYZsq{}}\PY{l+s+si}{\PYZpc{}s}\PY{l+s+s2}{\PYZsq{}}\PY{l+s+s2}{\PYZdq{}} \PY{o}{\PYZpc{}}\PY{k}{s}[1::2]) \PYZsh{}(0\PYZhy{}based indexing)
         \PY{n+nb}{print}\PY{p}{(}\PY{l+s+s2}{\PYZdq{}}\PY{l+s+s2}{The last five characters are }\PY{l+s+s2}{\PYZsq{}}\PY{l+s+si}{\PYZpc{}s}\PY{l+s+s2}{\PYZsq{}}\PY{l+s+s2}{\PYZdq{}} \PY{o}{\PYZpc{}} \PY{n}{s}\PY{p}{[}\PY{o}{\PYZhy{}}\PY{l+m+mi}{5}\PY{p}{:}\PY{p}{]}\PY{p}{)} \PY{c+c1}{\PYZsh{} 5th\PYZhy{}from\PYZhy{}last to end}
         
         \PY{c+c1}{\PYZsh{} Convert everything to uppercase}
         \PY{n+nb}{print}\PY{p}{(}\PY{l+s+s2}{\PYZdq{}}\PY{l+s+s2}{String in uppercase: }\PY{l+s+si}{\PYZpc{}s}\PY{l+s+s2}{\PYZdq{}} \PY{o}{\PYZpc{}} \PY{n}{s}\PY{o}{.}\PY{n}{upper}\PY{p}{(}\PY{p}{)}\PY{p}{)}
         
         \PY{c+c1}{\PYZsh{} Convert everything to lowercase}
         \PY{n+nb}{print}\PY{p}{(}\PY{l+s+s2}{\PYZdq{}}\PY{l+s+s2}{String in lowercase: }\PY{l+s+si}{\PYZpc{}s}\PY{l+s+s2}{\PYZdq{}} \PY{o}{\PYZpc{}} \PY{n}{s}\PY{o}{.}\PY{n}{lower}\PY{p}{(}\PY{p}{)}\PY{p}{)}
         
         \PY{c+c1}{\PYZsh{} Check how a string starts}
         \PY{k}{if} \PY{n}{s}\PY{o}{.}\PY{n}{startswith}\PY{p}{(}\PY{l+s+s2}{\PYZdq{}}\PY{l+s+s2}{Str}\PY{l+s+s2}{\PYZdq{}}\PY{p}{)}\PY{p}{:}
             \PY{n+nb}{print}\PY{p}{(}\PY{l+s+s2}{\PYZdq{}}\PY{l+s+s2}{String starts with }\PY{l+s+s2}{\PYZsq{}}\PY{l+s+s2}{Str}\PY{l+s+s2}{\PYZsq{}}\PY{l+s+s2}{. Good!}\PY{l+s+s2}{\PYZdq{}}\PY{p}{)}
         
         \PY{c+c1}{\PYZsh{} Check how a string ends}
         \PY{k}{if} \PY{n}{s}\PY{o}{.}\PY{n}{endswith}\PY{p}{(}\PY{l+s+s2}{\PYZdq{}}\PY{l+s+s2}{ome!}\PY{l+s+s2}{\PYZdq{}}\PY{p}{)}\PY{p}{:}
             \PY{n+nb}{print}\PY{p}{(}\PY{l+s+s2}{\PYZdq{}}\PY{l+s+s2}{String ends with }\PY{l+s+s2}{\PYZsq{}}\PY{l+s+s2}{ome!}\PY{l+s+s2}{\PYZsq{}}\PY{l+s+s2}{. Good!}\PY{l+s+s2}{\PYZdq{}}\PY{p}{)}
         
         \PY{c+c1}{\PYZsh{} Split the string into three separate strings,}
         \PY{c+c1}{\PYZsh{} each containing only a word}
         \PY{n+nb}{print}\PY{p}{(}\PY{l+s+s2}{\PYZdq{}}\PY{l+s+s2}{Split the words of the string: }\PY{l+s+si}{\PYZpc{}s}\PY{l+s+s2}{\PYZdq{}} \PY{o}{\PYZpc{}} \PY{n}{s}\PY{o}{.}\PY{n}{split}\PY{p}{(}\PY{l+s+s2}{\PYZdq{}}\PY{l+s+s2}{ }\PY{l+s+s2}{\PYZdq{}}\PY{p}{)}\PY{p}{)}
\end{Verbatim}


    \begin{Verbatim}[commandchars=\\\{\}]
Length of s = 20
The first occurrence of the letter a = 8
a occurs 2 times
The first five characters are 'Strin'
The next five characters are 'gs ar'
The thirteenth character is 'a'
The characters with odd index are 'tig r wsm!'
The last five characters are 'some!'
String in uppercase: STRINGS ARE AWESOME!
String in lowercase: strings are awesome!
String starts with 'Str'. Good!
String ends with 'ome!'. Good!
Split the words of the string: ['Strings', 'are', 'awesome!']

    \end{Verbatim}

    \subsubsection{Conditions}\label{conditions}

Like most programming languages, Python uses boolean variables to
evaluate conditions (ie. True, False). Python will return these
variables when a conditional statement is evaluated.

    \begin{Verbatim}[commandchars=\\\{\}]
{\color{incolor}In [{\color{incolor}36}]:} \PY{n}{x} \PY{o}{=} \PY{l+m+mi}{2}
         \PY{n+nb}{print}\PY{p}{(}\PY{n}{x} \PY{o}{==} \PY{l+m+mi}{2}\PY{p}{)} \PY{c+c1}{\PYZsh{} prints out True}
         \PY{n+nb}{print}\PY{p}{(}\PY{n}{x} \PY{o}{==} \PY{l+m+mi}{3}\PY{p}{)} \PY{c+c1}{\PYZsh{} prints out False}
         \PY{n+nb}{print}\PY{p}{(}\PY{n}{x} \PY{o}{\PYZlt{}} \PY{l+m+mi}{3}\PY{p}{)} \PY{c+c1}{\PYZsh{} prints out True}
\end{Verbatim}


    \begin{Verbatim}[commandchars=\\\{\}]
True
False
True

    \end{Verbatim}

    \paragraph{Boolean Operators}\label{boolean-operators}

Boolean operators allow for more complex Boolean expressions. The first
examples of this we'll look at are "and" and "or".

    \begin{Verbatim}[commandchars=\\\{\}]
{\color{incolor}In [{\color{incolor}39}]:} \PY{n}{name} \PY{o}{=} \PY{l+s+s2}{\PYZdq{}}\PY{l+s+s2}{Jacob}\PY{l+s+s2}{\PYZdq{}}
         \PY{n}{age} \PY{o}{=} \PY{l+m+mi}{24}
         
         \PY{k}{if} \PY{n}{name} \PY{o}{==} \PY{l+s+s2}{\PYZdq{}}\PY{l+s+s2}{Jacob}\PY{l+s+s2}{\PYZdq{}} \PY{o+ow}{and} \PY{n}{age} \PY{o}{==} \PY{l+m+mi}{24}\PY{p}{:}
             \PY{n+nb}{print}\PY{p}{(}\PY{l+s+s2}{\PYZdq{}}\PY{l+s+s2}{Your name is Jacob, and you are also 24 years old.}\PY{l+s+s2}{\PYZdq{}}\PY{p}{)}
         
         \PY{k}{if} \PY{n}{name} \PY{o}{==} \PY{l+s+s2}{\PYZdq{}}\PY{l+s+s2}{Jacob}\PY{l+s+s2}{\PYZdq{}} \PY{o+ow}{or} \PY{n}{name} \PY{o}{==} \PY{l+s+s2}{\PYZdq{}}\PY{l+s+s2}{Ryan}\PY{l+s+s2}{\PYZdq{}}\PY{p}{:}
             \PY{n+nb}{print}\PY{p}{(}\PY{l+s+s2}{\PYZdq{}}\PY{l+s+s2}{Your name is either Jacob or Ryan.}\PY{l+s+s2}{\PYZdq{}}\PY{p}{)}
             
         \PY{k}{if} \PY{n}{name} \PY{o}{==} \PY{l+s+s2}{\PYZdq{}}\PY{l+s+s2}{Kevin}\PY{l+s+s2}{\PYZdq{}} \PY{o+ow}{or} \PY{n}{name} \PY{o}{==} \PY{l+s+s2}{\PYZdq{}}\PY{l+s+s2}{Jasper}\PY{l+s+s2}{\PYZdq{}}\PY{p}{:}
             \PY{n+nb}{print}\PY{p}{(}\PY{l+s+s2}{\PYZdq{}}\PY{l+s+s2}{Your name is Kasper.}\PY{l+s+s2}{\PYZdq{}}\PY{p}{)}
\end{Verbatim}


    \begin{Verbatim}[commandchars=\\\{\}]
Your name is Jacob, and you are also 24 years old.
Your name is either Jacob or Ryan.

    \end{Verbatim}

    The next operator we'll discuss here is the "in" operator. This can be
used to check if specific objects exist inside of an iterable object
container, like a list.

    \begin{Verbatim}[commandchars=\\\{\}]
{\color{incolor}In [{\color{incolor}40}]:} \PY{n}{name} \PY{o}{=} \PY{l+s+s2}{\PYZdq{}}\PY{l+s+s2}{Jacob}\PY{l+s+s2}{\PYZdq{}}
         \PY{n}{name\PYZus{}list} \PY{o}{=} \PY{p}{[}\PY{l+s+s2}{\PYZdq{}}\PY{l+s+s2}{Jacob}\PY{l+s+s2}{\PYZdq{}}\PY{p}{,}\PY{l+s+s2}{\PYZdq{}}\PY{l+s+s2}{Ryan}\PY{l+s+s2}{\PYZdq{}}\PY{p}{]}
         
         \PY{k}{if} \PY{n}{name} \PY{o+ow}{in} \PY{n}{name\PYZus{}list}\PY{p}{:}
             \PY{n+nb}{print}\PY{p}{(}\PY{l+s+s2}{\PYZdq{}}\PY{l+s+s2}{Your name is either Jacob or Ryan.}\PY{l+s+s2}{\PYZdq{}}\PY{p}{)}
             
         \PY{k}{if} \PY{n}{name} \PY{o+ow}{in} \PY{p}{[}\PY{l+s+s2}{\PYZdq{}}\PY{l+s+s2}{Bill Gates}\PY{l+s+s2}{\PYZdq{}}\PY{p}{,}\PY{l+s+s2}{\PYZdq{}}\PY{l+s+s2}{Elon Musk}\PY{l+s+s2}{\PYZdq{}}\PY{p}{]}\PY{p}{:}
             \PY{n+nb}{print}\PY{p}{(}\PY{l+s+s2}{\PYZdq{}}\PY{l+s+s2}{Steal their wallet.}\PY{l+s+s2}{\PYZdq{}}\PY{p}{)}
\end{Verbatim}


    \begin{Verbatim}[commandchars=\\\{\}]
Your name is either Jacob or Ryan.

    \end{Verbatim}

    One nice thing about Python is that instead of using brackets or
something like that to define code blocks, it uses indentation. While
this might seem strange, it actually makes for nicer looking code (which
in turn is a little easier to read). The standard Python indentation is
4 spaces, although tabs and any other space size will work, as long as
it is consistent. Notice that code blocks do not need any termination.
So instead of having relatively ugly looking code like this:

    \begin{Verbatim}[commandchars=\\\{\}]
{\color{incolor}In [{\color{incolor}41}]:} \PY{k}{if} \PY{o}{\PYZlt{}}\PY{n}{statement} \PY{o+ow}{is}\PY{o}{=}\PY{l+s+s2}{\PYZdq{}}\PY{l+s+s2}{\PYZdq{}} \PY{n}{true}\PY{o}{=}\PY{l+s+s2}{\PYZdq{}}\PY{l+s+s2}{\PYZdq{}}\PY{o}{\PYZgt{}}\PY{p}{:}
             \PY{o}{\PYZlt{}}\PY{n}{do} \PY{n}{something}\PY{o}{=}\PY{l+s+s2}{\PYZdq{}}\PY{l+s+s2}{\PYZdq{}}\PY{o}{\PYZgt{}}
             \PY{o}{.}\PY{o}{.}\PY{o}{.}\PY{o}{.}
             \PY{o}{.}\PY{o}{.}\PY{o}{.}\PY{o}{.}
         \PY{k}{elif} \PY{o}{\PYZlt{}}\PY{n}{another} \PY{n}{statement}\PY{o}{=}\PY{l+s+s2}{\PYZdq{}}\PY{l+s+s2}{\PYZdq{}} \PY{o+ow}{is}\PY{o}{=}\PY{l+s+s2}{\PYZdq{}}\PY{l+s+s2}{\PYZdq{}} \PY{n}{true}\PY{o}{=}\PY{l+s+s2}{\PYZdq{}}\PY{l+s+s2}{\PYZdq{}}\PY{o}{\PYZgt{}}\PY{p}{:} \PY{c+c1}{\PYZsh{} else if}
             \PY{o}{\PYZlt{}}\PY{n}{do} \PY{n}{something}\PY{o}{=}\PY{l+s+s2}{\PYZdq{}}\PY{l+s+s2}{\PYZdq{}} \PY{k}{else}\PY{o}{=}\PY{l+s+s2}{\PYZdq{}}\PY{l+s+s2}{\PYZdq{}}\PY{o}{\PYZgt{}}
             \PY{o}{.}\PY{o}{.}\PY{o}{.}\PY{o}{.}
             \PY{o}{.}\PY{o}{.}\PY{o}{.}\PY{o}{.}
         \PY{k}{else}\PY{p}{:}
             \PY{o}{\PYZlt{}}\PY{n}{do} \PY{n}{another}\PY{o}{=}\PY{l+s+s2}{\PYZdq{}}\PY{l+s+s2}{\PYZdq{}} \PY{n}{thing}\PY{o}{=}\PY{l+s+s2}{\PYZdq{}}\PY{l+s+s2}{\PYZdq{}}\PY{o}{\PYZgt{}}
             \PY{o}{.}\PY{o}{.}\PY{o}{.}\PY{o}{.}
             \PY{o}{.}\PY{o}{.}\PY{o}{.}\PY{o}{.}
         \PY{o}{\PYZlt{}}\PY{o}{/}\PY{n}{do}\PY{o}{\PYZgt{}}\PY{o}{\PYZlt{}}\PY{o}{/}\PY{n}{do}\PY{o}{\PYZgt{}}\PY{o}{\PYZlt{}}\PY{o}{/}\PY{n}{another}\PY{o}{\PYZgt{}}\PY{o}{\PYZlt{}}\PY{o}{/}\PY{n}{do}\PY{o}{\PYZgt{}}\PY{o}{\PYZlt{}}\PY{o}{/}\PY{n}{statement}\PY{o}{\PYZgt{}}
\end{Verbatim}


    \begin{Verbatim}[commandchars=\\\{\}]

          File "<ipython-input-41-091a22c48e42>", line 1
        if <statement is="" true="">:
           \^{}
    SyntaxError: invalid syntax
    

    \end{Verbatim}

    We get nicer code that looks like this:

    \begin{Verbatim}[commandchars=\\\{\}]
{\color{incolor}In [{\color{incolor}43}]:} \PY{n}{x} \PY{o}{=} \PY{l+m+mi}{2}
         \PY{k}{if} \PY{n}{x} \PY{o}{==} \PY{l+m+mi}{2}\PY{p}{:}
             \PY{n+nb}{print}\PY{p}{(}\PY{l+s+s2}{\PYZdq{}}\PY{l+s+s2}{x equals two!}\PY{l+s+s2}{\PYZdq{}}\PY{p}{)}
         \PY{k}{else}\PY{p}{:}
             \PY{n+nb}{print}\PY{p}{(}\PY{l+s+s2}{\PYZdq{}}\PY{l+s+s2}{x does not equal to two.}\PY{l+s+s2}{\PYZdq{}}\PY{p}{)}
             
         \PY{n}{y} \PY{o}{=} \PY{l+m+mi}{5}
         \PY{k}{if} \PY{n}{y} \PY{o}{==} \PY{l+m+mi}{2}\PY{p}{:}
             \PY{n+nb}{print}\PY{p}{(}\PY{l+s+s2}{\PYZdq{}}\PY{l+s+s2}{y equals two!}\PY{l+s+s2}{\PYZdq{}}\PY{p}{)}
         \PY{k}{else}\PY{p}{:}
             \PY{n+nb}{print}\PY{p}{(}\PY{l+s+s2}{\PYZdq{}}\PY{l+s+s2}{y does not equal to two.}\PY{l+s+s2}{\PYZdq{}}\PY{p}{)}
\end{Verbatim}


    \begin{Verbatim}[commandchars=\\\{\}]
x equals two!
y does not equal to two.

    \end{Verbatim}

    A statement is evaulated as true if one of the following is correct: -
The "True" boolean variable is given, or calculated using an expression,
such as an arithmetic comparison. - An object which is not considered
"empty" is passed.

Here are some examples for objects which are considered as empty: - An
empty string: "" - An empty list: {[}{]} - The number zero: 0 - The
false boolean variable: False

    Next up we'll talk a little bit about the "is" operator. While the
\texttt{==} operator calculates whether or not a variable is equal to
another, matching the values of the variables, the "is" operator matches
the instances themselves. Most of the time we will find ourselves using
the \texttt{==} operator instead, but it's important to know that we
have this option as well. Below are some examples:

    \begin{Verbatim}[commandchars=\\\{\}]
{\color{incolor}In [{\color{incolor}47}]:} \PY{n}{x} \PY{o}{=} \PY{p}{[}\PY{l+m+mi}{1}\PY{p}{,}\PY{l+m+mi}{2}\PY{p}{,}\PY{l+m+mi}{3}\PY{p}{]}
         \PY{n}{tx} \PY{o}{=} \PY{n+nb}{type}\PY{p}{(}\PY{n}{x}\PY{p}{)}
         \PY{n}{y} \PY{o}{=} \PY{p}{[}\PY{l+m+mi}{1}\PY{p}{,}\PY{l+m+mi}{2}\PY{p}{,}\PY{l+m+mi}{3}\PY{p}{]}
         \PY{n}{ty} \PY{o}{=} \PY{n+nb}{type}\PY{p}{(}\PY{n}{y}\PY{p}{)}
         \PY{n+nb}{print}\PY{p}{(}\PY{n}{x} \PY{o}{==} \PY{n}{y}\PY{p}{)} \PY{c+c1}{\PYZsh{} Prints out True}
         \PY{n+nb}{print}\PY{p}{(}\PY{n}{x} \PY{o+ow}{is} \PY{n}{y}\PY{p}{)} \PY{c+c1}{\PYZsh{} Prints out False}
         \PY{n+nb}{print}\PY{p}{(}\PY{n}{x} \PY{o+ow}{is} \PY{n}{x}\PY{p}{)} \PY{c+c1}{\PYZsh{} Prints out True}
         \PY{n+nb}{print}\PY{p}{(}\PY{n}{tx} \PY{o+ow}{is} \PY{n}{ty}\PY{p}{)} \PY{c+c1}{\PYZsh{} Prints out True}
\end{Verbatim}


    \begin{Verbatim}[commandchars=\\\{\}]
True
False
True
True

    \end{Verbatim}

    The last Boolean operator we'll discuss in this section is the "not"
operator. Whereas most programming lanuages use \texttt{!} for
indicating the inverse of a Boolean statement, Python uses the actual
word \texttt{not}.

    \begin{Verbatim}[commandchars=\\\{\}]
{\color{incolor}In [{\color{incolor}48}]:} \PY{n+nb}{print}\PY{p}{(}\PY{o+ow}{not} \PY{k+kc}{False}\PY{p}{)} \PY{c+c1}{\PYZsh{} Prints out True}
         \PY{n+nb}{print}\PY{p}{(}\PY{p}{(}\PY{o+ow}{not} \PY{k+kc}{False}\PY{p}{)} \PY{o}{==} \PY{p}{(}\PY{k+kc}{False}\PY{p}{)}\PY{p}{)} \PY{c+c1}{\PYZsh{} Prints out False}
\end{Verbatim}


    \begin{Verbatim}[commandchars=\\\{\}]
True
False

    \end{Verbatim}

    \paragraph{Exercise}\label{exercise}

Change the variables in the first section, so that each if statement
resolves as True. The solution is in the second cell below.

    \begin{Verbatim}[commandchars=\\\{\}]
{\color{incolor}In [{\color{incolor}49}]:} \PY{c+c1}{\PYZsh{} change this code}
         \PY{n}{number} \PY{o}{=} \PY{l+m+mi}{10}
         \PY{n}{second\PYZus{}number} \PY{o}{=} \PY{l+m+mi}{10}
         \PY{n}{first\PYZus{}array} \PY{o}{=} \PY{p}{[}\PY{p}{]}
         \PY{n}{second\PYZus{}array} \PY{o}{=} \PY{p}{[}\PY{l+m+mi}{1}\PY{p}{,}\PY{l+m+mi}{2}\PY{p}{,}\PY{l+m+mi}{3}\PY{p}{]}
         
         \PY{k}{if} \PY{n}{number} \PY{o}{\PYZgt{}} \PY{l+m+mi}{15}\PY{p}{:}
             \PY{n+nb}{print}\PY{p}{(}\PY{l+s+s2}{\PYZdq{}}\PY{l+s+s2}{1}\PY{l+s+s2}{\PYZdq{}}\PY{p}{)}
         
         \PY{k}{if} \PY{n}{first\PYZus{}array}\PY{p}{:}
             \PY{n+nb}{print}\PY{p}{(}\PY{l+s+s2}{\PYZdq{}}\PY{l+s+s2}{2}\PY{l+s+s2}{\PYZdq{}}\PY{p}{)}
         
         \PY{k}{if} \PY{n+nb}{len}\PY{p}{(}\PY{n}{second\PYZus{}array}\PY{p}{)} \PY{o}{==} \PY{l+m+mi}{2}\PY{p}{:}
             \PY{n+nb}{print}\PY{p}{(}\PY{l+s+s2}{\PYZdq{}}\PY{l+s+s2}{3}\PY{l+s+s2}{\PYZdq{}}\PY{p}{)}
         
         \PY{k}{if} \PY{n+nb}{len}\PY{p}{(}\PY{n}{first\PYZus{}array}\PY{p}{)} \PY{o}{+} \PY{n+nb}{len}\PY{p}{(}\PY{n}{second\PYZus{}array}\PY{p}{)} \PY{o}{==} \PY{l+m+mi}{5}\PY{p}{:}
             \PY{n+nb}{print}\PY{p}{(}\PY{l+s+s2}{\PYZdq{}}\PY{l+s+s2}{4}\PY{l+s+s2}{\PYZdq{}}\PY{p}{)}
         
         \PY{k}{if} \PY{n}{first\PYZus{}array} \PY{o+ow}{and} \PY{n}{first\PYZus{}array}\PY{p}{[}\PY{l+m+mi}{0}\PY{p}{]} \PY{o}{==} \PY{l+m+mi}{1}\PY{p}{:}
             \PY{n+nb}{print}\PY{p}{(}\PY{l+s+s2}{\PYZdq{}}\PY{l+s+s2}{5}\PY{l+s+s2}{\PYZdq{}}\PY{p}{)}
\end{Verbatim}


    \begin{Verbatim}[commandchars=\\\{\}]
{\color{incolor}In [{\color{incolor}50}]:} \PY{c+c1}{\PYZsh{} change this code}
         \PY{n}{number} \PY{o}{=} \PY{l+m+mi}{20}
         \PY{n}{second\PYZus{}number} \PY{o}{=} \PY{l+m+mi}{10}
         \PY{n}{first\PYZus{}array} \PY{o}{=} \PY{p}{[}\PY{l+m+mi}{1}\PY{p}{,}\PY{l+m+mi}{2}\PY{p}{,}\PY{l+m+mi}{3}\PY{p}{]}
         \PY{n}{second\PYZus{}array} \PY{o}{=} \PY{p}{[}\PY{l+m+mi}{1}\PY{p}{,}\PY{l+m+mi}{2}\PY{p}{]}
         
         \PY{k}{if} \PY{n}{number} \PY{o}{\PYZgt{}} \PY{l+m+mi}{15}\PY{p}{:}
             \PY{n+nb}{print}\PY{p}{(}\PY{l+s+s2}{\PYZdq{}}\PY{l+s+s2}{1}\PY{l+s+s2}{\PYZdq{}}\PY{p}{)}
         
         \PY{k}{if} \PY{n}{first\PYZus{}array}\PY{p}{:} \PY{c+c1}{\PYZsh{} an empty list causes this statement to be passed}
             \PY{n+nb}{print}\PY{p}{(}\PY{l+s+s2}{\PYZdq{}}\PY{l+s+s2}{2}\PY{l+s+s2}{\PYZdq{}}\PY{p}{)}
         
         \PY{k}{if} \PY{n+nb}{len}\PY{p}{(}\PY{n}{second\PYZus{}array}\PY{p}{)} \PY{o}{==} \PY{l+m+mi}{2}\PY{p}{:}
             \PY{n+nb}{print}\PY{p}{(}\PY{l+s+s2}{\PYZdq{}}\PY{l+s+s2}{3}\PY{l+s+s2}{\PYZdq{}}\PY{p}{)}
         
         \PY{k}{if} \PY{n+nb}{len}\PY{p}{(}\PY{n}{first\PYZus{}array}\PY{p}{)} \PY{o}{+} \PY{n+nb}{len}\PY{p}{(}\PY{n}{second\PYZus{}array}\PY{p}{)} \PY{o}{==} \PY{l+m+mi}{5}\PY{p}{:}
             \PY{n+nb}{print}\PY{p}{(}\PY{l+s+s2}{\PYZdq{}}\PY{l+s+s2}{4}\PY{l+s+s2}{\PYZdq{}}\PY{p}{)}
         
         \PY{k}{if} \PY{n}{first\PYZus{}array} \PY{o+ow}{and} \PY{n}{first\PYZus{}array}\PY{p}{[}\PY{l+m+mi}{0}\PY{p}{]} \PY{o}{==} \PY{l+m+mi}{1}\PY{p}{:} \PY{c+c1}{\PYZsh{} ie. if first\PYZus{}array isn\PYZsq{}t empty and }
             \PY{n+nb}{print}\PY{p}{(}\PY{l+s+s2}{\PYZdq{}}\PY{l+s+s2}{5}\PY{l+s+s2}{\PYZdq{}}\PY{p}{)}                          \PY{c+c1}{\PYZsh{} the first entry is 1}
\end{Verbatim}


    \begin{Verbatim}[commandchars=\\\{\}]
1
2
3
4
5

    \end{Verbatim}

    \subsubsection{Loops}\label{loops}

There are two types of loops in Python, both of which we'll go over
here. Python uses "for" and "while" loops.

\paragraph{The "for" Loop}\label{the-for-loop}

For loops iterate over a given sequence. For example,

    \begin{Verbatim}[commandchars=\\\{\}]
{\color{incolor}In [{\color{incolor}54}]:} \PY{n}{primes} \PY{o}{=} \PY{p}{[}\PY{l+m+mi}{2}\PY{p}{,} \PY{l+m+mi}{3}\PY{p}{,} \PY{l+m+mi}{5}\PY{p}{,} \PY{l+m+mi}{7}\PY{p}{]}
         \PY{k}{for} \PY{n}{prime} \PY{o+ow}{in} \PY{n}{primes}\PY{p}{:}
             \PY{n+nb}{print}\PY{p}{(}\PY{n}{prime}\PY{p}{)}
             
         \PY{n}{names} \PY{o}{=} \PY{p}{[}\PY{l+s+s1}{\PYZsq{}}\PY{l+s+s1}{Ryan}\PY{l+s+s1}{\PYZsq{}}\PY{p}{,}\PY{l+s+s1}{\PYZsq{}}\PY{l+s+s1}{Jacob}\PY{l+s+s1}{\PYZsq{}}\PY{p}{,}\PY{l+s+s1}{\PYZsq{}}\PY{l+s+s1}{Eric}\PY{l+s+s1}{\PYZsq{}}\PY{p}{]}
         \PY{k}{for} \PY{n}{i} \PY{o+ow}{in} \PY{n}{names}\PY{p}{:}
             \PY{n+nb}{print}\PY{p}{(}\PY{n}{i}\PY{p}{)}
\end{Verbatim}


    \begin{Verbatim}[commandchars=\\\{\}]
2
3
5
7
Ryan
Jacob
Eric

    \end{Verbatim}

    We can also use the \texttt{range()} function to iterate over a sequence
of numbers. Note, the \texttt{range()} function returns a ne list with
numbers of the specified range. Also keep in mind that this function is
\(0\) based (meaning that it's indexed 0,1,2...).

    \begin{Verbatim}[commandchars=\\\{\}]
{\color{incolor}In [{\color{incolor}60}]:} \PY{c+c1}{\PYZsh{} Prints out the numbers 0,1,2,3,4}
         \PY{k}{for} \PY{n}{x} \PY{o+ow}{in} \PY{n+nb}{range}\PY{p}{(}\PY{l+m+mi}{5}\PY{p}{)}\PY{p}{:}
             \PY{n+nb}{print}\PY{p}{(}\PY{n}{x}\PY{p}{)}
         
         \PY{c+c1}{\PYZsh{} Prints out 3,4,5}
         \PY{k}{for} \PY{n}{x} \PY{o+ow}{in} \PY{n+nb}{range}\PY{p}{(}\PY{l+m+mi}{3}\PY{p}{,} \PY{l+m+mi}{6}\PY{p}{)}\PY{p}{:}
             \PY{n+nb}{print}\PY{p}{(}\PY{n}{x}\PY{p}{)}
         
         \PY{c+c1}{\PYZsh{} Prints out 3,5,7}
         \PY{k}{for} \PY{n}{x} \PY{o+ow}{in} \PY{n+nb}{range}\PY{p}{(}\PY{l+m+mi}{3}\PY{p}{,} \PY{l+m+mi}{8}\PY{p}{,} \PY{l+m+mi}{2}\PY{p}{)}\PY{p}{:}
             \PY{n+nb}{print}\PY{p}{(}\PY{n}{x}\PY{p}{)}
\end{Verbatim}


    \begin{Verbatim}[commandchars=\\\{\}]
0
1
2
3
4
3
4
5
3
5
7

    \end{Verbatim}

    \paragraph{The "while" Loop}\label{the-while-loop}

While loops will repeat for as long as a certain Boolean condition is
met. Be careful not to get yourself into infinite loops here! Here's an
example:

    \begin{Verbatim}[commandchars=\\\{\}]
{\color{incolor}In [{\color{incolor}62}]:} \PY{c+c1}{\PYZsh{} Prints out 0,1,2,3,4}
         
         \PY{n}{count} \PY{o}{=} \PY{l+m+mi}{0}           \PY{c+c1}{\PYZsh{} Here we\PYZsq{}re initializing the variable }
         \PY{k}{while} \PY{n}{count} \PY{o}{\PYZlt{}} \PY{l+m+mi}{5}\PY{p}{:}    \PY{c+c1}{\PYZsh{} that we\PYZsq{}ll iterate along inside the loop.}
             \PY{n+nb}{print}\PY{p}{(}\PY{n}{count}\PY{p}{)}
             \PY{n}{count} \PY{o}{+}\PY{o}{=} \PY{l+m+mi}{1}  \PY{c+c1}{\PYZsh{} This is the same as count = count + 1}
\end{Verbatim}


    \begin{Verbatim}[commandchars=\\\{\}]
0
1
2
3
4

    \end{Verbatim}

    \paragraph{"break" and "continue"
Statements}\label{break-and-continue-statements}

"break" is used to exit for or while loop. On the other hand, "continue"
is used to skip the current block, and return to the "for" or "while"
statement. As usual, here are some examples.

    \begin{Verbatim}[commandchars=\\\{\}]
{\color{incolor}In [{\color{incolor}64}]:} \PY{c+c1}{\PYZsh{} Prints out 0,1,2,3,4}
         
         \PY{n}{count} \PY{o}{=} \PY{l+m+mi}{0}
         \PY{k}{while} \PY{k+kc}{True}\PY{p}{:}
             \PY{n+nb}{print}\PY{p}{(}\PY{n}{count}\PY{p}{)}
             \PY{n}{count} \PY{o}{+}\PY{o}{=} \PY{l+m+mi}{1}
             \PY{k}{if} \PY{n}{count} \PY{o}{\PYZgt{}}\PY{o}{=} \PY{l+m+mi}{5}\PY{p}{:}
                 \PY{k}{break}            \PY{c+c1}{\PYZsh{} here we are giving the command }
                                  \PY{c+c1}{\PYZsh{} to break the loop and move on}
         \PY{c+c1}{\PYZsh{} Prints out only odd numbers \PYZhy{} 1,3,5,7,9}
         \PY{k}{for} \PY{n}{x} \PY{o+ow}{in} \PY{n+nb}{range}\PY{p}{(}\PY{l+m+mi}{10}\PY{p}{)}\PY{p}{:}
             \PY{c+c1}{\PYZsh{} Check if x is even}
             \PY{k}{if} \PY{n}{x} \PY{o}{\PYZpc{}} \PY{l+m+mi}{2} \PY{o}{==} \PY{l+m+mi}{0}\PY{p}{:}
                 \PY{k}{continue}        \PY{c+c1}{\PYZsh{} here we\PYZsq{}re saying if x is even, skip past it. }
             \PY{n+nb}{print}\PY{p}{(}\PY{n}{x}\PY{p}{)}            \PY{c+c1}{\PYZsh{} Otherwise, print x}
\end{Verbatim}


    \begin{Verbatim}[commandchars=\\\{\}]
0
1
2
3
4
1
3
5
7
9

    \end{Verbatim}

    \paragraph{What About "else"?}\label{what-about-else}

In Python, we can use "else" for loops, unlike languages like C. When
the loop condition of "for" or "while" statement fails then code part in
"else" is executed. And similarly to our above explanations, if a
"break" statement is executed inside the loop then the "else" is
skipped; and even if there is a "continue" statement, the "else" part
will be executed.

    \begin{Verbatim}[commandchars=\\\{\}]
{\color{incolor}In [{\color{incolor}1}]:} \PY{c+c1}{\PYZsh{} Prints out 0,1,2,3,4 and then it prints \PYZdq{}count value reached 5\PYZdq{}}
        
        \PY{n}{count}\PY{o}{=}\PY{l+m+mi}{0}
        \PY{k}{while}\PY{p}{(}\PY{n}{count}\PY{o}{\PYZlt{}}\PY{l+m+mi}{5}\PY{p}{)}\PY{p}{:}
            \PY{n+nb}{print}\PY{p}{(}\PY{n}{count}\PY{p}{)}
            \PY{n}{count} \PY{o}{+}\PY{o}{=}\PY{l+m+mi}{1}
        \PY{k}{else}\PY{p}{:}
            \PY{n+nb}{print}\PY{p}{(}\PY{l+s+s2}{\PYZdq{}}\PY{l+s+s2}{count value reached }\PY{l+s+si}{\PYZpc{}d}\PY{l+s+s2}{\PYZdq{}} \PY{o}{\PYZpc{}}\PY{p}{(}\PY{n}{count}\PY{p}{)}\PY{p}{)}
        
        \PY{c+c1}{\PYZsh{} Prints out 1,2,3,4}
        \PY{k}{for} \PY{n}{i} \PY{o+ow}{in} \PY{n+nb}{range}\PY{p}{(}\PY{l+m+mi}{1}\PY{p}{,} \PY{l+m+mi}{10}\PY{p}{)}\PY{p}{:}
            \PY{k}{if}\PY{p}{(}\PY{n}{i}\PY{o}{\PYZpc{}}\PY{k}{5}==0):
                \PY{k}{break}
            \PY{n+nb}{print}\PY{p}{(}\PY{n}{i}\PY{p}{)}
        \PY{k}{else}\PY{p}{:}
            \PY{n+nb}{print}\PY{p}{(}\PY{p}{(}\PY{l+s+s2}{\PYZdq{}}\PY{l+s+s2}{this is not printed because for loop is terminated because of break}\PY{l+s+s2}{\PYZdq{}}
                   \PY{l+s+s2}{\PYZdq{}}\PY{l+s+s2}{ but not due to fail in condition}\PY{l+s+s2}{\PYZdq{}}\PY{p}{)}\PY{p}{)}
\end{Verbatim}


    \begin{Verbatim}[commandchars=\\\{\}]
0
1
2
3
4
count value reached 5
1
2
3
4

    \end{Verbatim}

    \paragraph{Exercise}\label{exercise}

Loop through and print out all even numbers from the numbers list in the
same order they are received. Don't print any numbers that come after
\(237\) in the sequence. The solution is in the second code cell.

    \begin{Verbatim}[commandchars=\\\{\}]
{\color{incolor}In [{\color{incolor}2}]:} \PY{n}{numbers} \PY{o}{=} \PY{p}{[}
            \PY{l+m+mi}{951}\PY{p}{,} \PY{l+m+mi}{402}\PY{p}{,} \PY{l+m+mi}{984}\PY{p}{,} \PY{l+m+mi}{651}\PY{p}{,} \PY{l+m+mi}{360}\PY{p}{,} \PY{l+m+mi}{69}\PY{p}{,} \PY{l+m+mi}{408}\PY{p}{,} \PY{l+m+mi}{319}\PY{p}{,} \PY{l+m+mi}{601}\PY{p}{,} \PY{l+m+mi}{485}\PY{p}{,} \PY{l+m+mi}{980}\PY{p}{,} \PY{l+m+mi}{507}\PY{p}{,} \PY{l+m+mi}{725}\PY{p}{,} \PY{l+m+mi}{547}\PY{p}{,} \PY{l+m+mi}{544}\PY{p}{,}
            \PY{l+m+mi}{615}\PY{p}{,} \PY{l+m+mi}{83}\PY{p}{,} \PY{l+m+mi}{165}\PY{p}{,} \PY{l+m+mi}{141}\PY{p}{,} \PY{l+m+mi}{501}\PY{p}{,} \PY{l+m+mi}{263}\PY{p}{,} \PY{l+m+mi}{617}\PY{p}{,} \PY{l+m+mi}{865}\PY{p}{,} \PY{l+m+mi}{575}\PY{p}{,} \PY{l+m+mi}{219}\PY{p}{,} \PY{l+m+mi}{390}\PY{p}{,} \PY{l+m+mi}{984}\PY{p}{,} \PY{l+m+mi}{592}\PY{p}{,} \PY{l+m+mi}{236}\PY{p}{,} \PY{l+m+mi}{105}\PY{p}{,} \PY{l+m+mi}{942}\PY{p}{,} \PY{l+m+mi}{941}\PY{p}{,}
            \PY{l+m+mi}{386}\PY{p}{,} \PY{l+m+mi}{462}\PY{p}{,} \PY{l+m+mi}{47}\PY{p}{,} \PY{l+m+mi}{418}\PY{p}{,} \PY{l+m+mi}{907}\PY{p}{,} \PY{l+m+mi}{344}\PY{p}{,} \PY{l+m+mi}{236}\PY{p}{,} \PY{l+m+mi}{375}\PY{p}{,} \PY{l+m+mi}{823}\PY{p}{,} \PY{l+m+mi}{566}\PY{p}{,} \PY{l+m+mi}{597}\PY{p}{,} \PY{l+m+mi}{978}\PY{p}{,} \PY{l+m+mi}{328}\PY{p}{,} \PY{l+m+mi}{615}\PY{p}{,} \PY{l+m+mi}{953}\PY{p}{,} \PY{l+m+mi}{345}\PY{p}{,}
            \PY{l+m+mi}{399}\PY{p}{,} \PY{l+m+mi}{162}\PY{p}{,} \PY{l+m+mi}{758}\PY{p}{,} \PY{l+m+mi}{219}\PY{p}{,} \PY{l+m+mi}{918}\PY{p}{,} \PY{l+m+mi}{237}\PY{p}{,} \PY{l+m+mi}{412}\PY{p}{,} \PY{l+m+mi}{566}\PY{p}{,} \PY{l+m+mi}{826}\PY{p}{,} \PY{l+m+mi}{248}\PY{p}{,} \PY{l+m+mi}{866}\PY{p}{,} \PY{l+m+mi}{950}\PY{p}{,} \PY{l+m+mi}{626}\PY{p}{,} \PY{l+m+mi}{949}\PY{p}{,} \PY{l+m+mi}{687}\PY{p}{,} \PY{l+m+mi}{217}\PY{p}{,}
            \PY{l+m+mi}{815}\PY{p}{,} \PY{l+m+mi}{67}\PY{p}{,} \PY{l+m+mi}{104}\PY{p}{,} \PY{l+m+mi}{58}\PY{p}{,} \PY{l+m+mi}{512}\PY{p}{,} \PY{l+m+mi}{24}\PY{p}{,} \PY{l+m+mi}{892}\PY{p}{,} \PY{l+m+mi}{894}\PY{p}{,} \PY{l+m+mi}{767}\PY{p}{,} \PY{l+m+mi}{553}\PY{p}{,} \PY{l+m+mi}{81}\PY{p}{,} \PY{l+m+mi}{379}\PY{p}{,} \PY{l+m+mi}{843}\PY{p}{,} \PY{l+m+mi}{831}\PY{p}{,} \PY{l+m+mi}{445}\PY{p}{,} \PY{l+m+mi}{742}\PY{p}{,} \PY{l+m+mi}{717}\PY{p}{,}
            \PY{l+m+mi}{958}\PY{p}{,} \PY{l+m+mi}{609}\PY{p}{,} \PY{l+m+mi}{842}\PY{p}{,} \PY{l+m+mi}{451}\PY{p}{,} \PY{l+m+mi}{688}\PY{p}{,} \PY{l+m+mi}{753}\PY{p}{,} \PY{l+m+mi}{854}\PY{p}{,} \PY{l+m+mi}{685}\PY{p}{,} \PY{l+m+mi}{93}\PY{p}{,} \PY{l+m+mi}{857}\PY{p}{,} \PY{l+m+mi}{440}\PY{p}{,} \PY{l+m+mi}{380}\PY{p}{,} \PY{l+m+mi}{126}\PY{p}{,} \PY{l+m+mi}{721}\PY{p}{,} \PY{l+m+mi}{328}\PY{p}{,} \PY{l+m+mi}{753}\PY{p}{,} \PY{l+m+mi}{470}\PY{p}{,}
            \PY{l+m+mi}{743}\PY{p}{,} \PY{l+m+mi}{527}
        \PY{p}{]}
        
        \PY{c+c1}{\PYZsh{} your code goes here}
\end{Verbatim}


    \begin{Verbatim}[commandchars=\\\{\}]
{\color{incolor}In [{\color{incolor}7}]:} \PY{n}{numbers} \PY{o}{=} \PY{p}{[}
            \PY{l+m+mi}{951}\PY{p}{,} \PY{l+m+mi}{402}\PY{p}{,} \PY{l+m+mi}{984}\PY{p}{,} \PY{l+m+mi}{651}\PY{p}{,} \PY{l+m+mi}{360}\PY{p}{,} \PY{l+m+mi}{69}\PY{p}{,} \PY{l+m+mi}{408}\PY{p}{,} \PY{l+m+mi}{319}\PY{p}{,} \PY{l+m+mi}{601}\PY{p}{,} \PY{l+m+mi}{485}\PY{p}{,} \PY{l+m+mi}{980}\PY{p}{,} \PY{l+m+mi}{507}\PY{p}{,} \PY{l+m+mi}{725}\PY{p}{,} \PY{l+m+mi}{547}\PY{p}{,} \PY{l+m+mi}{544}\PY{p}{,}
            \PY{l+m+mi}{615}\PY{p}{,} \PY{l+m+mi}{83}\PY{p}{,} \PY{l+m+mi}{165}\PY{p}{,} \PY{l+m+mi}{141}\PY{p}{,} \PY{l+m+mi}{501}\PY{p}{,} \PY{l+m+mi}{263}\PY{p}{,} \PY{l+m+mi}{617}\PY{p}{,} \PY{l+m+mi}{865}\PY{p}{,} \PY{l+m+mi}{575}\PY{p}{,} \PY{l+m+mi}{219}\PY{p}{,} \PY{l+m+mi}{390}\PY{p}{,} \PY{l+m+mi}{984}\PY{p}{,} \PY{l+m+mi}{592}\PY{p}{,} \PY{l+m+mi}{236}\PY{p}{,} \PY{l+m+mi}{105}\PY{p}{,} \PY{l+m+mi}{942}\PY{p}{,} \PY{l+m+mi}{941}\PY{p}{,}
            \PY{l+m+mi}{386}\PY{p}{,} \PY{l+m+mi}{462}\PY{p}{,} \PY{l+m+mi}{47}\PY{p}{,} \PY{l+m+mi}{418}\PY{p}{,} \PY{l+m+mi}{907}\PY{p}{,} \PY{l+m+mi}{344}\PY{p}{,} \PY{l+m+mi}{236}\PY{p}{,} \PY{l+m+mi}{375}\PY{p}{,} \PY{l+m+mi}{823}\PY{p}{,} \PY{l+m+mi}{566}\PY{p}{,} \PY{l+m+mi}{597}\PY{p}{,} \PY{l+m+mi}{978}\PY{p}{,} \PY{l+m+mi}{328}\PY{p}{,} \PY{l+m+mi}{615}\PY{p}{,} \PY{l+m+mi}{953}\PY{p}{,} \PY{l+m+mi}{345}\PY{p}{,}
            \PY{l+m+mi}{399}\PY{p}{,} \PY{l+m+mi}{162}\PY{p}{,} \PY{l+m+mi}{758}\PY{p}{,} \PY{l+m+mi}{219}\PY{p}{,} \PY{l+m+mi}{918}\PY{p}{,} \PY{l+m+mi}{237}\PY{p}{,} \PY{l+m+mi}{412}\PY{p}{,} \PY{l+m+mi}{566}\PY{p}{,} \PY{l+m+mi}{826}\PY{p}{,} \PY{l+m+mi}{248}\PY{p}{,} \PY{l+m+mi}{866}\PY{p}{,} \PY{l+m+mi}{950}\PY{p}{,} \PY{l+m+mi}{626}\PY{p}{,} \PY{l+m+mi}{949}\PY{p}{,} \PY{l+m+mi}{687}\PY{p}{,} \PY{l+m+mi}{217}\PY{p}{,}
            \PY{l+m+mi}{815}\PY{p}{,} \PY{l+m+mi}{67}\PY{p}{,} \PY{l+m+mi}{104}\PY{p}{,} \PY{l+m+mi}{58}\PY{p}{,} \PY{l+m+mi}{512}\PY{p}{,} \PY{l+m+mi}{24}\PY{p}{,} \PY{l+m+mi}{892}\PY{p}{,} \PY{l+m+mi}{894}\PY{p}{,} \PY{l+m+mi}{767}\PY{p}{,} \PY{l+m+mi}{553}\PY{p}{,} \PY{l+m+mi}{81}\PY{p}{,} \PY{l+m+mi}{379}\PY{p}{,} \PY{l+m+mi}{843}\PY{p}{,} \PY{l+m+mi}{831}\PY{p}{,} \PY{l+m+mi}{445}\PY{p}{,} \PY{l+m+mi}{742}\PY{p}{,} \PY{l+m+mi}{717}\PY{p}{,}
            \PY{l+m+mi}{958}\PY{p}{,} \PY{l+m+mi}{609}\PY{p}{,} \PY{l+m+mi}{842}\PY{p}{,} \PY{l+m+mi}{451}\PY{p}{,} \PY{l+m+mi}{688}\PY{p}{,} \PY{l+m+mi}{753}\PY{p}{,} \PY{l+m+mi}{854}\PY{p}{,} \PY{l+m+mi}{685}\PY{p}{,} \PY{l+m+mi}{93}\PY{p}{,} \PY{l+m+mi}{857}\PY{p}{,} \PY{l+m+mi}{440}\PY{p}{,} \PY{l+m+mi}{380}\PY{p}{,} \PY{l+m+mi}{126}\PY{p}{,} \PY{l+m+mi}{721}\PY{p}{,} \PY{l+m+mi}{328}\PY{p}{,} \PY{l+m+mi}{753}\PY{p}{,} \PY{l+m+mi}{470}\PY{p}{,}
            \PY{l+m+mi}{743}\PY{p}{,} \PY{l+m+mi}{527}
        \PY{p}{]}
        
        \PY{k}{for} \PY{n}{i} \PY{o+ow}{in} \PY{n}{numbers}\PY{p}{:}
            
            \PY{k}{if} \PY{n}{i} \PY{o}{==} \PY{l+m+mi}{237}\PY{p}{:}
                \PY{k}{break}
                
            \PY{k}{if} \PY{n}{i} \PY{o}{\PYZpc{}} \PY{l+m+mi}{2} \PY{o}{==} \PY{l+m+mi}{1}\PY{p}{:}
                \PY{k}{continue}
                
            \PY{n+nb}{print}\PY{p}{(}\PY{n}{i}\PY{p}{)}
\end{Verbatim}


    \begin{Verbatim}[commandchars=\\\{\}]
402
984
360
408
980
544
390
984
592
236
942
386
462
418
344
236
566
978
328
162
758
918

    \end{Verbatim}

    \subsubsection{Functions}\label{functions}

Functions are a nice way to divide and organize code into blocks.
Functions often times make code more readable, and save time if we need
to do a task multiple times. So how do we write functions in Python?

As we saw with loops, Python makes use of blocks and indentation to
divide code. Below is an example of what a code block looks like (Ignore
the \#'s, they are being used to comment code out).

    \begin{Verbatim}[commandchars=\\\{\}]
{\color{incolor}In [{\color{incolor}10}]:} \PY{c+c1}{\PYZsh{}block\PYZus{}head:}
         \PY{c+c1}{\PYZsh{}    1st block line}
         \PY{c+c1}{\PYZsh{}    2nd block line}
         \PY{c+c1}{\PYZsh{}    ...}
\end{Verbatim}


    Functions in python are defined using the block keyword "def", followed
with the function's name as the block's name. For example:

    \begin{Verbatim}[commandchars=\\\{\}]
{\color{incolor}In [{\color{incolor}11}]:} \PY{k}{def} \PY{n+nf}{my\PYZus{}first\PYZus{}function}\PY{p}{(}\PY{p}{)}\PY{p}{:}
             \PY{n+nb}{print}\PY{p}{(}\PY{l+s+s2}{\PYZdq{}}\PY{l+s+s2}{Hey mom look at me!}\PY{l+s+s2}{\PYZdq{}}\PY{p}{)}
\end{Verbatim}


    Note that the function doesn't run until you call it:

    \begin{Verbatim}[commandchars=\\\{\}]
{\color{incolor}In [{\color{incolor}12}]:} \PY{n}{my\PYZus{}first\PYZus{}function}\PY{p}{(}\PY{p}{)}
\end{Verbatim}


    \begin{Verbatim}[commandchars=\\\{\}]
Hey mom look at me!

    \end{Verbatim}

    Functions can also receive arguments, or variables that are passed
through the caller to the function. Here is an example:

    \begin{Verbatim}[commandchars=\\\{\}]
{\color{incolor}In [{\color{incolor}13}]:} \PY{k}{def} \PY{n+nf}{my\PYZus{}first\PYZus{}function\PYZus{}wargs}\PY{p}{(}\PY{n}{username}\PY{p}{,} \PY{n}{greeting}\PY{p}{)}\PY{p}{:}
             \PY{n+nb}{print}\PY{p}{(}\PY{l+s+s2}{\PYZdq{}}\PY{l+s+s2}{Hello, }\PY{l+s+si}{\PYZpc{}s}\PY{l+s+s2}{ , From My Function!, I wish you }\PY{l+s+si}{\PYZpc{}s}\PY{l+s+s2}{\PYZdq{}} \PY{o}{\PYZpc{}} \PY{p}{(}\PY{n}{username}\PY{p}{,} \PY{n}{greeting}\PY{p}{)}\PY{p}{)}
\end{Verbatim}


    Then we call a function with arguments similarly to above, but put our
inputs in the parentheses in order.

    \begin{Verbatim}[commandchars=\\\{\}]
{\color{incolor}In [{\color{incolor}14}]:} \PY{n}{my\PYZus{}first\PYZus{}function\PYZus{}wargs}\PY{p}{(}\PY{l+s+s2}{\PYZdq{}}\PY{l+s+s2}{obewanjacobi}\PY{l+s+s2}{\PYZdq{}}\PY{p}{,}\PY{l+s+s2}{\PYZdq{}}\PY{l+s+s2}{were dead}\PY{l+s+s2}{\PYZdq{}}\PY{p}{)}
\end{Verbatim}


    \begin{Verbatim}[commandchars=\\\{\}]
Hello, obewanjacobi , From My Function!, I wish you were dead

    \end{Verbatim}

    Functions may also return a value to the caller. We do this by using the
return statement.

    \begin{Verbatim}[commandchars=\\\{\}]
{\color{incolor}In [{\color{incolor}15}]:} \PY{k}{def} \PY{n+nf}{sum\PYZus{}two\PYZus{}numbers}\PY{p}{(}\PY{n}{a}\PY{p}{,} \PY{n}{b}\PY{p}{)}\PY{p}{:}
             \PY{k}{return} \PY{n}{a} \PY{o}{+} \PY{n}{b}
\end{Verbatim}


    Then similarly, we can run the function immediately and have it print
the output, or we can save the output to a variable.

    \begin{Verbatim}[commandchars=\\\{\}]
{\color{incolor}In [{\color{incolor}17}]:} \PY{c+c1}{\PYZsh{} Run the statement only}
         \PY{n}{sum\PYZus{}two\PYZus{}numbers}\PY{p}{(}\PY{l+m+mi}{2}\PY{p}{,}\PY{l+m+mi}{3}\PY{p}{)}
         
         \PY{c+c1}{\PYZsh{} Return the output to the variable x}
         \PY{n}{x} \PY{o}{=} \PY{n}{sum\PYZus{}two\PYZus{}numbers}\PY{p}{(}\PY{l+m+mi}{5}\PY{p}{,}\PY{l+m+mi}{5}\PY{p}{)}
\end{Verbatim}


    \paragraph{Exercise}\label{exercise}

In this exercise you'll use an existing function, and while adding your
own to create a fully functional program.

\begin{itemize}
\item
  Add a function named \texttt{list\_benefits()} that returns the
  following list of strings: "More organized code", "More readable
  code", "Easier code reuse", "Allowing programmers to share and connect
  code together"
\item
  Add a function named \texttt{build\_sentence(info)} which receives a
  single argument containing a string and returns a sentence starting
  with the given string and ending with the string " is a benefit of
  functions!"
\item
  Run and see all the functions work together!
\end{itemize}

The solution is in the second code cell below.

    \begin{Verbatim}[commandchars=\\\{\}]
{\color{incolor}In [{\color{incolor}19}]:} \PY{c+c1}{\PYZsh{} Modify this function to return a list of strings as defined above}
         \PY{k}{def} \PY{n+nf}{list\PYZus{}benefits}\PY{p}{(}\PY{p}{)}\PY{p}{:}
             \PY{k}{pass}
         
         \PY{c+c1}{\PYZsh{} Modify this function to concatenate to each benefit }
         \PY{c+c1}{\PYZsh{} \PYZhy{} \PYZdq{} is a benefit of functions!\PYZdq{}}
         \PY{k}{def} \PY{n+nf}{build\PYZus{}sentence}\PY{p}{(}\PY{n}{benefit}\PY{p}{)}\PY{p}{:}
             \PY{k}{pass}
         
         \PY{k}{def} \PY{n+nf}{name\PYZus{}the\PYZus{}benefits\PYZus{}of\PYZus{}functions}\PY{p}{(}\PY{p}{)}\PY{p}{:}
             \PY{n}{list\PYZus{}of\PYZus{}benefits} \PY{o}{=} \PY{n}{list\PYZus{}benefits}\PY{p}{(}\PY{p}{)}
             \PY{k}{for} \PY{n}{benefit} \PY{o+ow}{in} \PY{n}{list\PYZus{}of\PYZus{}benefits}\PY{p}{:}
                 \PY{n+nb}{print}\PY{p}{(}\PY{n}{build\PYZus{}sentence}\PY{p}{(}\PY{n}{benefit}\PY{p}{)}\PY{p}{)}
\end{Verbatim}


    \begin{Verbatim}[commandchars=\\\{\}]
{\color{incolor}In [{\color{incolor}20}]:} \PY{c+c1}{\PYZsh{} Modify this function to return a list of strings as defined above}
         \PY{k}{def} \PY{n+nf}{list\PYZus{}benefits}\PY{p}{(}\PY{p}{)}\PY{p}{:}
             \PY{n}{benefits} \PY{o}{=} \PY{p}{[}\PY{l+s+s2}{\PYZdq{}}\PY{l+s+s2}{More organized code}\PY{l+s+s2}{\PYZdq{}}\PY{p}{,} \PY{l+s+s2}{\PYZdq{}}\PY{l+s+s2}{More readable code}\PY{l+s+s2}{\PYZdq{}}\PY{p}{,} 
                         \PY{l+s+s2}{\PYZdq{}}\PY{l+s+s2}{Easier code reuse}\PY{l+s+s2}{\PYZdq{}}\PY{p}{,} 
                         \PY{l+s+s2}{\PYZdq{}}\PY{l+s+s2}{Allowing programmers to share and connect code together}\PY{l+s+s2}{\PYZdq{}}\PY{p}{]}
             \PY{k}{return} \PY{n}{benefits}
         
         \PY{c+c1}{\PYZsh{} Modify this function to concatenate to each benefit }
         \PY{c+c1}{\PYZsh{} \PYZhy{} \PYZdq{} is a benefit of functions!\PYZdq{}}
         \PY{k}{def} \PY{n+nf}{build\PYZus{}sentence}\PY{p}{(}\PY{n}{benefit}\PY{p}{)}\PY{p}{:}
             \PY{k}{return} \PY{l+s+s2}{\PYZdq{}}\PY{l+s+si}{\PYZpc{}s}\PY{l+s+s2}{ is a benefit of functions!}\PY{l+s+s2}{\PYZdq{}} \PY{o}{\PYZpc{}} \PY{n}{benefit}
         
         \PY{k}{def} \PY{n+nf}{name\PYZus{}the\PYZus{}benefits\PYZus{}of\PYZus{}functions}\PY{p}{(}\PY{p}{)}\PY{p}{:}
             \PY{n}{list\PYZus{}of\PYZus{}benefits} \PY{o}{=} \PY{n}{list\PYZus{}benefits}\PY{p}{(}\PY{p}{)}
             \PY{k}{for} \PY{n}{benefit} \PY{o+ow}{in} \PY{n}{list\PYZus{}of\PYZus{}benefits}\PY{p}{:}
                 \PY{n+nb}{print}\PY{p}{(}\PY{n}{build\PYZus{}sentence}\PY{p}{(}\PY{n}{benefit}\PY{p}{)}\PY{p}{)}
         
         \PY{n}{name\PYZus{}the\PYZus{}benefits\PYZus{}of\PYZus{}functions}\PY{p}{(}\PY{p}{)}
\end{Verbatim}


    \begin{Verbatim}[commandchars=\\\{\}]
More organized code is a benefit of functions!
More readable code is a benefit of functions!
Easier code reuse is a benefit of functions!
Allowing programmers to share and connect code together is a benefit of functions!

    \end{Verbatim}

    \subsubsection{Classes and Objects}\label{classes-and-objects}

Objects are an encapsulation of variables and functions into a single
entity. Objects get their variables and functions from classes. Classes
are essentially a template to create your objects.

A very basic class would look something like this:

    \begin{Verbatim}[commandchars=\\\{\}]
{\color{incolor}In [{\color{incolor}1}]:} \PY{k}{class} \PY{n+nc}{MyClass}\PY{p}{:}
            \PY{n}{variable} \PY{o}{=} \PY{l+s+s2}{\PYZdq{}}\PY{l+s+s2}{blah}\PY{l+s+s2}{\PYZdq{}}
        
            \PY{k}{def} \PY{n+nf}{function}\PY{p}{(}\PY{n+nb+bp}{self}\PY{p}{)}\PY{p}{:}
                \PY{n+nb}{print}\PY{p}{(}\PY{l+s+s2}{\PYZdq{}}\PY{l+s+s2}{This is a message inside the class.}\PY{l+s+s2}{\PYZdq{}}\PY{p}{)}
\end{Verbatim}


    We'll explain why you have to include that "self" as a parameter a
little bit later. First, to assign the above class(template) to an
object you would do the following:

    \begin{Verbatim}[commandchars=\\\{\}]
{\color{incolor}In [{\color{incolor}2}]:} \PY{k}{class} \PY{n+nc}{MyClass}\PY{p}{:}
            \PY{n}{variable} \PY{o}{=} \PY{l+s+s2}{\PYZdq{}}\PY{l+s+s2}{blah}\PY{l+s+s2}{\PYZdq{}}
        
            \PY{k}{def} \PY{n+nf}{function}\PY{p}{(}\PY{n+nb+bp}{self}\PY{p}{)}\PY{p}{:}
                \PY{n+nb}{print}\PY{p}{(}\PY{l+s+s2}{\PYZdq{}}\PY{l+s+s2}{This is a message inside the class.}\PY{l+s+s2}{\PYZdq{}}\PY{p}{)}
        
        \PY{n}{myobjectx} \PY{o}{=} \PY{n}{MyClass}\PY{p}{(}\PY{p}{)}
\end{Verbatim}


    Now the variable "myobjectx" holds an object of the class "MyClass" that
contains the variable and the function defined within the class called
"MyClass". If this isn't making sense quite yet, fret not. More examples
should clear things up in no time.

    \paragraph{Accessing Object Variables}\label{accessing-object-variables}

Say you need a variable out of the class we just made above. To access
the variable inside of the newly created object "myobjectx" you would do
the following:

    \begin{Verbatim}[commandchars=\\\{\}]
{\color{incolor}In [{\color{incolor}3}]:} \PY{k}{class} \PY{n+nc}{MyClass}\PY{p}{:}
            \PY{n}{variable} \PY{o}{=} \PY{l+s+s2}{\PYZdq{}}\PY{l+s+s2}{blah}\PY{l+s+s2}{\PYZdq{}}
        
            \PY{k}{def} \PY{n+nf}{function}\PY{p}{(}\PY{n+nb+bp}{self}\PY{p}{)}\PY{p}{:}
                \PY{n+nb}{print}\PY{p}{(}\PY{l+s+s2}{\PYZdq{}}\PY{l+s+s2}{This is a message inside the class.}\PY{l+s+s2}{\PYZdq{}}\PY{p}{)}
        
        \PY{n}{myobjectx} \PY{o}{=} \PY{n}{MyClass}\PY{p}{(}\PY{p}{)}
        
        \PY{n}{myobjectx}\PY{o}{.}\PY{n}{variable}
\end{Verbatim}


\begin{Verbatim}[commandchars=\\\{\}]
{\color{outcolor}Out[{\color{outcolor}3}]:} 'blah'
\end{Verbatim}
            
    As you can see, when we call the class, put in a period, and then call
the variable we desire, it outputs that variable. Variables in classes
can come in handy when you have a lot of variables and need to organize
them in a clean fashion.

You can create multiple different objects that are of the same
class(have the same variables and functions defined). However, each
object contains independent copies of the variables defined in the
class. For instance, if we were to define another object with the
"MyClass" class and then change the string in the variable above:

    \begin{Verbatim}[commandchars=\\\{\}]
{\color{incolor}In [{\color{incolor}4}]:} \PY{k}{class} \PY{n+nc}{MyClass}\PY{p}{:}
            \PY{n}{variable} \PY{o}{=} \PY{l+s+s2}{\PYZdq{}}\PY{l+s+s2}{blah}\PY{l+s+s2}{\PYZdq{}}
        
            \PY{k}{def} \PY{n+nf}{function}\PY{p}{(}\PY{n+nb+bp}{self}\PY{p}{)}\PY{p}{:}
                \PY{n+nb}{print}\PY{p}{(}\PY{l+s+s2}{\PYZdq{}}\PY{l+s+s2}{This is a message inside the class.}\PY{l+s+s2}{\PYZdq{}}\PY{p}{)}
        
        \PY{n}{myobjectx} \PY{o}{=} \PY{n}{MyClass}\PY{p}{(}\PY{p}{)}
        \PY{n}{myobjecty} \PY{o}{=} \PY{n}{MyClass}\PY{p}{(}\PY{p}{)}
        
        \PY{n}{myobjecty}\PY{o}{.}\PY{n}{variable} \PY{o}{=} \PY{l+s+s2}{\PYZdq{}}\PY{l+s+s2}{yackity}\PY{l+s+s2}{\PYZdq{}}
        
        \PY{c+c1}{\PYZsh{} Then print out both values}
        \PY{n+nb}{print}\PY{p}{(}\PY{n}{myobjectx}\PY{o}{.}\PY{n}{variable}\PY{p}{)}
        \PY{n+nb}{print}\PY{p}{(}\PY{n}{myobjecty}\PY{o}{.}\PY{n}{variable}\PY{p}{)}
\end{Verbatim}


    \begin{Verbatim}[commandchars=\\\{\}]
blah
yackity

    \end{Verbatim}

    In the above example, we changed the variable in the class by assigning
it something new. But notice it didn't affect the class saved under
\texttt{myobjectx}. It also didn't change the base \texttt{MyClass}
class made, so if we were to make a \texttt{myobjectz\ =\ MyClass()}, we
would still get the original class.

\paragraph{Accessing Object Functions}\label{accessing-object-functions}

To access a function inside of an object you use notation similar to
accessing a variable:

    \begin{Verbatim}[commandchars=\\\{\}]
{\color{incolor}In [{\color{incolor}5}]:} \PY{k}{class} \PY{n+nc}{MyClass}\PY{p}{:}
            \PY{n}{variable} \PY{o}{=} \PY{l+s+s2}{\PYZdq{}}\PY{l+s+s2}{blah}\PY{l+s+s2}{\PYZdq{}}
        
            \PY{k}{def} \PY{n+nf}{function}\PY{p}{(}\PY{n+nb+bp}{self}\PY{p}{)}\PY{p}{:}
                \PY{n+nb}{print}\PY{p}{(}\PY{l+s+s2}{\PYZdq{}}\PY{l+s+s2}{This is a message inside the class.}\PY{l+s+s2}{\PYZdq{}}\PY{p}{)}
        
        \PY{n}{myobjectx} \PY{o}{=} \PY{n}{MyClass}\PY{p}{(}\PY{p}{)}
        
        \PY{n}{myobjectx}\PY{o}{.}\PY{n}{function}\PY{p}{(}\PY{p}{)}
\end{Verbatim}


    \begin{Verbatim}[commandchars=\\\{\}]
This is a message inside the class.

    \end{Verbatim}

    As you can see, this handles the same way as accessing a variable from a
class does.

\paragraph{Exercise}\label{exercise}

We have a class defined for vehicles. Create two new vehicles called
car1 and car2. Set car1 to be a red convertible worth
\texttt{\$}60,000.00 with a name of Fer, and car2 to be a blue van named
Jump worth \texttt{\$}10,000.00. The solution will be in the second cell
below.

    \begin{Verbatim}[commandchars=\\\{\}]
{\color{incolor}In [{\color{incolor}6}]:} \PY{c+c1}{\PYZsh{} define the Vehicle class}
        \PY{k}{class} \PY{n+nc}{Vehicle}\PY{p}{:}
            \PY{n}{name} \PY{o}{=} \PY{l+s+s2}{\PYZdq{}}\PY{l+s+s2}{\PYZdq{}}
            \PY{n}{kind} \PY{o}{=} \PY{l+s+s2}{\PYZdq{}}\PY{l+s+s2}{car}\PY{l+s+s2}{\PYZdq{}}
            \PY{n}{color} \PY{o}{=} \PY{l+s+s2}{\PYZdq{}}\PY{l+s+s2}{\PYZdq{}}
            \PY{n}{value} \PY{o}{=} \PY{l+m+mf}{100.00}
            \PY{k}{def} \PY{n+nf}{description}\PY{p}{(}\PY{n+nb+bp}{self}\PY{p}{)}\PY{p}{:}
                \PY{n}{desc\PYZus{}str} \PY{o}{=} \PY{p}{(}\PY{l+s+s2}{\PYZdq{}}\PY{l+s+si}{\PYZpc{}s}\PY{l+s+s2}{ is a }\PY{l+s+si}{\PYZpc{}s}\PY{l+s+s2}{ }\PY{l+s+si}{\PYZpc{}s}\PY{l+s+s2}{ worth \PYZdl{}}\PY{l+s+si}{\PYZpc{}.2f}\PY{l+s+s2}{.}\PY{l+s+s2}{\PYZdq{}} \PY{o}{\PYZpc{}} 
                            \PY{p}{(}\PY{n+nb+bp}{self}\PY{o}{.}\PY{n}{name}\PY{p}{,} \PY{n+nb+bp}{self}\PY{o}{.}\PY{n}{color}\PY{p}{,} \PY{n+nb+bp}{self}\PY{o}{.}\PY{n}{kind}\PY{p}{,} \PY{n+nb+bp}{self}\PY{o}{.}\PY{n}{value}\PY{p}{)}\PY{p}{)}
                \PY{k}{return} \PY{n}{desc\PYZus{}str}
        \PY{c+c1}{\PYZsh{} your code goes here}
        
        \PY{c+c1}{\PYZsh{} test code}
        \PY{n+nb}{print}\PY{p}{(}\PY{n}{car1}\PY{o}{.}\PY{n}{description}\PY{p}{(}\PY{p}{)}\PY{p}{)}
        \PY{n+nb}{print}\PY{p}{(}\PY{n}{car2}\PY{o}{.}\PY{n}{description}\PY{p}{(}\PY{p}{)}\PY{p}{)}
\end{Verbatim}


    \begin{Verbatim}[commandchars=\\\{\}]

        ---------------------------------------------------------------------------

        NameError                                 Traceback (most recent call last)

        <ipython-input-6-cd3d01fe16f8> in <module>()
         11 
         12 \# test code
    ---> 13 print(car1.description())
         14 print(car2.description())
    

        NameError: name 'car1' is not defined

    \end{Verbatim}

    \begin{Verbatim}[commandchars=\\\{\}]
{\color{incolor}In [{\color{incolor}50}]:} \PY{c+c1}{\PYZsh{} define the Vehicle class}
         \PY{k}{class} \PY{n+nc}{Vehicle}\PY{p}{:}
             \PY{n}{name} \PY{o}{=} \PY{l+s+s2}{\PYZdq{}}\PY{l+s+s2}{\PYZdq{}}
             \PY{n}{kind} \PY{o}{=} \PY{l+s+s2}{\PYZdq{}}\PY{l+s+s2}{car}\PY{l+s+s2}{\PYZdq{}}
             \PY{n}{color} \PY{o}{=} \PY{l+s+s2}{\PYZdq{}}\PY{l+s+s2}{\PYZdq{}}
             \PY{n}{value} \PY{o}{=} \PY{l+m+mf}{100.00}
             \PY{k}{def} \PY{n+nf}{description}\PY{p}{(}\PY{n+nb+bp}{self}\PY{p}{)}\PY{p}{:}
                  \PY{n}{desc\PYZus{}str} \PY{o}{=} \PY{p}{(}\PY{l+s+s2}{\PYZdq{}}\PY{l+s+si}{\PYZpc{}s}\PY{l+s+s2}{ is a }\PY{l+s+si}{\PYZpc{}s}\PY{l+s+s2}{ }\PY{l+s+si}{\PYZpc{}s}\PY{l+s+s2}{ worth \PYZdl{}}\PY{l+s+si}{\PYZpc{}.2f}\PY{l+s+s2}{.}\PY{l+s+s2}{\PYZdq{}} \PY{o}{\PYZpc{}} 
                             \PY{p}{(}\PY{n+nb+bp}{self}\PY{o}{.}\PY{n}{name}\PY{p}{,} \PY{n+nb+bp}{self}\PY{o}{.}\PY{n}{color}\PY{p}{,} \PY{n+nb+bp}{self}\PY{o}{.}\PY{n}{kind}\PY{p}{,} \PY{n+nb+bp}{self}\PY{o}{.}\PY{n}{value}\PY{p}{)}\PY{p}{)}
                  \PY{k}{return} \PY{n}{desc\PYZus{}str}
         \PY{c+c1}{\PYZsh{} your code goes here}
         
         \PY{n}{car1} \PY{o}{=} \PY{n}{Vehicle}\PY{p}{(}\PY{p}{)}
         \PY{n}{car1}\PY{o}{.}\PY{n}{name} \PY{o}{=} \PY{l+s+s1}{\PYZsq{}}\PY{l+s+s1}{Fer}\PY{l+s+s1}{\PYZsq{}}
         \PY{n}{car1}\PY{o}{.}\PY{n}{kind} \PY{o}{=} \PY{l+s+s1}{\PYZsq{}}\PY{l+s+s1}{convertible}\PY{l+s+s1}{\PYZsq{}}
         \PY{n}{car1}\PY{o}{.}\PY{n}{color} \PY{o}{=} \PY{l+s+s1}{\PYZsq{}}\PY{l+s+s1}{red}\PY{l+s+s1}{\PYZsq{}}
         \PY{n}{car1}\PY{o}{.}\PY{n}{value} \PY{o}{=} \PY{l+m+mf}{60000.00}
         
         \PY{n}{car2} \PY{o}{=} \PY{n}{Vehicle}\PY{p}{(}\PY{p}{)}
         \PY{n}{car2}\PY{o}{.}\PY{n}{name} \PY{o}{=} \PY{l+s+s1}{\PYZsq{}}\PY{l+s+s1}{Jump}\PY{l+s+s1}{\PYZsq{}}
         \PY{n}{car2}\PY{o}{.}\PY{n}{kind} \PY{o}{=} \PY{l+s+s1}{\PYZsq{}}\PY{l+s+s1}{van}\PY{l+s+s1}{\PYZsq{}}
         \PY{n}{car2}\PY{o}{.}\PY{n}{color} \PY{o}{=} \PY{l+s+s1}{\PYZsq{}}\PY{l+s+s1}{blue}\PY{l+s+s1}{\PYZsq{}}
         \PY{n}{car2}\PY{o}{.}\PY{n}{value} \PY{o}{=} \PY{l+m+mf}{10000.00}
         
         \PY{c+c1}{\PYZsh{} test code}
         \PY{n+nb}{print}\PY{p}{(}\PY{n}{car1}\PY{o}{.}\PY{n}{description}\PY{p}{(}\PY{p}{)}\PY{p}{)}
         \PY{n+nb}{print}\PY{p}{(}\PY{n}{car2}\PY{o}{.}\PY{n}{description}\PY{p}{(}\PY{p}{)}\PY{p}{)}
\end{Verbatim}


    \begin{Verbatim}[commandchars=\\\{\}]
Fer is a red convertible worth \$60000.00.
Jump is a blue van worth \$10000.00.

    \end{Verbatim}

    \subsubsection{Dictionaries}\label{dictionaries}

A dictionary is a data type in Python that is similar to an array, but
instead of working with indexes, it works with keys and values. Each
value stored in a dictionary can be accessed using a key, which is any
type of object (a string, a number, a list, etc.) instead of using its
index to address it.

For example, a database of phone numbers could be stored using a
dictionary like this:

    \begin{Verbatim}[commandchars=\\\{\}]
{\color{incolor}In [{\color{incolor}9}]:} \PY{n}{phonebook} \PY{o}{=} \PY{p}{\PYZob{}}\PY{p}{\PYZcb{}} \PY{c+c1}{\PYZsh{} initialize the dictionary}
        \PY{n}{phonebook}\PY{p}{[}\PY{l+s+s2}{\PYZdq{}}\PY{l+s+s2}{Frank}\PY{l+s+s2}{\PYZdq{}}\PY{p}{]} \PY{o}{=} \PY{l+m+mi}{5028477566}
        \PY{n}{phonebook}\PY{p}{[}\PY{l+s+s2}{\PYZdq{}}\PY{l+s+s2}{Stacy}\PY{l+s+s2}{\PYZdq{}}\PY{p}{]} \PY{o}{=} \PY{l+m+mi}{5028377264}
        \PY{n}{phonebook}\PY{p}{[}\PY{l+s+s2}{\PYZdq{}}\PY{l+s+s2}{Stacy}\PY{l+s+s2}{\PYZsq{}}\PY{l+s+s2}{s Mom}\PY{l+s+s2}{\PYZdq{}}\PY{p}{]} \PY{o}{=} \PY{l+m+mi}{5028675309}
        \PY{n+nb}{print}\PY{p}{(}\PY{n}{phonebook}\PY{p}{)}
\end{Verbatim}


    \begin{Verbatim}[commandchars=\\\{\}]
\{'Frank': 5028477566, 'Stacy': 5028377264, "Stacy's Mom": 5028675309\}

    \end{Verbatim}

    Alternatively, a dictionary can be initialized with the same values in
the following notation (this way you can make the whole dictionary in
one step):

    \begin{Verbatim}[commandchars=\\\{\}]
{\color{incolor}In [{\color{incolor}10}]:} \PY{n}{phonebook} \PY{o}{=} \PY{p}{\PYZob{}}
             \PY{l+s+s2}{\PYZdq{}}\PY{l+s+s2}{Frank}\PY{l+s+s2}{\PYZdq{}} \PY{p}{:} \PY{l+m+mi}{5028477566}\PY{p}{,}
             \PY{l+s+s2}{\PYZdq{}}\PY{l+s+s2}{Stacy}\PY{l+s+s2}{\PYZdq{}} \PY{p}{:} \PY{l+m+mi}{5028377264}\PY{p}{,}
             \PY{l+s+s2}{\PYZdq{}}\PY{l+s+s2}{Stacy}\PY{l+s+s2}{\PYZsq{}}\PY{l+s+s2}{s Mom}\PY{l+s+s2}{\PYZdq{}} \PY{p}{:} \PY{l+m+mi}{5028675309}
         \PY{p}{\PYZcb{}}
         \PY{n+nb}{print}\PY{p}{(}\PY{n}{phonebook}\PY{p}{)}
\end{Verbatim}


    \begin{Verbatim}[commandchars=\\\{\}]
\{'Frank': 5028477566, 'Stacy': 5028377264, "Stacy's Mom": 5028675309\}

    \end{Verbatim}

    Like lists, we can also iterate over a dictionary. However, a
dictionary, unlike a list, does not keep the order of the values stored
in it. To iterate over key value pairs, use the following syntax:

    \begin{Verbatim}[commandchars=\\\{\}]
{\color{incolor}In [{\color{incolor}11}]:} \PY{n}{phonebook} \PY{o}{=} \PY{p}{\PYZob{}}
             \PY{l+s+s2}{\PYZdq{}}\PY{l+s+s2}{Frank}\PY{l+s+s2}{\PYZdq{}} \PY{p}{:} \PY{l+m+mi}{5028477566}\PY{p}{,}
             \PY{l+s+s2}{\PYZdq{}}\PY{l+s+s2}{Stacy}\PY{l+s+s2}{\PYZdq{}} \PY{p}{:} \PY{l+m+mi}{5028377264}\PY{p}{,}
             \PY{l+s+s2}{\PYZdq{}}\PY{l+s+s2}{Stacy}\PY{l+s+s2}{\PYZsq{}}\PY{l+s+s2}{s Mom}\PY{l+s+s2}{\PYZdq{}} \PY{p}{:} \PY{l+m+mi}{5028675309}
         \PY{p}{\PYZcb{}}
         \PY{k}{for} \PY{n}{name}\PY{p}{,} \PY{n}{number} \PY{o+ow}{in} \PY{n}{phonebook}\PY{o}{.}\PY{n}{items}\PY{p}{(}\PY{p}{)}\PY{p}{:}
             \PY{n+nb}{print}\PY{p}{(}\PY{l+s+s2}{\PYZdq{}}\PY{l+s+s2}{Phone number of }\PY{l+s+si}{\PYZpc{}s}\PY{l+s+s2}{ is }\PY{l+s+si}{\PYZpc{}d}\PY{l+s+s2}{\PYZdq{}} \PY{o}{\PYZpc{}} \PY{p}{(}\PY{n}{name}\PY{p}{,} \PY{n}{number}\PY{p}{)}\PY{p}{)}
\end{Verbatim}


    \begin{Verbatim}[commandchars=\\\{\}]
Phone number of Frank is 5028477566
Phone number of Stacy is 5028377264
Phone number of Stacy's Mom is 5028675309

    \end{Verbatim}

    You can think of this as saying for each name and it's given number
inside the phonebook (which are called its items), do the command in the
loop. The variables \texttt{name} and \texttt{number} named in the for
loop are just placeholder values to help us as users keep track of
what's going on.

Say we want to remove a value from a dictionary. There are 2 ways to do
this, both are demonstrated below:

    \begin{Verbatim}[commandchars=\\\{\}]
{\color{incolor}In [{\color{incolor}16}]:} \PY{n}{phonebook} \PY{o}{=} \PY{p}{\PYZob{}}
             \PY{l+s+s2}{\PYZdq{}}\PY{l+s+s2}{Frank}\PY{l+s+s2}{\PYZdq{}} \PY{p}{:} \PY{l+m+mi}{5028477566}\PY{p}{,}
             \PY{l+s+s2}{\PYZdq{}}\PY{l+s+s2}{Stacy}\PY{l+s+s2}{\PYZdq{}} \PY{p}{:} \PY{l+m+mi}{5028377264}\PY{p}{,}
             \PY{l+s+s2}{\PYZdq{}}\PY{l+s+s2}{Stacy}\PY{l+s+s2}{\PYZsq{}}\PY{l+s+s2}{s Mom}\PY{l+s+s2}{\PYZdq{}} \PY{p}{:} \PY{l+m+mi}{5028675309}
         \PY{p}{\PYZcb{}}
         \PY{k}{del} \PY{n}{phonebook}\PY{p}{[}\PY{l+s+s2}{\PYZdq{}}\PY{l+s+s2}{Frank}\PY{l+s+s2}{\PYZdq{}}\PY{p}{]} \PY{c+c1}{\PYZsh{} Because who needs a dude\PYZsq{}s phone number?}
         \PY{n+nb}{print}\PY{p}{(}\PY{n}{phonebook}\PY{p}{)}
\end{Verbatim}


    \begin{Verbatim}[commandchars=\\\{\}]
\{'Stacy': 5028377264, "Stacy's Mom": 5028675309\}

    \end{Verbatim}

    \begin{Verbatim}[commandchars=\\\{\}]
{\color{incolor}In [{\color{incolor}17}]:} \PY{n}{phonebook}\PY{o}{.}\PY{n}{pop}\PY{p}{(}\PY{l+s+s2}{\PYZdq{}}\PY{l+s+s2}{Stacy}\PY{l+s+s2}{\PYZdq{}}\PY{p}{)} \PY{c+c1}{\PYZsh{} Stacy can\PYZsq{}t you see, you\PYZsq{}re just not the girl for me?}
         \PY{n+nb}{print}\PY{p}{(}\PY{n}{phonebook}\PY{p}{)}
\end{Verbatim}


    \begin{Verbatim}[commandchars=\\\{\}]
\{"Stacy's Mom": 5028675309\}

    \end{Verbatim}

    And now we have all the numbers in our dictionary that really matter.

\paragraph{Exercise}\label{exercise}

Add "Jake" to the phonebook with the phone number 5024152985, and remove
Frank from the phonebook. The solution is in the second code cell below.

    \begin{Verbatim}[commandchars=\\\{\}]
{\color{incolor}In [{\color{incolor}18}]:} \PY{n}{phonebook} \PY{o}{=} \PY{p}{\PYZob{}}
             \PY{l+s+s2}{\PYZdq{}}\PY{l+s+s2}{Frank}\PY{l+s+s2}{\PYZdq{}} \PY{p}{:} \PY{l+m+mi}{5028477566}\PY{p}{,}
             \PY{l+s+s2}{\PYZdq{}}\PY{l+s+s2}{Stacy}\PY{l+s+s2}{\PYZdq{}} \PY{p}{:} \PY{l+m+mi}{5028377264}\PY{p}{,}
             \PY{l+s+s2}{\PYZdq{}}\PY{l+s+s2}{Stacy}\PY{l+s+s2}{\PYZsq{}}\PY{l+s+s2}{s Mom}\PY{l+s+s2}{\PYZdq{}} \PY{p}{:} \PY{l+m+mi}{5028675309}
         \PY{p}{\PYZcb{}}
         
         \PY{c+c1}{\PYZsh{} write your code here}
         
         
         \PY{c+c1}{\PYZsh{} testing code}
         \PY{k}{if} \PY{l+s+s2}{\PYZdq{}}\PY{l+s+s2}{Jake}\PY{l+s+s2}{\PYZdq{}} \PY{o+ow}{in} \PY{n}{phonebook}\PY{p}{:}
             \PY{n+nb}{print}\PY{p}{(}\PY{l+s+s2}{\PYZdq{}}\PY{l+s+s2}{Jake is listed in the phonebook.}\PY{l+s+s2}{\PYZdq{}}\PY{p}{)}
         \PY{k}{if} \PY{l+s+s2}{\PYZdq{}}\PY{l+s+s2}{Frank}\PY{l+s+s2}{\PYZdq{}} \PY{o+ow}{not} \PY{o+ow}{in} \PY{n}{phonebook}\PY{p}{:}
             \PY{n+nb}{print}\PY{p}{(}\PY{l+s+s2}{\PYZdq{}}\PY{l+s+s2}{Frank is not listed in the phonebook.}\PY{l+s+s2}{\PYZdq{}}\PY{p}{)}
\end{Verbatim}


    \begin{Verbatim}[commandchars=\\\{\}]
{\color{incolor}In [{\color{incolor}19}]:} \PY{n}{phonebook} \PY{o}{=} \PY{p}{\PYZob{}}
             \PY{l+s+s2}{\PYZdq{}}\PY{l+s+s2}{Frank}\PY{l+s+s2}{\PYZdq{}} \PY{p}{:} \PY{l+m+mi}{5028477566}\PY{p}{,}
             \PY{l+s+s2}{\PYZdq{}}\PY{l+s+s2}{Stacy}\PY{l+s+s2}{\PYZdq{}} \PY{p}{:} \PY{l+m+mi}{5028377264}\PY{p}{,}
             \PY{l+s+s2}{\PYZdq{}}\PY{l+s+s2}{Stacy}\PY{l+s+s2}{\PYZsq{}}\PY{l+s+s2}{s Mom}\PY{l+s+s2}{\PYZdq{}} \PY{p}{:} \PY{l+m+mi}{5028675309}
         \PY{p}{\PYZcb{}}
         
         \PY{c+c1}{\PYZsh{} write your code here}
         
         \PY{n}{phonebook}\PY{o}{.}\PY{n}{pop}\PY{p}{(}\PY{l+s+s2}{\PYZdq{}}\PY{l+s+s2}{Frank}\PY{l+s+s2}{\PYZdq{}}\PY{p}{)}
         \PY{n}{phonebook}\PY{p}{[}\PY{l+s+s2}{\PYZdq{}}\PY{l+s+s2}{Jake}\PY{l+s+s2}{\PYZdq{}}\PY{p}{]} \PY{o}{=} \PY{l+m+mi}{5024152985}
         
         \PY{c+c1}{\PYZsh{} testing code}
         \PY{k}{if} \PY{l+s+s2}{\PYZdq{}}\PY{l+s+s2}{Jake}\PY{l+s+s2}{\PYZdq{}} \PY{o+ow}{in} \PY{n}{phonebook}\PY{p}{:}
             \PY{n+nb}{print}\PY{p}{(}\PY{l+s+s2}{\PYZdq{}}\PY{l+s+s2}{Jake is listed in the phonebook.}\PY{l+s+s2}{\PYZdq{}}\PY{p}{)}
         \PY{k}{if} \PY{l+s+s2}{\PYZdq{}}\PY{l+s+s2}{Frank}\PY{l+s+s2}{\PYZdq{}} \PY{o+ow}{not} \PY{o+ow}{in} \PY{n}{phonebook}\PY{p}{:}
             \PY{n+nb}{print}\PY{p}{(}\PY{l+s+s2}{\PYZdq{}}\PY{l+s+s2}{Frank is not listed in the phonebook.}\PY{l+s+s2}{\PYZdq{}}\PY{p}{)}
\end{Verbatim}


    \begin{Verbatim}[commandchars=\\\{\}]
Jake is listed in the phonebook.
Frank is not listed in the phonebook.

    \end{Verbatim}

    \subsubsection{Modules and Packages}\label{modules-and-packages}

We did it, we made it to the last lesson under "\textbf{Learn the
Basics}", feels like forever, doesn't it? This will be one of the more
complex sections we go over. But no need to worry, that's why I wrote
this little guy up. Hope this helps!

A module is a piece of software that has a specific functionality. For
example, imagine you're building an app in Python. When doing this for
example, you would have one module be responsible for the server, or
what is calculated and run in the background. Then you would have
another module to control the UI (user interface), and would control
what is presented on screen. In this example, each module is a different
file, and can be edited separately.

\paragraph{Writing Modules}\label{writing-modules}

In this section we will give outlines and templates of how to write your
own modules, but keep in mind that these modules don't work without some
actual code in them. For that reason, to prevent from printing errors,
we will simply leave the code commented out. If the code shows
\texttt{\#\#}, then it is an actual comment, whereas if the code shows
\texttt{\#}, then that's example code.

Modules in Python are simply Python files with a .py extension. The name
of the module will be the name of the file. A Python module can have a
set of functions, classes or variables defined and implemented. In the
example of building an application, we will have two files, we will
have:

    \begin{Verbatim}[commandchars=\\\{\}]
{\color{incolor}In [{\color{incolor}20}]:} \PY{c+c1}{\PYZsh{}myapp/            \PYZhy{} the directory where the modules are stored}
         \PY{c+c1}{\PYZsh{}myapp/server.py   \PYZhy{} the server module, what runs in the background}
         \PY{c+c1}{\PYZsh{}myapp/ui.py       \PYZhy{} the UI module, controlling what is printed on screen}
\end{Verbatim}


    The Python script \texttt{server.py} will implement the app. It will use
a function, maybe called \texttt{draw\_app} from the file
\texttt{ui.py}, or in other words, the \texttt{ui} module, that
implements the logic for printing the app on the screen.

Modules are imported from other modules using the \texttt{import}
command. In this example, the \texttt{server.py} script may look
something like this:

    \begin{Verbatim}[commandchars=\\\{\}]
{\color{incolor}In [{\color{incolor}21}]:} \PY{c+c1}{\PYZsh{}\PYZsh{} server.py}
         \PY{c+c1}{\PYZsh{}\PYZsh{} import the ui module}
         \PY{c+c1}{\PYZsh{}import ui}
         
         \PY{c+c1}{\PYZsh{}def play\PYZus{}app():}
         \PY{c+c1}{\PYZsh{}    ...}
         
         \PY{c+c1}{\PYZsh{}def main():}
         \PY{c+c1}{\PYZsh{}    result = play\PYZus{}app()}
         \PY{c+c1}{\PYZsh{}    ui.ui\PYZus{}app(result)}
         
         \PY{c+c1}{\PYZsh{}\PYZsh{} this means that if this script is executed, then }
         \PY{c+c1}{\PYZsh{}\PYZsh{} main() will be executed}
         \PY{c+c1}{\PYZsh{}if \PYZus{}\PYZus{}name\PYZus{}\PYZus{} == \PYZsq{}\PYZus{}\PYZus{}main\PYZus{}\PYZus{}\PYZsq{}:}
         \PY{c+c1}{\PYZsh{}    main()}
\end{Verbatim}


    And the \texttt{ui} module may look something like this:

    \begin{Verbatim}[commandchars=\\\{\}]
{\color{incolor}In [{\color{incolor}23}]:} \PY{c+c1}{\PYZsh{}\PYZsh{} ui.py}
         
         \PY{c+c1}{\PYZsh{}def ui\PYZus{}app():}
         \PY{c+c1}{\PYZsh{}    ...}
         
         \PY{c+c1}{\PYZsh{}def clear\PYZus{}screen(screen):}
         \PY{c+c1}{\PYZsh{}    ...}
\end{Verbatim}


    In this example, the \texttt{server} module imports the \texttt{load}
module, which enables it to use functions implemented in that module.
The \texttt{main} function would use the local function
\texttt{play\_app} to run the app, and then print the result of the app
using a function implemented in the \texttt{ui} module called
\texttt{ui\_app}. To use the function \texttt{ui\_app} from the
\texttt{ui} module, we would need to specify in which module the
function is implemented, using the dot operator. To reference the
\texttt{ui\_app} function from the \texttt{server} module, we would need
to import the \texttt{ui} module and only then call
\texttt{ui.ui\_app()}.

When the \texttt{import\ ui} directive will run, the Python interpreter
will look for a file in the directory which the script was executed
from, by the name of the module with a \texttt{.py} suffix, so in our
case it will try to look for \texttt{ui.py}. If it will find one, it
will import it. If not, he will continue to look for built-in modules.

You may have noticed that when importing a module, a \texttt{.pyc} file
appears, which is a compiled Python file. Python compiles files into
Python bytecode so that it won't have to parse the files each time
modules are loaded. If a \texttt{.pyc} file exists, it gets loaded
instead of the \texttt{.py} file, but this process is transparent to the
user.

\paragraph{Importing Module Objects to the Current
Namespace}\label{importing-module-objects-to-the-current-namespace}

We may also import the function \texttt{ui\_app} directly into the main
script's namespace, by using the \texttt{from} command.

    \begin{Verbatim}[commandchars=\\\{\}]
{\color{incolor}In [{\color{incolor}1}]:} \PY{c+c1}{\PYZsh{}\PYZsh{} app.py}
        \PY{c+c1}{\PYZsh{}\PYZsh{} import the ui module}
        \PY{c+c1}{\PYZsh{}from ui import ui\PYZus{}app}
        
        \PY{c+c1}{\PYZsh{}def main():}
        \PY{c+c1}{\PYZsh{}    result = play\PYZus{}app()}
        \PY{c+c1}{\PYZsh{}    ui\PYZus{}app(result)}
\end{Verbatim}


    You may have noticed that in this example, \texttt{ui\_app} does not
precede with the name of the module it is imported from, because we've
specified the module name in the \texttt{import} command.

The advantages of using this notation is that it is easier to use the
functions inside the current module because you don't need to specify
which module the function comes from. However, any namespace cannot have
two objects with the exact same name, so the \texttt{import} command may
replace an existing object in the namespace.

\paragraph{Import all Objects From a
Module}\label{import-all-objects-from-a-module}

We may also use the \texttt{import\ *} command to import all objects
from a specific module, like this:

    \begin{Verbatim}[commandchars=\\\{\}]
{\color{incolor}In [{\color{incolor}2}]:} \PY{c+c1}{\PYZsh{}\PYZsh{} app.py}
        \PY{c+c1}{\PYZsh{}\PYZsh{} import the ui module}
        \PY{c+c1}{\PYZsh{}from ui import *}
        
        \PY{c+c1}{\PYZsh{}def main():}
        \PY{c+c1}{\PYZsh{}    result = play\PYZus{}app()}
        \PY{c+c1}{\PYZsh{}    ui\PYZus{}app(result)}
\end{Verbatim}


    This might be a bit risky as changes in the module might affect the
module which imports it, but it is shorter and also does not require you
to specify which objects you wish to import from the module.

\paragraph{Custom Import Name}\label{custom-import-name}

We may also load modules under any name we want. This is useful when we
want to import a module conditionally to use the same name in the rest
of the code.

For example, if you have two \texttt{ui} modules with slighty different
names - you may do the following:

    \begin{Verbatim}[commandchars=\\\{\}]
{\color{incolor}In [{\color{incolor}3}]:} \PY{c+c1}{\PYZsh{}\PYZsh{} app.py}
        \PY{c+c1}{\PYZsh{}\PYZsh{} import the ui module}
        \PY{c+c1}{\PYZsh{}if visual\PYZus{}mode:}
        \PY{c+c1}{\PYZsh{}    \PYZsh{} in visual mode, we print using graphics}
        \PY{c+c1}{\PYZsh{}    import ui\PYZus{}visual as ui}
        \PY{c+c1}{\PYZsh{}else:}
        \PY{c+c1}{\PYZsh{}    \PYZsh{} in textual mode, we print out text}
        \PY{c+c1}{\PYZsh{}    import ui\PYZus{}textual as ui}
        
        \PY{c+c1}{\PYZsh{}def main():}
        \PY{c+c1}{\PYZsh{}    result = play\PYZus{}app()}
        \PY{c+c1}{\PYZsh{}    \PYZsh{} this can either be visual or textual depending on visual\PYZus{}mode}
        \PY{c+c1}{\PYZsh{}    ui.ui\PYZus{}app(result)}
\end{Verbatim}


    \paragraph{Module Initialization}\label{module-initialization}

The first time a module is loaded into a running Python script, it is
initialized by executing the code in the module once. If another module
in your code imports the same module again, it will not be loaded twice
but once only - so local variables inside the module act as a
"singleton" - they are initialized only once.

This is useful to know, because this means that you can rely on this
behavior for initializing objects. For example:

    \begin{Verbatim}[commandchars=\\\{\}]
{\color{incolor}In [{\color{incolor}4}]:} \PY{c+c1}{\PYZsh{}\PYZsh{} ui.py}
        
        \PY{c+c1}{\PYZsh{}def ui\PYZus{}app():}
        \PY{c+c1}{\PYZsh{}    \PYZsh{} when clearing the screen we can use the main }
        \PY{c+c1}{\PYZsh{}    \PYZsh{} screen object initialized in this module}
        \PY{c+c1}{\PYZsh{}    clear\PYZus{}screen(main\PYZus{}screen)}
        \PY{c+c1}{\PYZsh{}    ...}
        
        \PY{c+c1}{\PYZsh{}def clear\PYZus{}screen(screen):}
        \PY{c+c1}{\PYZsh{}    ...}
        
        \PY{c+c1}{\PYZsh{}class Screen():}
        \PY{c+c1}{\PYZsh{}    ...}
        
        \PY{c+c1}{\PYZsh{}\PYZsh{} initialize main\PYZus{}screen as a singleton}
        \PY{c+c1}{\PYZsh{}main\PYZus{}screen = Screen()}
\end{Verbatim}


    \paragraph{Extending Module Load Path}\label{extending-module-load-path}

There are a couple of ways we could tell the Python interpreter where to
look for modules, aside from the default, which is the local directory
and the built-in modules. You could either use the environment variable
\texttt{PYTHONPATH} to specify additional directories to look for
modules in, like this:

    \begin{Verbatim}[commandchars=\\\{\}]
{\color{incolor}In [{\color{incolor}5}]:} \PY{c+c1}{\PYZsh{}PYTHONPATH=/foo python app.py}
\end{Verbatim}


    \begin{Verbatim}[commandchars=\\\{\}]

          File "<ipython-input-5-a2d2c7aeb4ec>", line 1
        PYTHONPATH=/foo python app.py
                   \^{}
    SyntaxError: invalid syntax
    

    \end{Verbatim}

    This will execute \texttt{app.py}, and will enable the script to load
modules from the foo directory as well as the local directory.

Another method is the sys.path.append function. You may execute it
before running an import command:

    \begin{Verbatim}[commandchars=\\\{\}]
{\color{incolor}In [{\color{incolor}6}]:} \PY{c+c1}{\PYZsh{}sys.path.append(\PYZdq{}/foo\PYZdq{})}
\end{Verbatim}


    \begin{Verbatim}[commandchars=\\\{\}]

        ---------------------------------------------------------------------------

        NameError                                 Traceback (most recent call last)

        <ipython-input-6-cff76065f6a5> in <module>()
    ----> 1 sys.path.append("/foo")
    

        NameError: name 'sys' is not defined

    \end{Verbatim}

    This will add the \texttt{foo} directory to the list of paths to look
for modules in as well.

    \paragraph{Exploring Built-In Modules}\label{exploring-built-in-modules}

Check out the full list of built-in modules in the Python standard
library here: https://docs.python.org/3/library/.

Two very important functions come in handy when exploring modules in
Python - the \texttt{dir} and \texttt{help} functions.

If we want to import the module \texttt{urllib}, which enables us to
create read data from URLs, we simply \texttt{import} the module:

    \begin{Verbatim}[commandchars=\\\{\}]
{\color{incolor}In [{\color{incolor}8}]:} \PY{c+c1}{\PYZsh{} import the library}
        \PY{k+kn}{import} \PY{n+nn}{urllib}
        
        \PY{c+c1}{\PYZsh{}\PYZsh{} use it}
        \PY{c+c1}{\PYZsh{}urllib.urlopen(...)}
\end{Verbatim}


    We can look for which functions are implemented in each module by using
the \texttt{dir} function:

    \begin{Verbatim}[commandchars=\\\{\}]
{\color{incolor}In [{\color{incolor}9}]:} \PY{k+kn}{import} \PY{n+nn}{urllib}
        \PY{n+nb}{dir}\PY{p}{(}\PY{n}{urllib}\PY{p}{)}
\end{Verbatim}


\begin{Verbatim}[commandchars=\\\{\}]
{\color{outcolor}Out[{\color{outcolor}9}]:} ['\_\_builtins\_\_',
         '\_\_cached\_\_',
         '\_\_doc\_\_',
         '\_\_file\_\_',
         '\_\_loader\_\_',
         '\_\_name\_\_',
         '\_\_package\_\_',
         '\_\_path\_\_',
         '\_\_spec\_\_',
         'error',
         'parse',
         'request',
         'response']
\end{Verbatim}
            
    When we find the function in the module we want to use, we can read
about it more using the \texttt{help} function, inside the Python
interpreter.

    \begin{Verbatim}[commandchars=\\\{\}]
{\color{incolor}In [{\color{incolor}11}]:} \PY{n}{help}\PY{p}{(}\PY{n}{urllib}\PY{o}{.}\PY{n}{parse}\PY{p}{)}
\end{Verbatim}


    \begin{Verbatim}[commandchars=\\\{\}]
Help on module urllib.parse in urllib:

NAME
    urllib.parse - Parse (absolute and relative) URLs.

DESCRIPTION
    urlparse module is based upon the following RFC specifications.
    
    RFC 3986 (STD66): "Uniform Resource Identifiers" by T. Berners-Lee, R. Fielding
    and L.  Masinter, January 2005.
    
    RFC 2732 : "Format for Literal IPv6 Addresses in URL's by R.Hinden, B.Carpenter
    and L.Masinter, December 1999.
    
    RFC 2396:  "Uniform Resource Identifiers (URI)": Generic Syntax by T.
    Berners-Lee, R. Fielding, and L. Masinter, August 1998.
    
    RFC 2368: "The mailto URL scheme", by P.Hoffman , L Masinter, J. Zawinski, July 1998.
    
    RFC 1808: "Relative Uniform Resource Locators", by R. Fielding, UC Irvine, June
    1995.
    
    RFC 1738: "Uniform Resource Locators (URL)" by T. Berners-Lee, L. Masinter, M.
    McCahill, December 1994
    
    RFC 3986 is considered the current standard and any future changes to
    urlparse module should conform with it.  The urlparse module is
    currently not entirely compliant with this RFC due to defacto
    scenarios for parsing, and for backward compatibility purposes, some
    parsing quirks from older RFCs are retained. The testcases in
    test\_urlparse.py provides a good indicator of parsing behavior.

CLASSES
    DefragResult(builtins.tuple)
        DefragResult(DefragResult, \_ResultMixinStr)
        DefragResultBytes(DefragResult, \_ResultMixinBytes)
    ParseResult(builtins.tuple)
        ParseResult(ParseResult, \_NetlocResultMixinStr)
        ParseResultBytes(ParseResult, \_NetlocResultMixinBytes)
    SplitResult(builtins.tuple)
        SplitResult(SplitResult, \_NetlocResultMixinStr)
        SplitResultBytes(SplitResult, \_NetlocResultMixinBytes)
    \_NetlocResultMixinBytes(\_NetlocResultMixinBase, \_ResultMixinBytes)
        ParseResultBytes(ParseResult, \_NetlocResultMixinBytes)
        SplitResultBytes(SplitResult, \_NetlocResultMixinBytes)
    \_NetlocResultMixinStr(\_NetlocResultMixinBase, \_ResultMixinStr)
        ParseResult(ParseResult, \_NetlocResultMixinStr)
        SplitResult(SplitResult, \_NetlocResultMixinStr)
    \_ResultMixinBytes(builtins.object)
        DefragResultBytes(DefragResult, \_ResultMixinBytes)
    \_ResultMixinStr(builtins.object)
        DefragResult(DefragResult, \_ResultMixinStr)
    
    class DefragResult(DefragResult, \_ResultMixinStr)
     |  DefragResult(url, fragment)
     |  
     |  A 2-tuple that contains the url without fragment identifier and the fragment
     |  identifier as a separate argument.
     |  
     |  Method resolution order:
     |      DefragResult
     |      DefragResult
     |      builtins.tuple
     |      \_ResultMixinStr
     |      builtins.object
     |  
     |  Methods defined here:
     |  
     |  geturl(self)
     |  
     |  ----------------------------------------------------------------------
     |  Data and other attributes defined here:
     |  
     |  \_encoded\_counterpart = <class 'urllib.parse.DefragResultBytes'>
     |      DefragResult(url, fragment)
     |      
     |      A 2-tuple that contains the url without fragment identifier and the fragment
     |      identifier as a separate argument.
     |  
     |  ----------------------------------------------------------------------
     |  Methods inherited from DefragResult:
     |  
     |  \_\_getnewargs\_\_(self)
     |      Return self as a plain tuple.  Used by copy and pickle.
     |  
     |  \_\_repr\_\_(self)
     |      Return a nicely formatted representation string
     |  
     |  \_asdict(self)
     |      Return a new OrderedDict which maps field names to their values.
     |  
     |  \_replace(\_self, **kwds)
     |      Return a new DefragResult object replacing specified fields with new values
     |  
     |  ----------------------------------------------------------------------
     |  Class methods inherited from DefragResult:
     |  
     |  \_make(iterable, new=<built-in method \_\_new\_\_ of type object at 0x000000005CDCC0D0>, len=<built-in function len>) from builtins.type
     |      Make a new DefragResult object from a sequence or iterable
     |  
     |  ----------------------------------------------------------------------
     |  Static methods inherited from DefragResult:
     |  
     |  \_\_new\_\_(\_cls, url, fragment)
     |      Create new instance of DefragResult(url, fragment)
     |  
     |  ----------------------------------------------------------------------
     |  Data descriptors inherited from DefragResult:
     |  
     |  url
     |      The URL with no fragment identifier.
     |  
     |  fragment
     |      Fragment identifier separated from URL, that allows indirect identification of a
     |      secondary resource by reference to a primary resource and additional identifying
     |      information.
     |  
     |  ----------------------------------------------------------------------
     |  Data and other attributes inherited from DefragResult:
     |  
     |  \_fields = ('url', 'fragment')
     |  
     |  \_source = "from builtins import property as \_property, tupl{\ldots}\_itemget{\ldots}
     |  
     |  ----------------------------------------------------------------------
     |  Methods inherited from builtins.tuple:
     |  
     |  \_\_add\_\_(self, value, /)
     |      Return self+value.
     |  
     |  \_\_contains\_\_(self, key, /)
     |      Return key in self.
     |  
     |  \_\_eq\_\_(self, value, /)
     |      Return self==value.
     |  
     |  \_\_ge\_\_(self, value, /)
     |      Return self>=value.
     |  
     |  \_\_getattribute\_\_(self, name, /)
     |      Return getattr(self, name).
     |  
     |  \_\_getitem\_\_(self, key, /)
     |      Return self[key].
     |  
     |  \_\_gt\_\_(self, value, /)
     |      Return self>value.
     |  
     |  \_\_hash\_\_(self, /)
     |      Return hash(self).
     |  
     |  \_\_iter\_\_(self, /)
     |      Implement iter(self).
     |  
     |  \_\_le\_\_(self, value, /)
     |      Return self<=value.
     |  
     |  \_\_len\_\_(self, /)
     |      Return len(self).
     |  
     |  \_\_lt\_\_(self, value, /)
     |      Return self<value.
     |  
     |  \_\_mul\_\_(self, value, /)
     |      Return self*value.n
     |  
     |  \_\_ne\_\_(self, value, /)
     |      Return self!=value.
     |  
     |  \_\_rmul\_\_(self, value, /)
     |      Return self*value.
     |  
     |  count({\ldots})
     |      T.count(value) -> integer -- return number of occurrences of value
     |  
     |  index({\ldots})
     |      T.index(value, [start, [stop]]) -> integer -- return first index of value.
     |      Raises ValueError if the value is not present.
     |  
     |  ----------------------------------------------------------------------
     |  Methods inherited from \_ResultMixinStr:
     |  
     |  encode(self, encoding='ascii', errors='strict')
    
    class DefragResultBytes(DefragResult, \_ResultMixinBytes)
     |  DefragResult(url, fragment)
     |  
     |  A 2-tuple that contains the url without fragment identifier and the fragment
     |  identifier as a separate argument.
     |  
     |  Method resolution order:
     |      DefragResultBytes
     |      DefragResult
     |      builtins.tuple
     |      \_ResultMixinBytes
     |      builtins.object
     |  
     |  Methods defined here:
     |  
     |  geturl(self)
     |  
     |  ----------------------------------------------------------------------
     |  Data and other attributes defined here:
     |  
     |  \_decoded\_counterpart = <class 'urllib.parse.DefragResult'>
     |      DefragResult(url, fragment)
     |      
     |      A 2-tuple that contains the url without fragment identifier and the fragment
     |      identifier as a separate argument.
     |  
     |  ----------------------------------------------------------------------
     |  Methods inherited from DefragResult:
     |  
     |  \_\_getnewargs\_\_(self)
     |      Return self as a plain tuple.  Used by copy and pickle.
     |  
     |  \_\_repr\_\_(self)
     |      Return a nicely formatted representation string
     |  
     |  \_asdict(self)
     |      Return a new OrderedDict which maps field names to their values.
     |  
     |  \_replace(\_self, **kwds)
     |      Return a new DefragResult object replacing specified fields with new values
     |  
     |  ----------------------------------------------------------------------
     |  Class methods inherited from DefragResult:
     |  
     |  \_make(iterable, new=<built-in method \_\_new\_\_ of type object at 0x000000005CDCC0D0>, len=<built-in function len>) from builtins.type
     |      Make a new DefragResult object from a sequence or iterable
     |  
     |  ----------------------------------------------------------------------
     |  Static methods inherited from DefragResult:
     |  
     |  \_\_new\_\_(\_cls, url, fragment)
     |      Create new instance of DefragResult(url, fragment)
     |  
     |  ----------------------------------------------------------------------
     |  Data descriptors inherited from DefragResult:
     |  
     |  url
     |      The URL with no fragment identifier.
     |  
     |  fragment
     |      Fragment identifier separated from URL, that allows indirect identification of a
     |      secondary resource by reference to a primary resource and additional identifying
     |      information.
     |  
     |  ----------------------------------------------------------------------
     |  Data and other attributes inherited from DefragResult:
     |  
     |  \_fields = ('url', 'fragment')
     |  
     |  \_source = "from builtins import property as \_property, tupl{\ldots}\_itemget{\ldots}
     |  
     |  ----------------------------------------------------------------------
     |  Methods inherited from builtins.tuple:
     |  
     |  \_\_add\_\_(self, value, /)
     |      Return self+value.
     |  
     |  \_\_contains\_\_(self, key, /)
     |      Return key in self.
     |  
     |  \_\_eq\_\_(self, value, /)
     |      Return self==value.
     |  
     |  \_\_ge\_\_(self, value, /)
     |      Return self>=value.
     |  
     |  \_\_getattribute\_\_(self, name, /)
     |      Return getattr(self, name).
     |  
     |  \_\_getitem\_\_(self, key, /)
     |      Return self[key].
     |  
     |  \_\_gt\_\_(self, value, /)
     |      Return self>value.
     |  
     |  \_\_hash\_\_(self, /)
     |      Return hash(self).
     |  
     |  \_\_iter\_\_(self, /)
     |      Implement iter(self).
     |  
     |  \_\_le\_\_(self, value, /)
     |      Return self<=value.
     |  
     |  \_\_len\_\_(self, /)
     |      Return len(self).
     |  
     |  \_\_lt\_\_(self, value, /)
     |      Return self<value.
     |  
     |  \_\_mul\_\_(self, value, /)
     |      Return self*value.n
     |  
     |  \_\_ne\_\_(self, value, /)
     |      Return self!=value.
     |  
     |  \_\_rmul\_\_(self, value, /)
     |      Return self*value.
     |  
     |  count({\ldots})
     |      T.count(value) -> integer -- return number of occurrences of value
     |  
     |  index({\ldots})
     |      T.index(value, [start, [stop]]) -> integer -- return first index of value.
     |      Raises ValueError if the value is not present.
     |  
     |  ----------------------------------------------------------------------
     |  Methods inherited from \_ResultMixinBytes:
     |  
     |  decode(self, encoding='ascii', errors='strict')
    
    class ParseResult(ParseResult, \_NetlocResultMixinStr)
     |  ParseResult(scheme, netloc, path, params,  query, fragment)
     |  
     |  A 6-tuple that contains components of a parsed URL.
     |  
     |  Method resolution order:
     |      ParseResult
     |      ParseResult
     |      builtins.tuple
     |      \_NetlocResultMixinStr
     |      \_NetlocResultMixinBase
     |      \_ResultMixinStr
     |      builtins.object
     |  
     |  Methods defined here:
     |  
     |  geturl(self)
     |  
     |  ----------------------------------------------------------------------
     |  Data and other attributes defined here:
     |  
     |  \_encoded\_counterpart = <class 'urllib.parse.ParseResultBytes'>
     |      ParseResult(scheme, netloc, path, params,  query, fragment)
     |      
     |      A 6-tuple that contains components of a parsed URL.
     |  
     |  ----------------------------------------------------------------------
     |  Methods inherited from ParseResult:
     |  
     |  \_\_getnewargs\_\_(self)
     |      Return self as a plain tuple.  Used by copy and pickle.
     |  
     |  \_\_repr\_\_(self)
     |      Return a nicely formatted representation string
     |  
     |  \_asdict(self)
     |      Return a new OrderedDict which maps field names to their values.
     |  
     |  \_replace(\_self, **kwds)
     |      Return a new ParseResult object replacing specified fields with new values
     |  
     |  ----------------------------------------------------------------------
     |  Class methods inherited from ParseResult:
     |  
     |  \_make(iterable, new=<built-in method \_\_new\_\_ of type object at 0x000000005CDCC0D0>, len=<built-in function len>) from builtins.type
     |      Make a new ParseResult object from a sequence or iterable
     |  
     |  ----------------------------------------------------------------------
     |  Static methods inherited from ParseResult:
     |  
     |  \_\_new\_\_(\_cls, scheme, netloc, path, params, query, fragment)
     |      Create new instance of ParseResult(scheme, netloc, path, params, query, fragment)
     |  
     |  ----------------------------------------------------------------------
     |  Data descriptors inherited from ParseResult:
     |  
     |  scheme
     |      Specifies URL scheme for the request.
     |  
     |  netloc
     |      Network location where the request is made to.
     |  
     |  path
     |      The hierarchical path, such as the path to a file to download.
     |  
     |  params
     |      Parameters for last path element used to dereference the URI in order to provide
     |      access to perform some operation on the resource.
     |  
     |  query
     |      The query component, that contains non-hierarchical data, that along with data
     |      in path component, identifies a resource in the scope of URI's scheme and
     |      network location.
     |  
     |  fragment
     |      Fragment identifier, that allows indirect identification of a secondary resource
     |      by reference to a primary resource and additional identifying information.
     |  
     |  ----------------------------------------------------------------------
     |  Data and other attributes inherited from ParseResult:
     |  
     |  \_fields = ('scheme', 'netloc', 'path', 'params', 'query', 'fragment')
     |  
     |  \_source = "from builtins import property as \_property, tupl{\ldots}\_itemget{\ldots}
     |  
     |  ----------------------------------------------------------------------
     |  Methods inherited from builtins.tuple:
     |  
     |  \_\_add\_\_(self, value, /)
     |      Return self+value.
     |  
     |  \_\_contains\_\_(self, key, /)
     |      Return key in self.
     |  
     |  \_\_eq\_\_(self, value, /)
     |      Return self==value.
     |  
     |  \_\_ge\_\_(self, value, /)
     |      Return self>=value.
     |  
     |  \_\_getattribute\_\_(self, name, /)
     |      Return getattr(self, name).
     |  
     |  \_\_getitem\_\_(self, key, /)
     |      Return self[key].
     |  
     |  \_\_gt\_\_(self, value, /)
     |      Return self>value.
     |  
     |  \_\_hash\_\_(self, /)
     |      Return hash(self).
     |  
     |  \_\_iter\_\_(self, /)
     |      Implement iter(self).
     |  
     |  \_\_le\_\_(self, value, /)
     |      Return self<=value.
     |  
     |  \_\_len\_\_(self, /)
     |      Return len(self).
     |  
     |  \_\_lt\_\_(self, value, /)
     |      Return self<value.
     |  
     |  \_\_mul\_\_(self, value, /)
     |      Return self*value.n
     |  
     |  \_\_ne\_\_(self, value, /)
     |      Return self!=value.
     |  
     |  \_\_rmul\_\_(self, value, /)
     |      Return self*value.
     |  
     |  count({\ldots})
     |      T.count(value) -> integer -- return number of occurrences of value
     |  
     |  index({\ldots})
     |      T.index(value, [start, [stop]]) -> integer -- return first index of value.
     |      Raises ValueError if the value is not present.
     |  
     |  ----------------------------------------------------------------------
     |  Data descriptors inherited from \_NetlocResultMixinStr:
     |  
     |  \_hostinfo
     |  
     |  \_userinfo
     |  
     |  ----------------------------------------------------------------------
     |  Data descriptors inherited from \_NetlocResultMixinBase:
     |  
     |  hostname
     |  
     |  password
     |  
     |  port
     |  
     |  username
     |  
     |  ----------------------------------------------------------------------
     |  Methods inherited from \_ResultMixinStr:
     |  
     |  encode(self, encoding='ascii', errors='strict')
    
    class ParseResultBytes(ParseResult, \_NetlocResultMixinBytes)
     |  ParseResult(scheme, netloc, path, params,  query, fragment)
     |  
     |  A 6-tuple that contains components of a parsed URL.
     |  
     |  Method resolution order:
     |      ParseResultBytes
     |      ParseResult
     |      builtins.tuple
     |      \_NetlocResultMixinBytes
     |      \_NetlocResultMixinBase
     |      \_ResultMixinBytes
     |      builtins.object
     |  
     |  Methods defined here:
     |  
     |  geturl(self)
     |  
     |  ----------------------------------------------------------------------
     |  Data and other attributes defined here:
     |  
     |  \_decoded\_counterpart = <class 'urllib.parse.ParseResult'>
     |      ParseResult(scheme, netloc, path, params,  query, fragment)
     |      
     |      A 6-tuple that contains components of a parsed URL.
     |  
     |  ----------------------------------------------------------------------
     |  Methods inherited from ParseResult:
     |  
     |  \_\_getnewargs\_\_(self)
     |      Return self as a plain tuple.  Used by copy and pickle.
     |  
     |  \_\_repr\_\_(self)
     |      Return a nicely formatted representation string
     |  
     |  \_asdict(self)
     |      Return a new OrderedDict which maps field names to their values.
     |  
     |  \_replace(\_self, **kwds)
     |      Return a new ParseResult object replacing specified fields with new values
     |  
     |  ----------------------------------------------------------------------
     |  Class methods inherited from ParseResult:
     |  
     |  \_make(iterable, new=<built-in method \_\_new\_\_ of type object at 0x000000005CDCC0D0>, len=<built-in function len>) from builtins.type
     |      Make a new ParseResult object from a sequence or iterable
     |  
     |  ----------------------------------------------------------------------
     |  Static methods inherited from ParseResult:
     |  
     |  \_\_new\_\_(\_cls, scheme, netloc, path, params, query, fragment)
     |      Create new instance of ParseResult(scheme, netloc, path, params, query, fragment)
     |  
     |  ----------------------------------------------------------------------
     |  Data descriptors inherited from ParseResult:
     |  
     |  scheme
     |      Specifies URL scheme for the request.
     |  
     |  netloc
     |      Network location where the request is made to.
     |  
     |  path
     |      The hierarchical path, such as the path to a file to download.
     |  
     |  params
     |      Parameters for last path element used to dereference the URI in order to provide
     |      access to perform some operation on the resource.
     |  
     |  query
     |      The query component, that contains non-hierarchical data, that along with data
     |      in path component, identifies a resource in the scope of URI's scheme and
     |      network location.
     |  
     |  fragment
     |      Fragment identifier, that allows indirect identification of a secondary resource
     |      by reference to a primary resource and additional identifying information.
     |  
     |  ----------------------------------------------------------------------
     |  Data and other attributes inherited from ParseResult:
     |  
     |  \_fields = ('scheme', 'netloc', 'path', 'params', 'query', 'fragment')
     |  
     |  \_source = "from builtins import property as \_property, tupl{\ldots}\_itemget{\ldots}
     |  
     |  ----------------------------------------------------------------------
     |  Methods inherited from builtins.tuple:
     |  
     |  \_\_add\_\_(self, value, /)
     |      Return self+value.
     |  
     |  \_\_contains\_\_(self, key, /)
     |      Return key in self.
     |  
     |  \_\_eq\_\_(self, value, /)
     |      Return self==value.
     |  
     |  \_\_ge\_\_(self, value, /)
     |      Return self>=value.
     |  
     |  \_\_getattribute\_\_(self, name, /)
     |      Return getattr(self, name).
     |  
     |  \_\_getitem\_\_(self, key, /)
     |      Return self[key].
     |  
     |  \_\_gt\_\_(self, value, /)
     |      Return self>value.
     |  
     |  \_\_hash\_\_(self, /)
     |      Return hash(self).
     |  
     |  \_\_iter\_\_(self, /)
     |      Implement iter(self).
     |  
     |  \_\_le\_\_(self, value, /)
     |      Return self<=value.
     |  
     |  \_\_len\_\_(self, /)
     |      Return len(self).
     |  
     |  \_\_lt\_\_(self, value, /)
     |      Return self<value.
     |  
     |  \_\_mul\_\_(self, value, /)
     |      Return self*value.n
     |  
     |  \_\_ne\_\_(self, value, /)
     |      Return self!=value.
     |  
     |  \_\_rmul\_\_(self, value, /)
     |      Return self*value.
     |  
     |  count({\ldots})
     |      T.count(value) -> integer -- return number of occurrences of value
     |  
     |  index({\ldots})
     |      T.index(value, [start, [stop]]) -> integer -- return first index of value.
     |      Raises ValueError if the value is not present.
     |  
     |  ----------------------------------------------------------------------
     |  Data descriptors inherited from \_NetlocResultMixinBytes:
     |  
     |  \_hostinfo
     |  
     |  \_userinfo
     |  
     |  ----------------------------------------------------------------------
     |  Data descriptors inherited from \_NetlocResultMixinBase:
     |  
     |  hostname
     |  
     |  password
     |  
     |  port
     |  
     |  username
     |  
     |  ----------------------------------------------------------------------
     |  Methods inherited from \_ResultMixinBytes:
     |  
     |  decode(self, encoding='ascii', errors='strict')
    
    class SplitResult(SplitResult, \_NetlocResultMixinStr)
     |  SplitResult(scheme, netloc, path, query, fragment)
     |  
     |  A 5-tuple that contains the different components of a URL. Similar to
     |  ParseResult, but does not split params.
     |  
     |  Method resolution order:
     |      SplitResult
     |      SplitResult
     |      builtins.tuple
     |      \_NetlocResultMixinStr
     |      \_NetlocResultMixinBase
     |      \_ResultMixinStr
     |      builtins.object
     |  
     |  Methods defined here:
     |  
     |  geturl(self)
     |  
     |  ----------------------------------------------------------------------
     |  Data and other attributes defined here:
     |  
     |  \_encoded\_counterpart = <class 'urllib.parse.SplitResultBytes'>
     |      SplitResult(scheme, netloc, path, query, fragment)
     |      
     |      A 5-tuple that contains the different components of a URL. Similar to
     |      ParseResult, but does not split params.
     |  
     |  ----------------------------------------------------------------------
     |  Methods inherited from SplitResult:
     |  
     |  \_\_getnewargs\_\_(self)
     |      Return self as a plain tuple.  Used by copy and pickle.
     |  
     |  \_\_repr\_\_(self)
     |      Return a nicely formatted representation string
     |  
     |  \_asdict(self)
     |      Return a new OrderedDict which maps field names to their values.
     |  
     |  \_replace(\_self, **kwds)
     |      Return a new SplitResult object replacing specified fields with new values
     |  
     |  ----------------------------------------------------------------------
     |  Class methods inherited from SplitResult:
     |  
     |  \_make(iterable, new=<built-in method \_\_new\_\_ of type object at 0x000000005CDCC0D0>, len=<built-in function len>) from builtins.type
     |      Make a new SplitResult object from a sequence or iterable
     |  
     |  ----------------------------------------------------------------------
     |  Static methods inherited from SplitResult:
     |  
     |  \_\_new\_\_(\_cls, scheme, netloc, path, query, fragment)
     |      Create new instance of SplitResult(scheme, netloc, path, query, fragment)
     |  
     |  ----------------------------------------------------------------------
     |  Data descriptors inherited from SplitResult:
     |  
     |  scheme
     |      Specifies URL scheme for the request.
     |  
     |  netloc
     |      Network location where the request is made to.
     |  
     |  path
     |      The hierarchical path, such as the path to a file to download.
     |  
     |  query
     |      The query component, that contains non-hierarchical data, that along with data
     |      in path component, identifies a resource in the scope of URI's scheme and
     |      network location.
     |  
     |  fragment
     |      Fragment identifier, that allows indirect identification of a secondary resource
     |      by reference to a primary resource and additional identifying information.
     |  
     |  ----------------------------------------------------------------------
     |  Data and other attributes inherited from SplitResult:
     |  
     |  \_fields = ('scheme', 'netloc', 'path', 'query', 'fragment')
     |  
     |  \_source = "from builtins import property as \_property, tupl{\ldots}\_itemget{\ldots}
     |  
     |  ----------------------------------------------------------------------
     |  Methods inherited from builtins.tuple:
     |  
     |  \_\_add\_\_(self, value, /)
     |      Return self+value.
     |  
     |  \_\_contains\_\_(self, key, /)
     |      Return key in self.
     |  
     |  \_\_eq\_\_(self, value, /)
     |      Return self==value.
     |  
     |  \_\_ge\_\_(self, value, /)
     |      Return self>=value.
     |  
     |  \_\_getattribute\_\_(self, name, /)
     |      Return getattr(self, name).
     |  
     |  \_\_getitem\_\_(self, key, /)
     |      Return self[key].
     |  
     |  \_\_gt\_\_(self, value, /)
     |      Return self>value.
     |  
     |  \_\_hash\_\_(self, /)
     |      Return hash(self).
     |  
     |  \_\_iter\_\_(self, /)
     |      Implement iter(self).
     |  
     |  \_\_le\_\_(self, value, /)
     |      Return self<=value.
     |  
     |  \_\_len\_\_(self, /)
     |      Return len(self).
     |  
     |  \_\_lt\_\_(self, value, /)
     |      Return self<value.
     |  
     |  \_\_mul\_\_(self, value, /)
     |      Return self*value.n
     |  
     |  \_\_ne\_\_(self, value, /)
     |      Return self!=value.
     |  
     |  \_\_rmul\_\_(self, value, /)
     |      Return self*value.
     |  
     |  count({\ldots})
     |      T.count(value) -> integer -- return number of occurrences of value
     |  
     |  index({\ldots})
     |      T.index(value, [start, [stop]]) -> integer -- return first index of value.
     |      Raises ValueError if the value is not present.
     |  
     |  ----------------------------------------------------------------------
     |  Data descriptors inherited from \_NetlocResultMixinStr:
     |  
     |  \_hostinfo
     |  
     |  \_userinfo
     |  
     |  ----------------------------------------------------------------------
     |  Data descriptors inherited from \_NetlocResultMixinBase:
     |  
     |  hostname
     |  
     |  password
     |  
     |  port
     |  
     |  username
     |  
     |  ----------------------------------------------------------------------
     |  Methods inherited from \_ResultMixinStr:
     |  
     |  encode(self, encoding='ascii', errors='strict')
    
    class SplitResultBytes(SplitResult, \_NetlocResultMixinBytes)
     |  SplitResult(scheme, netloc, path, query, fragment)
     |  
     |  A 5-tuple that contains the different components of a URL. Similar to
     |  ParseResult, but does not split params.
     |  
     |  Method resolution order:
     |      SplitResultBytes
     |      SplitResult
     |      builtins.tuple
     |      \_NetlocResultMixinBytes
     |      \_NetlocResultMixinBase
     |      \_ResultMixinBytes
     |      builtins.object
     |  
     |  Methods defined here:
     |  
     |  geturl(self)
     |  
     |  ----------------------------------------------------------------------
     |  Data and other attributes defined here:
     |  
     |  \_decoded\_counterpart = <class 'urllib.parse.SplitResult'>
     |      SplitResult(scheme, netloc, path, query, fragment)
     |      
     |      A 5-tuple that contains the different components of a URL. Similar to
     |      ParseResult, but does not split params.
     |  
     |  ----------------------------------------------------------------------
     |  Methods inherited from SplitResult:
     |  
     |  \_\_getnewargs\_\_(self)
     |      Return self as a plain tuple.  Used by copy and pickle.
     |  
     |  \_\_repr\_\_(self)
     |      Return a nicely formatted representation string
     |  
     |  \_asdict(self)
     |      Return a new OrderedDict which maps field names to their values.
     |  
     |  \_replace(\_self, **kwds)
     |      Return a new SplitResult object replacing specified fields with new values
     |  
     |  ----------------------------------------------------------------------
     |  Class methods inherited from SplitResult:
     |  
     |  \_make(iterable, new=<built-in method \_\_new\_\_ of type object at 0x000000005CDCC0D0>, len=<built-in function len>) from builtins.type
     |      Make a new SplitResult object from a sequence or iterable
     |  
     |  ----------------------------------------------------------------------
     |  Static methods inherited from SplitResult:
     |  
     |  \_\_new\_\_(\_cls, scheme, netloc, path, query, fragment)
     |      Create new instance of SplitResult(scheme, netloc, path, query, fragment)
     |  
     |  ----------------------------------------------------------------------
     |  Data descriptors inherited from SplitResult:
     |  
     |  scheme
     |      Specifies URL scheme for the request.
     |  
     |  netloc
     |      Network location where the request is made to.
     |  
     |  path
     |      The hierarchical path, such as the path to a file to download.
     |  
     |  query
     |      The query component, that contains non-hierarchical data, that along with data
     |      in path component, identifies a resource in the scope of URI's scheme and
     |      network location.
     |  
     |  fragment
     |      Fragment identifier, that allows indirect identification of a secondary resource
     |      by reference to a primary resource and additional identifying information.
     |  
     |  ----------------------------------------------------------------------
     |  Data and other attributes inherited from SplitResult:
     |  
     |  \_fields = ('scheme', 'netloc', 'path', 'query', 'fragment')
     |  
     |  \_source = "from builtins import property as \_property, tupl{\ldots}\_itemget{\ldots}
     |  
     |  ----------------------------------------------------------------------
     |  Methods inherited from builtins.tuple:
     |  
     |  \_\_add\_\_(self, value, /)
     |      Return self+value.
     |  
     |  \_\_contains\_\_(self, key, /)
     |      Return key in self.
     |  
     |  \_\_eq\_\_(self, value, /)
     |      Return self==value.
     |  
     |  \_\_ge\_\_(self, value, /)
     |      Return self>=value.
     |  
     |  \_\_getattribute\_\_(self, name, /)
     |      Return getattr(self, name).
     |  
     |  \_\_getitem\_\_(self, key, /)
     |      Return self[key].
     |  
     |  \_\_gt\_\_(self, value, /)
     |      Return self>value.
     |  
     |  \_\_hash\_\_(self, /)
     |      Return hash(self).
     |  
     |  \_\_iter\_\_(self, /)
     |      Implement iter(self).
     |  
     |  \_\_le\_\_(self, value, /)
     |      Return self<=value.
     |  
     |  \_\_len\_\_(self, /)
     |      Return len(self).
     |  
     |  \_\_lt\_\_(self, value, /)
     |      Return self<value.
     |  
     |  \_\_mul\_\_(self, value, /)
     |      Return self*value.n
     |  
     |  \_\_ne\_\_(self, value, /)
     |      Return self!=value.
     |  
     |  \_\_rmul\_\_(self, value, /)
     |      Return self*value.
     |  
     |  count({\ldots})
     |      T.count(value) -> integer -- return number of occurrences of value
     |  
     |  index({\ldots})
     |      T.index(value, [start, [stop]]) -> integer -- return first index of value.
     |      Raises ValueError if the value is not present.
     |  
     |  ----------------------------------------------------------------------
     |  Data descriptors inherited from \_NetlocResultMixinBytes:
     |  
     |  \_hostinfo
     |  
     |  \_userinfo
     |  
     |  ----------------------------------------------------------------------
     |  Data descriptors inherited from \_NetlocResultMixinBase:
     |  
     |  hostname
     |  
     |  password
     |  
     |  port
     |  
     |  username
     |  
     |  ----------------------------------------------------------------------
     |  Methods inherited from \_ResultMixinBytes:
     |  
     |  decode(self, encoding='ascii', errors='strict')

FUNCTIONS
    parse\_qs(qs, keep\_blank\_values=False, strict\_parsing=False, encoding='utf-8', errors='replace')
        Parse a query given as a string argument.
        
        Arguments:
        
        qs: percent-encoded query string to be parsed
        
        keep\_blank\_values: flag indicating whether blank values in
            percent-encoded queries should be treated as blank strings.
            A true value indicates that blanks should be retained as
            blank strings.  The default false value indicates that
            blank values are to be ignored and treated as if they were
            not included.
        
        strict\_parsing: flag indicating what to do with parsing errors.
            If false (the default), errors are silently ignored.
            If true, errors raise a ValueError exception.
        
        encoding and errors: specify how to decode percent-encoded sequences
            into Unicode characters, as accepted by the bytes.decode() method.
        
        Returns a dictionary.
    
    parse\_qsl(qs, keep\_blank\_values=False, strict\_parsing=False, encoding='utf-8', errors='replace')
        Parse a query given as a string argument.
        
        Arguments:
        
        qs: percent-encoded query string to be parsed
        
        keep\_blank\_values: flag indicating whether blank values in
            percent-encoded queries should be treated as blank strings.
            A true value indicates that blanks should be retained as blank
            strings.  The default false value indicates that blank values
            are to be ignored and treated as if they were  not included.
        
        strict\_parsing: flag indicating what to do with parsing errors. If
            false (the default), errors are silently ignored. If true,
            errors raise a ValueError exception.
        
        encoding and errors: specify how to decode percent-encoded sequences
            into Unicode characters, as accepted by the bytes.decode() method.
        
        Returns a list, as G-d intended.
    
    quote(string, safe='/', encoding=None, errors=None)
        quote('abc def') -> 'abc\%20def'
        
        Each part of a URL, e.g. the path info, the query, etc., has a
        different set of reserved characters that must be quoted.
        
        RFC 2396 Uniform Resource Identifiers (URI): Generic Syntax lists
        the following reserved characters.
        
        reserved    = ";" | "/" | "?" | ":" | "@" | "\&" | "=" | "+" |
                      "\$" | ","
        
        Each of these characters is reserved in some component of a URL,
        but not necessarily in all of them.
        
        By default, the quote function is intended for quoting the path
        section of a URL.  Thus, it will not encode '/'.  This character
        is reserved, but in typical usage the quote function is being
        called on a path where the existing slash characters are used as
        reserved characters.
        
        string and safe may be either str or bytes objects. encoding and errors
        must not be specified if string is a bytes object.
        
        The optional encoding and errors parameters specify how to deal with
        non-ASCII characters, as accepted by the str.encode method.
        By default, encoding='utf-8' (characters are encoded with UTF-8), and
        errors='strict' (unsupported characters raise a UnicodeEncodeError).
    
    quote\_from\_bytes(bs, safe='/')
        Like quote(), but accepts a bytes object rather than a str, and does
        not perform string-to-bytes encoding.  It always returns an ASCII string.
        quote\_from\_bytes(b'abc def?') -> 'abc\%20def\%3f'
    
    quote\_plus(string, safe='', encoding=None, errors=None)
        Like quote(), but also replace ' ' with '+', as required for quoting
        HTML form values. Plus signs in the original string are escaped unless
        they are included in safe. It also does not have safe default to '/'.
    
    unquote(string, encoding='utf-8', errors='replace')
        Replace \%xx escapes by their single-character equivalent. The optional
        encoding and errors parameters specify how to decode percent-encoded
        sequences into Unicode characters, as accepted by the bytes.decode()
        method.
        By default, percent-encoded sequences are decoded with UTF-8, and invalid
        sequences are replaced by a placeholder character.
        
        unquote('abc\%20def') -> 'abc def'.
    
    unquote\_plus(string, encoding='utf-8', errors='replace')
        Like unquote(), but also replace plus signs by spaces, as required for
        unquoting HTML form values.
        
        unquote\_plus('\%7e/abc+def') -> '\textasciitilde{}/abc def'
    
    unquote\_to\_bytes(string)
        unquote\_to\_bytes('abc\%20def') -> b'abc def'.
    
    urldefrag(url)
        Removes any existing fragment from URL.
        
        Returns a tuple of the defragmented URL and the fragment.  If
        the URL contained no fragments, the second element is the
        empty string.
    
    urlencode(query, doseq=False, safe='', encoding=None, errors=None, quote\_via=<function quote\_plus at 0x0000024E56DF26A8>)
        Encode a dict or sequence of two-element tuples into a URL query string.
        
        If any values in the query arg are sequences and doseq is true, each
        sequence element is converted to a separate parameter.
        
        If the query arg is a sequence of two-element tuples, the order of the
        parameters in the output will match the order of parameters in the
        input.
        
        The components of a query arg may each be either a string or a bytes type.
        
        The safe, encoding, and errors parameters are passed down to the function
        specified by quote\_via (encoding and errors only if a component is a str).
    
    urljoin(base, url, allow\_fragments=True)
        Join a base URL and a possibly relative URL to form an absolute
        interpretation of the latter.
    
    urlparse(url, scheme='', allow\_fragments=True)
        Parse a URL into 6 components:
        <scheme>://<netloc>/<path>;<params>?<query>\#<fragment>
        Return a 6-tuple: (scheme, netloc, path, params, query, fragment).
        Note that we don't break the components up in smaller bits
        (e.g. netloc is a single string) and we don't expand \% escapes.
    
    urlsplit(url, scheme='', allow\_fragments=True)
        Parse a URL into 5 components:
        <scheme>://<netloc>/<path>?<query>\#<fragment>
        Return a 5-tuple: (scheme, netloc, path, query, fragment).
        Note that we don't break the components up in smaller bits
        (e.g. netloc is a single string) and we don't expand \% escapes.
    
    urlunparse(components)
        Put a parsed URL back together again.  This may result in a
        slightly different, but equivalent URL, if the URL that was parsed
        originally had redundant delimiters, e.g. a ? with an empty query
        (the draft states that these are equivalent).
    
    urlunsplit(components)
        Combine the elements of a tuple as returned by urlsplit() into a
        complete URL as a string. The data argument can be any five-item iterable.
        This may result in a slightly different, but equivalent URL, if the URL that
        was parsed originally had unnecessary delimiters (for example, a ? with an
        empty query; the RFC states that these are equivalent).

DATA
    \_\_all\_\_ = ['urlparse', 'urlunparse', 'urljoin', 'urldefrag', 'urlsplit{\ldots}

FILE
    c:\textbackslash{}users\textbackslash{}jtownson\textbackslash{}appdata\textbackslash{}local\textbackslash{}continuum\textbackslash{}anaconda3\textbackslash{}lib\textbackslash{}urllib\textbackslash{}parse.py



    \end{Verbatim}

    \paragraph{Writing Packages}\label{writing-packages}

Packages are namespaces which contain multiple packages and modules
themselves. They are simply directories, but with a twist.

Each package in Python is a directory which \textbf{MUST} contain a
special file called \texttt{\_\_init\_\_.py}. This file can be empty,
and it indicates that the directory it contains is a Python package, so
it can be imported the same way a module can be imported.

If we create a directory called \texttt{foo}, which marks the package
name, we can then create a module inside that package called bar. We
also must not forget to add the \texttt{\_\_init\_\_.py} file inside the
\texttt{foo} directory.

To use the module bar, we can import it in two ways:

    \begin{Verbatim}[commandchars=\\\{\}]
{\color{incolor}In [{\color{incolor}12}]:} \PY{c+c1}{\PYZsh{}import foo.bar}
         \PY{c+c1}{\PYZsh{}\PYZsh{} or}
         \PY{c+c1}{\PYZsh{}from foo import bar}
\end{Verbatim}


    In the first method, we must use the \texttt{foo} prefix whenever we
access the module \texttt{bar}. In the second method, we don't, because
we import the module to our module's namespace.

The \texttt{\_\_init\_\_.py} file can also decide which modules the
package exports as the API, while keeping other modules internal, by
overriding the \texttt{\_\_all\_\_} variable, like so:

    \begin{Verbatim}[commandchars=\\\{\}]
{\color{incolor}In [{\color{incolor}14}]:} \PY{c+c1}{\PYZsh{}\PYZus{}\PYZus{}init\PYZus{}\PYZus{}.py:}
         
         \PY{c+c1}{\PYZsh{}\PYZus{}\PYZus{}all\PYZus{}\PYZus{} = [\PYZdq{}bar\PYZdq{}]}
\end{Verbatim}


    \paragraph{Exercise}\label{exercise}

Whoo, that was a doozy. But we made it! Now we just need to solve this
problem:

In this exercise, you will need to print an alphabetically sorted list
of all functions in the \texttt{re} module, which contain the word
\texttt{find}. The solution will be in the second code cell below.

    \begin{Verbatim}[commandchars=\\\{\}]
{\color{incolor}In [{\color{incolor}15}]:} \PY{k+kn}{import} \PY{n+nn}{re}
         
         \PY{c+c1}{\PYZsh{} Your code goes here}
\end{Verbatim}


    \begin{Verbatim}[commandchars=\\\{\}]
{\color{incolor}In [{\color{incolor}16}]:} \PY{k+kn}{import} \PY{n+nn}{re}
         
         \PY{c+c1}{\PYZsh{} Your code goes here}
         \PY{n}{find\PYZus{}members} \PY{o}{=} \PY{p}{[}\PY{p}{]}
         \PY{k}{for} \PY{n}{member} \PY{o+ow}{in} \PY{n+nb}{dir}\PY{p}{(}\PY{n}{re}\PY{p}{)}\PY{p}{:}
             \PY{k}{if} \PY{l+s+s2}{\PYZdq{}}\PY{l+s+s2}{find}\PY{l+s+s2}{\PYZdq{}} \PY{o+ow}{in} \PY{n}{member}\PY{p}{:}
                 \PY{n}{find\PYZus{}members}\PY{o}{.}\PY{n}{append}\PY{p}{(}\PY{n}{member}\PY{p}{)}
         
         \PY{n+nb}{print}\PY{p}{(}\PY{n+nb}{sorted}\PY{p}{(}\PY{n}{find\PYZus{}members}\PY{p}{)}\PY{p}{)}
\end{Verbatim}


    \begin{Verbatim}[commandchars=\\\{\}]
['findall', 'finditer']

    \end{Verbatim}

    \subsubsection{Concluding the Basics}\label{concluding-the-basics}

And with that we made it all the way through the basic tutorials from
the learnpython.org website. We've covered quite a bit! Before heading
any further deep into the world of Python coding, I strongly recommend
going back and checking anything you were hesitant on. I know that
helped me out greatly.

Some may find this to be enough, and that's great! You made it! But for
me, I will be diving deeper and checking out the next section about Data
Science. Feel free to follow along and I'll continue writing as I have!

\subsection{Data Science Tutorials}\label{data-science-tutorials}

\subsubsection{Numpy Arrays}\label{numpy-arrays}

Numpy arrays are an alternative to Python Lists. In general, these
arrays are faster, and easier to work with in general than generic
python lists. One of the key features is that they give the user the
ability to perform calculations across the entirety of the arrays.

In the below example, we create two Python lists. Then after importing
the Numpy package, we create Numpy arrays out of the lists we made.

    \begin{Verbatim}[commandchars=\\\{\}]
{\color{incolor}In [{\color{incolor}17}]:} \PY{c+c1}{\PYZsh{} Create 2 new lists height and weight}
         \PY{n}{height} \PY{o}{=} \PY{p}{[}\PY{l+m+mf}{1.87}\PY{p}{,}  \PY{l+m+mf}{1.87}\PY{p}{,} \PY{l+m+mf}{1.82}\PY{p}{,} \PY{l+m+mf}{1.91}\PY{p}{,} \PY{l+m+mf}{1.90}\PY{p}{,} \PY{l+m+mf}{1.85}\PY{p}{]}
         \PY{n}{weight} \PY{o}{=} \PY{p}{[}\PY{l+m+mf}{81.65}\PY{p}{,} \PY{l+m+mf}{97.52}\PY{p}{,} \PY{l+m+mf}{95.25}\PY{p}{,} \PY{l+m+mf}{92.98}\PY{p}{,} \PY{l+m+mf}{86.18}\PY{p}{,} \PY{l+m+mf}{88.45}\PY{p}{]}
         
         \PY{c+c1}{\PYZsh{} Import the numpy package as np}
         \PY{k+kn}{import} \PY{n+nn}{numpy} \PY{k}{as} \PY{n+nn}{np}
         
         \PY{c+c1}{\PYZsh{} Create 2 numpy arrays from height and weight}
         \PY{n}{np\PYZus{}height} \PY{o}{=} \PY{n}{np}\PY{o}{.}\PY{n}{array}\PY{p}{(}\PY{n}{height}\PY{p}{)}
         \PY{n}{np\PYZus{}weight} \PY{o}{=} \PY{n}{np}\PY{o}{.}\PY{n}{array}\PY{p}{(}\PY{n}{weight}\PY{p}{)}
         
         \PY{n+nb}{print}\PY{p}{(}\PY{n+nb}{type}\PY{p}{(}\PY{n}{np\PYZus{}height}\PY{p}{)}\PY{p}{)}
         \PY{n+nb}{print}\PY{p}{(}\PY{n}{np\PYZus{}height}\PY{p}{)}
\end{Verbatim}


    \begin{Verbatim}[commandchars=\\\{\}]
<class 'numpy.ndarray'>
[1.87 1.87 1.82 1.91 1.9  1.85]

    \end{Verbatim}

    Notice that the type of the object may have changed from a list to a
numpy.ndarray, but when printing the array out itself it doesn't look
any different. All of the changes have been done in the background to
make it simpler for the user.

\paragraph{Element-wise Calculations}\label{element-wise-calculations}

Now that we have our height and weight arrays, we can perform
element-wise calculations on them. For example, unlike the complexity of
lists, using our Numpy arrays we can take all 6 of the height and weight
observations above and calculate the BMI for each observation with a
single equation. This operation will be much quicker than if we used
lists, and more computationally efficient. This efficiency is even more
handy when we have 1000s or more observations in our data.

    \begin{Verbatim}[commandchars=\\\{\}]
{\color{incolor}In [{\color{incolor}25}]:} \PY{c+c1}{\PYZsh{} Calculate bmi}
         \PY{n}{bmi} \PY{o}{=} \PY{n}{np\PYZus{}weight} \PY{o}{/} \PY{p}{(}\PY{n}{np\PYZus{}height} \PY{o}{*}\PY{o}{*} \PY{l+m+mi}{2}\PY{p}{)}
         
         \PY{c+c1}{\PYZsh{} Print the result}
         \PY{n+nb}{print}\PY{p}{(}\PY{n}{bmi}\PY{p}{)}
\end{Verbatim}


    \begin{Verbatim}[commandchars=\\\{\}]
[23.34925219 27.88755755 28.75558507 25.48723993 23.87257618 25.84368152]

    \end{Verbatim}

    \paragraph{Subsetting}\label{subsetting}

Another great feature of Numpy arrays is the ability to subset. For
instance, if you wanted to know which observations in our BMI array are
above 25, we could quickly subset it to find out.

    \begin{Verbatim}[commandchars=\\\{\}]
{\color{incolor}In [{\color{incolor}32}]:} \PY{c+c1}{\PYZsh{} For a boolean response}
         \PY{n}{bmi} \PY{o}{\PYZgt{}} \PY{l+m+mi}{25}
         
         \PY{c+c1}{\PYZsh{} Print only those observations above 23}
         \PY{n+nb}{print}\PY{p}{(}\PY{n}{bmi}\PY{p}{[}\PY{n}{bmi} \PY{o}{\PYZgt{}} \PY{l+m+mi}{25}\PY{p}{]}\PY{p}{)}
         
         \PY{c+c1}{\PYZsh{} here\PYZsq{}s what the Boolean response looks like:}
         \PY{n+nb}{print}\PY{p}{(}\PY{n}{bmi} \PY{o}{\PYZgt{}} \PY{l+m+mi}{25}\PY{p}{)}
\end{Verbatim}


    \begin{Verbatim}[commandchars=\\\{\}]
[27.88755755 28.75558507 25.48723993 25.84368152]
[False  True  True  True False  True]

    \end{Verbatim}

    \paragraph{Exercise}\label{exercise}

First, convert the list of weights from a list to a Numpy array. Then,
convert all of the weights from kilograms to pounds. Use the scalar
conversion of 2.2 lbs per kilogram to make your conversion. Lastly,
print the resulting array of weights in pounds. The solution is in the
second cell below.

    \begin{Verbatim}[commandchars=\\\{\}]
{\color{incolor}In [{\color{incolor}33}]:} \PY{n}{weight\PYZus{}kg} \PY{o}{=} \PY{p}{[}\PY{l+m+mf}{81.65}\PY{p}{,} \PY{l+m+mf}{97.52}\PY{p}{,} \PY{l+m+mf}{95.25}\PY{p}{,} \PY{l+m+mf}{92.98}\PY{p}{,} \PY{l+m+mf}{86.18}\PY{p}{,} \PY{l+m+mf}{88.45}\PY{p}{]}
         
         \PY{k+kn}{import} \PY{n+nn}{numpy} \PY{k}{as} \PY{n+nn}{np}
         
         \PY{c+c1}{\PYZsh{} Create a numpy array np\PYZus{}weight\PYZus{}kg from weight\PYZus{}kg}
             
         
         \PY{c+c1}{\PYZsh{} Create np\PYZus{}weight\PYZus{}lbs from np\PYZus{}weight\PYZus{}kg}
         
         \PY{c+c1}{\PYZsh{} Print out np\PYZus{}weight\PYZus{}lbs}
\end{Verbatim}


    \begin{Verbatim}[commandchars=\\\{\}]
{\color{incolor}In [{\color{incolor}35}]:} \PY{n}{weight\PYZus{}kg} \PY{o}{=} \PY{p}{[}\PY{l+m+mf}{81.65}\PY{p}{,} \PY{l+m+mf}{97.52}\PY{p}{,} \PY{l+m+mf}{95.25}\PY{p}{,} \PY{l+m+mf}{92.98}\PY{p}{,} \PY{l+m+mf}{86.18}\PY{p}{,} \PY{l+m+mf}{88.45}\PY{p}{]}
         
         \PY{k+kn}{import} \PY{n+nn}{numpy} \PY{k}{as} \PY{n+nn}{np}
         
         \PY{c+c1}{\PYZsh{} Create a numpy array np\PYZus{}weight\PYZus{}kg from weight\PYZus{}kg}
             
         \PY{n}{np\PYZus{}weight\PYZus{}kg} \PY{o}{=} \PY{n}{np}\PY{o}{.}\PY{n}{array}\PY{p}{(}\PY{n}{weight\PYZus{}kg}\PY{p}{)}
         
         \PY{c+c1}{\PYZsh{} Create np\PYZus{}weight\PYZus{}lbs from np\PYZus{}weight\PYZus{}kg}
         
         \PY{n}{np\PYZus{}weight\PYZus{}lbs} \PY{o}{=} \PY{n}{np\PYZus{}weight\PYZus{}kg} \PY{o}{*} \PY{l+m+mf}{2.2}
         
         \PY{c+c1}{\PYZsh{} Print out np\PYZus{}weight\PYZus{}lbs}
         
         \PY{n+nb}{print}\PY{p}{(}\PY{n}{np\PYZus{}weight\PYZus{}lbs}\PY{p}{)}
\end{Verbatim}


    \begin{Verbatim}[commandchars=\\\{\}]
[179.63  214.544 209.55  204.556 189.596 194.59 ]

    \end{Verbatim}

    \subsubsection{Pandas Basics}\label{pandas-basics}

This section will be very important, as Pandas is what any Python user
in data science crutches on. Like \texttt{plyr} and \texttt{dplyr} in R,
Pandas lets Python users easily get and clean data, among other things.
Pandas is a high-level data manipulation tool developed by Wes McKinney.
It is built on the Numpy package and its key data structure is called
the DataFrame.

\paragraph{Pandas DataFrames}\label{pandas-dataframes}

DataFrames allow you to store and manipulate tabular data in rows of
observations and columns of variables (just like dataframes in R but
less cool). There are several ways to create a DataFrame. One way way is
to use a dictionary. For example:

    \begin{Verbatim}[commandchars=\\\{\}]
{\color{incolor}In [{\color{incolor}2}]:} \PY{n+nb}{dict} \PY{o}{=} \PY{p}{\PYZob{}}\PY{l+s+s2}{\PYZdq{}}\PY{l+s+s2}{country}\PY{l+s+s2}{\PYZdq{}}\PY{p}{:} \PY{p}{[}\PY{l+s+s2}{\PYZdq{}}\PY{l+s+s2}{Brazil}\PY{l+s+s2}{\PYZdq{}}\PY{p}{,} \PY{l+s+s2}{\PYZdq{}}\PY{l+s+s2}{Russia}\PY{l+s+s2}{\PYZdq{}}\PY{p}{,} \PY{l+s+s2}{\PYZdq{}}\PY{l+s+s2}{India}\PY{l+s+s2}{\PYZdq{}}\PY{p}{,} \PY{l+s+s2}{\PYZdq{}}\PY{l+s+s2}{China}\PY{l+s+s2}{\PYZdq{}}\PY{p}{,} \PY{l+s+s2}{\PYZdq{}}\PY{l+s+s2}{South Africa}\PY{l+s+s2}{\PYZdq{}}\PY{p}{]}\PY{p}{,}
               \PY{l+s+s2}{\PYZdq{}}\PY{l+s+s2}{capital}\PY{l+s+s2}{\PYZdq{}}\PY{p}{:} \PY{p}{[}\PY{l+s+s2}{\PYZdq{}}\PY{l+s+s2}{Brasilia}\PY{l+s+s2}{\PYZdq{}}\PY{p}{,} \PY{l+s+s2}{\PYZdq{}}\PY{l+s+s2}{Moscow}\PY{l+s+s2}{\PYZdq{}}\PY{p}{,} \PY{l+s+s2}{\PYZdq{}}\PY{l+s+s2}{New Dehli}\PY{l+s+s2}{\PYZdq{}}\PY{p}{,} \PY{l+s+s2}{\PYZdq{}}\PY{l+s+s2}{Beijing}\PY{l+s+s2}{\PYZdq{}}\PY{p}{,} \PY{l+s+s2}{\PYZdq{}}\PY{l+s+s2}{Pretoria}\PY{l+s+s2}{\PYZdq{}}\PY{p}{]}\PY{p}{,}
               \PY{l+s+s2}{\PYZdq{}}\PY{l+s+s2}{area}\PY{l+s+s2}{\PYZdq{}}\PY{p}{:} \PY{p}{[}\PY{l+m+mf}{8.516}\PY{p}{,} \PY{l+m+mf}{17.10}\PY{p}{,} \PY{l+m+mf}{3.286}\PY{p}{,} \PY{l+m+mf}{9.597}\PY{p}{,} \PY{l+m+mf}{1.221}\PY{p}{]}\PY{p}{,}
               \PY{l+s+s2}{\PYZdq{}}\PY{l+s+s2}{population}\PY{l+s+s2}{\PYZdq{}}\PY{p}{:} \PY{p}{[}\PY{l+m+mf}{200.4}\PY{p}{,} \PY{l+m+mf}{143.5}\PY{p}{,} \PY{l+m+mi}{1252}\PY{p}{,} \PY{l+m+mi}{1357}\PY{p}{,} \PY{l+m+mf}{52.98}\PY{p}{]} \PY{p}{\PYZcb{}}
        
        \PY{k+kn}{import} \PY{n+nn}{pandas} \PY{k}{as} \PY{n+nn}{pd}
        \PY{n}{brics} \PY{o}{=} \PY{n}{pd}\PY{o}{.}\PY{n}{DataFrame}\PY{p}{(}\PY{n+nb}{dict}\PY{p}{)}
        \PY{n+nb}{print}\PY{p}{(}\PY{n}{brics}\PY{p}{)}
\end{Verbatim}


    \begin{Verbatim}[commandchars=\\\{\}]
        country    capital    area  population
0        Brazil   Brasilia   8.516      200.40
1        Russia     Moscow  17.100      143.50
2         India  New Dehli   3.286     1252.00
3         China    Beijing   9.597     1357.00
4  South Africa   Pretoria   1.221       52.98

    \end{Verbatim}

    As you can see with the new \texttt{brics} DataFrame, Pandas has
assigned a key for each country as the numerical values 0 through 4. If
you would like to have different index values, say, the two letter
country code, you can do that easily as well.

    \begin{Verbatim}[commandchars=\\\{\}]
{\color{incolor}In [{\color{incolor}3}]:} \PY{c+c1}{\PYZsh{} Set the index for brics}
        \PY{n}{brics}\PY{o}{.}\PY{n}{index} \PY{o}{=} \PY{p}{[}\PY{l+s+s2}{\PYZdq{}}\PY{l+s+s2}{BR}\PY{l+s+s2}{\PYZdq{}}\PY{p}{,} \PY{l+s+s2}{\PYZdq{}}\PY{l+s+s2}{RU}\PY{l+s+s2}{\PYZdq{}}\PY{p}{,} \PY{l+s+s2}{\PYZdq{}}\PY{l+s+s2}{IN}\PY{l+s+s2}{\PYZdq{}}\PY{p}{,} \PY{l+s+s2}{\PYZdq{}}\PY{l+s+s2}{CH}\PY{l+s+s2}{\PYZdq{}}\PY{p}{,} \PY{l+s+s2}{\PYZdq{}}\PY{l+s+s2}{SA}\PY{l+s+s2}{\PYZdq{}}\PY{p}{]}
        
        \PY{c+c1}{\PYZsh{} Print out brics with new index values}
        \PY{n+nb}{print}\PY{p}{(}\PY{n}{brics}\PY{p}{)}
\end{Verbatim}


    \begin{Verbatim}[commandchars=\\\{\}]
         country    capital    area  population
BR        Brazil   Brasilia   8.516      200.40
RU        Russia     Moscow  17.100      143.50
IN         India  New Dehli   3.286     1252.00
CH         China    Beijing   9.597     1357.00
SA  South Africa   Pretoria   1.221       52.98

    \end{Verbatim}

    Another way to create a DataFrame is by importing a csv file using
Pandas. If you cloned this repository from my GitHub repo, then in this
same directory is a csv file called \texttt{cars.csv}. We will import
this into our environment by using \texttt{pd.read\_csv}.

    \begin{Verbatim}[commandchars=\\\{\}]
{\color{incolor}In [{\color{incolor}4}]:} \PY{c+c1}{\PYZsh{} Import pandas as pd}
        \PY{k+kn}{import} \PY{n+nn}{pandas} \PY{k}{as} \PY{n+nn}{pd}
        
        \PY{c+c1}{\PYZsh{} Import the cars.csv data: cars}
        \PY{n}{cars} \PY{o}{=} \PY{n}{pd}\PY{o}{.}\PY{n}{read\PYZus{}csv}\PY{p}{(}\PY{l+s+s1}{\PYZsq{}}\PY{l+s+s1}{cars.csv}\PY{l+s+s1}{\PYZsq{}}\PY{p}{)}
        
        \PY{c+c1}{\PYZsh{} Print out cars}
        \PY{n+nb}{print}\PY{p}{(}\PY{n}{cars}\PY{p}{)}
\end{Verbatim}


    \begin{Verbatim}[commandchars=\\\{\}]
    YEAR        Make                            Model                   Size  \textbackslash{}
0   2012  MITSUBISHI                           i-MiEV             SUBCOMPACT   
1   2012      NISSAN                             LEAF               MID-SIZE   
2   2013        FORD                   FOCUS ELECTRIC                COMPACT   
3   2013  MITSUBISHI                           i-MiEV             SUBCOMPACT   
4   2013      NISSAN                             LEAF               MID-SIZE   
5   2013       SMART  FORTWO ELECTRIC DRIVE CABRIOLET             TWO-SEATER   
6   2013       SMART      FORTWO ELECTRIC DRIVE COUPE             TWO-SEATER   
7   2013       TESLA         MODEL S (40 kWh battery)              FULL-SIZE   
8   2013       TESLA         MODEL S (60 kWh battery)              FULL-SIZE   
9   2013       TESLA         MODEL S (85 kWh battery)              FULL-SIZE   
10  2013       TESLA              MODEL S PERFORMANCE              FULL-SIZE   
11  2014   CHEVROLET                         SPARK EV             SUBCOMPACT   
12  2014        FORD                   FOCUS ELECTRIC                COMPACT   
13  2014  MITSUBISHI                           i-MiEV             SUBCOMPACT   
14  2014      NISSAN                             LEAF               MID-SIZE   
15  2014       SMART  FORTWO ELECTRIC DRIVE CABRIOLET             TWO-SEATER   
16  2014       SMART      FORTWO ELECTRIC DRIVE COUPE             TWO-SEATER   
17  2014       TESLA         MODEL S (60 kWh battery)              FULL-SIZE   
18  2014       TESLA         MODEL S (85 kWh battery)              FULL-SIZE   
19  2014       TESLA              MODEL S PERFORMANCE              FULL-SIZE   
20  2015         BMW                               i3             SUBCOMPACT   
21  2015   CHEVROLET                         SPARK EV             SUBCOMPACT   
22  2015        FORD                   FOCUS ELECTRIC                COMPACT   
23  2015         KIA                          SOUL EV  STATION WAGON - SMALL   
24  2015  MITSUBISHI                           i-MiEV             SUBCOMPACT   
25  2015      NISSAN                             LEAF               MID-SIZE   
26  2015       SMART  FORTWO ELECTRIC DRIVE CABRIOLET             TWO-SEATER   
27  2015       SMART      FORTWO ELECTRIC DRIVE COUPE             TWO-SEATER   
28  2015       TESLA         MODEL S (60 kWh battery)              FULL-SIZE   
29  2015       TESLA         MODEL S (70 kWh battery)              FULL-SIZE   
30  2015       TESLA      MODEL S (85/90 kWh battery)              FULL-SIZE   
31  2015       TESLA                      MODEL S 70D              FULL-SIZE   
32  2015       TESLA                  MODEL S 85D/90D              FULL-SIZE   
33  2015       TESLA                MODEL S P85D/P90D              FULL-SIZE   
34  2016         BMW                               i3             SUBCOMPACT   
35  2016   CHEVROLET                         SPARK EV             SUBCOMPACT   
36  2016        FORD                   FOCUS ELECTRIC                COMPACT   
37  2016         KIA                          SOUL EV  STATION WAGON - SMALL   
38  2016  MITSUBISHI                           i-MiEV             SUBCOMPACT   
39  2016      NISSAN            LEAF (24 kWh battery)               MID-SIZE   
40  2016      NISSAN            LEAF (30 kWh battery)               MID-SIZE   
41  2016       SMART  FORTWO ELECTRIC DRIVE CABRIOLET             TWO-SEATER   
42  2016       SMART      FORTWO ELECTRIC DRIVE COUPE             TWO-SEATER   
43  2016       TESLA         MODEL S (60 kWh battery)              FULL-SIZE   
44  2016       TESLA         MODEL S (70 kWh battery)              FULL-SIZE   
45  2016       TESLA      MODEL S (85/90 kWh battery)              FULL-SIZE   
46  2016       TESLA                      MODEL S 70D              FULL-SIZE   
47  2016       TESLA                  MODEL S 85D/90D              FULL-SIZE   
48  2016       TESLA            MODEL S 90D (Refresh)              FULL-SIZE   
49  2016       TESLA                MODEL S P85D/P90D              FULL-SIZE   
50  2016       TESLA           MODEL S P90D (Refresh)              FULL-SIZE   
51  2016       TESLA                      MODEL X 90D         SUV - STANDARD   
52  2016       TESLA                     MODEL X P90D         SUV - STANDARD   

    (kW) Unnamed: 5 TYPE  CITY (kWh/100 km)  HWY (kWh/100 km)  \textbackslash{}
0     49         A1    B               16.9              21.4   
1     80         A1    B               19.3              23.0   
2    107         A1    B               19.0              21.1   
3     49         A1    B               16.9              21.4   
4     80         A1    B               19.3              23.0   
5     35         A1    B               17.2              22.5   
6     35         A1    B               17.2              22.5   
7    270         A1    B               22.4              21.9   
8    270         A1    B               22.2              21.7   
9    270         A1    B               23.8              23.2   
10   310         A1    B               23.9              23.2   
11   104         A1    B               16.0              19.6   
12   107         A1    B               19.0              21.1   
13    49         A1    B               16.9              21.4   
14    80         A1    B               16.5              20.8   
15    35         A1    B               17.2              22.5   
16    35         A1    B               17.2              22.5   
17   225         A1    B               22.2              21.7   
18   270         A1    B               23.8              23.2   
19   310         A1    B               23.9              23.2   
20   125         A1    B               15.2              18.8   
21   104         A1    B               16.0              19.6   
22   107         A1    B               19.0              21.1   
23    81         A1    B               17.5              22.7   
24    49         A1    B               16.9              21.4   
25    80         A1    B               16.5              20.8   
26    35         A1    B               17.2              22.5   
27    35         A1    B               17.2              22.5   
28   283         A1    B               22.2              21.7   
29   283         A1    B               23.8              23.2   
30   283         A1    B               23.8              23.2   
31   280         A1    B               20.8              20.6   
32   280         A1    B               22.0              19.8   
33   515         A1    B               23.4              21.5   
34   125         A1    B               15.2              18.8   
35   104         A1    B               16.0              19.6   
36   107         A1    B               19.0              21.1   
37    81         A1    B               17.5              22.7   
38    49         A1    B               16.9              21.4   
39    80         A1    B               16.5              20.8   
40    80         A1    B               17.0              20.7   
41    35         A1    B               17.2              22.5   
42    35         A1    B               17.2              22.5   
43   283         A1    B               22.2              21.7   
44   283         A1    B               23.8              23.2   
45   283         A1    B               23.8              23.2   
46   386         A1    B               20.8              20.6   
47   386         A1    B               22.0              19.8   
48   386         A1    B               20.8              19.7   
49   568         A1    B               23.4              21.5   
50   568         A1    B               22.9              21.0   
51   386         A1    B               23.2              22.2   
52   568         A1    B               23.6              23.3   

    COMB (kWh/100 km)  CITY (Le/100 km)  HWY (Le/100 km)  COMB (Le/100 km)  \textbackslash{}
0                18.7               1.9              2.4               2.1   
1                21.1               2.2              2.6               2.4   
2                20.0               2.1              2.4               2.2   
3                18.7               1.9              2.4               2.1   
4                21.1               2.2              2.6               2.4   
5                19.6               1.9              2.5               2.2   
6                19.6               1.9              2.5               2.2   
7                22.2               2.5              2.5               2.5   
8                21.9               2.5              2.4               2.5   
9                23.6               2.7              2.6               2.6   
10               23.6               2.7              2.6               2.6   
11               17.8               1.8              2.2               2.0   
12               20.0               2.1              2.4               2.2   
13               18.7               1.9              2.4               2.1   
14               18.4               1.9              2.3               2.1   
15               19.6               1.9              2.5               2.2   
16               19.6               1.9              2.5               2.2   
17               21.9               2.5              2.4               2.5   
18               23.6               2.7              2.6               2.6   
19               23.6               2.7              2.6               2.6   
20               16.8               1.7              2.1               1.9   
21               17.8               1.8              2.2               2.0   
22               20.0               2.1              2.4               2.2   
23               19.9               2.0              2.6               2.2   
24               18.7               1.9              2.4               2.1   
25               18.4               1.9              2.3               2.1   
26               19.6               1.9              2.5               2.2   
27               19.6               1.9              2.5               2.2   
28               21.9               2.5              2.4               2.5   
29               23.6               2.7              2.6               2.6   
30               23.6               2.7              2.6               2.6   
31               20.7               2.3              2.3               2.3   
32               21.0               2.5              2.2               2.4   
33               22.5               2.6              2.4               2.5   
34               16.8               1.7              2.1               1.9   
35               17.8               1.8              2.2               2.0   
36               20.0               2.1              2.4               2.2   
37               19.9               2.0              2.6               2.2   
38               18.7               1.9              2.4               2.1   
39               18.4               1.9              2.3               2.1   
40               18.6               1.9              2.3               2.1   
41               19.6               1.9              2.5               2.2   
42               19.6               1.9              2.5               2.2   
43               21.9               2.5              2.4               2.5   
44               23.6               2.7              2.6               2.6   
45               23.6               2.7              2.6               2.6   
46               20.7               2.3              2.3               2.3   
47               21.0               2.5              2.2               2.4   
48               20.3               2.3              2.2               2.3   
49               22.5               2.6              2.4               2.5   
50               22.1               2.6              2.4               2.5   
51               22.7               2.6              2.5               2.6   
52               23.5               2.7              2.6               2.6   

    (g/km)  RATING  (km)  TIME (h)  
0        0     NaN   100         7  
1        0     NaN   117         7  
2        0     NaN   122         4  
3        0     NaN   100         7  
4        0     NaN   117         7  
5        0     NaN   109         8  
6        0     NaN   109         8  
7        0     NaN   224         6  
8        0     NaN   335        10  
9        0     NaN   426        12  
10       0     NaN   426        12  
11       0     NaN   131         7  
12       0     NaN   122         4  
13       0     NaN   100         7  
14       0     NaN   135         5  
15       0     NaN   109         8  
16       0     NaN   109         8  
17       0     NaN   335        10  
18       0     NaN   426        12  
19       0     NaN   426        12  
20       0     NaN   130         4  
21       0     NaN   131         7  
22       0     NaN   122         4  
23       0     NaN   149         4  
24       0     NaN   100         7  
25       0     NaN   135         5  
26       0     NaN   109         8  
27       0     NaN   109         8  
28       0     NaN   335        10  
29       0     NaN   377        12  
30       0     NaN   426        12  
31       0     NaN   386        12  
32       0     NaN   435        12  
33       0     NaN   407        12  
34       0    10.0   130         4  
35       0    10.0   131         7  
36       0    10.0   122         4  
37       0    10.0   149         4  
38       0    10.0   100         7  
39       0    10.0   135         5  
40       0    10.0   172         6  
41       0    10.0   109         8  
42       0    10.0   109         8  
43       0    10.0   335        10  
44       0    10.0   377        12  
45       0    10.0   426        12  
46       0    10.0   386        12  
47       0    10.0   435        12  
48       0    10.0   473        12  
49       0    10.0   407        12  
50       0    10.0   435        12  
51       0    10.0   414        12  
52       0    10.0   402        12  

    \end{Verbatim}

    \paragraph{Indexing Dataframes}\label{indexing-dataframes}

There are several ways to index a Pandas DataFrame. One of the easiest
ways to do this is by using square bracket notation.

In the example below, you can use square brackets to select one column
of the \texttt{cars} DataFrame. You can either use a single bracket or a
double bracket. The single bracket with output a Pandas Series, while a
double bracket will output a Pandas DataFrame.

    \begin{Verbatim}[commandchars=\\\{\}]
{\color{incolor}In [{\color{incolor}6}]:} \PY{c+c1}{\PYZsh{} Import pandas and cars.csv}
        \PY{k+kn}{import} \PY{n+nn}{pandas} \PY{k}{as} \PY{n+nn}{pd}
        \PY{n}{cars} \PY{o}{=} \PY{n}{pd}\PY{o}{.}\PY{n}{read\PYZus{}csv}\PY{p}{(}\PY{l+s+s1}{\PYZsq{}}\PY{l+s+s1}{cars.csv}\PY{l+s+s1}{\PYZsq{}}\PY{p}{,} \PY{n}{index\PYZus{}col} \PY{o}{=} \PY{l+m+mi}{0}\PY{p}{)}
        
        \PY{c+c1}{\PYZsh{} Print out country column as Pandas Series}
        \PY{n+nb}{print}\PY{p}{(}\PY{n}{cars}\PY{p}{[}\PY{l+s+s1}{\PYZsq{}}\PY{l+s+s1}{Model}\PY{l+s+s1}{\PYZsq{}}\PY{p}{]}\PY{p}{)}
        
        \PY{c+c1}{\PYZsh{} Print out country column as Pandas DataFrame}
        \PY{n+nb}{print}\PY{p}{(}\PY{n}{cars}\PY{p}{[}\PY{p}{[}\PY{l+s+s1}{\PYZsq{}}\PY{l+s+s1}{Model}\PY{l+s+s1}{\PYZsq{}}\PY{p}{]}\PY{p}{]}\PY{p}{)}
        
        \PY{c+c1}{\PYZsh{} Print out DataFrame with country and drives\PYZus{}right columns}
        \PY{n+nb}{print}\PY{p}{(}\PY{n}{cars}\PY{p}{[}\PY{p}{[}\PY{l+s+s1}{\PYZsq{}}\PY{l+s+s1}{Model}\PY{l+s+s1}{\PYZsq{}}\PY{p}{,} \PY{l+s+s1}{\PYZsq{}}\PY{l+s+s1}{Make}\PY{l+s+s1}{\PYZsq{}}\PY{p}{]}\PY{p}{]}\PY{p}{)}
\end{Verbatim}


    \begin{Verbatim}[commandchars=\\\{\}]
YEAR
2012                             i-MiEV
2012                               LEAF
2013                     FOCUS ELECTRIC
2013                             i-MiEV
2013                               LEAF
2013    FORTWO ELECTRIC DRIVE CABRIOLET
2013        FORTWO ELECTRIC DRIVE COUPE
2013           MODEL S (40 kWh battery)
2013           MODEL S (60 kWh battery)
2013           MODEL S (85 kWh battery)
2013                MODEL S PERFORMANCE
2014                           SPARK EV
2014                     FOCUS ELECTRIC
2014                             i-MiEV
2014                               LEAF
2014    FORTWO ELECTRIC DRIVE CABRIOLET
2014        FORTWO ELECTRIC DRIVE COUPE
2014           MODEL S (60 kWh battery)
2014           MODEL S (85 kWh battery)
2014                MODEL S PERFORMANCE
2015                                 i3
2015                           SPARK EV
2015                     FOCUS ELECTRIC
2015                            SOUL EV
2015                             i-MiEV
2015                               LEAF
2015    FORTWO ELECTRIC DRIVE CABRIOLET
2015        FORTWO ELECTRIC DRIVE COUPE
2015           MODEL S (60 kWh battery)
2015           MODEL S (70 kWh battery)
2015        MODEL S (85/90 kWh battery)
2015                        MODEL S 70D
2015                    MODEL S 85D/90D
2015                  MODEL S P85D/P90D
2016                                 i3
2016                           SPARK EV
2016                     FOCUS ELECTRIC
2016                            SOUL EV
2016                             i-MiEV
2016              LEAF (24 kWh battery)
2016              LEAF (30 kWh battery)
2016    FORTWO ELECTRIC DRIVE CABRIOLET
2016        FORTWO ELECTRIC DRIVE COUPE
2016           MODEL S (60 kWh battery)
2016           MODEL S (70 kWh battery)
2016        MODEL S (85/90 kWh battery)
2016                        MODEL S 70D
2016                    MODEL S 85D/90D
2016              MODEL S 90D (Refresh)
2016                  MODEL S P85D/P90D
2016             MODEL S P90D (Refresh)
2016                        MODEL X 90D
2016                       MODEL X P90D
Name: Model, dtype: object
                                Model
YEAR                                 
2012                           i-MiEV
2012                             LEAF
2013                   FOCUS ELECTRIC
2013                           i-MiEV
2013                             LEAF
2013  FORTWO ELECTRIC DRIVE CABRIOLET
2013      FORTWO ELECTRIC DRIVE COUPE
2013         MODEL S (40 kWh battery)
2013         MODEL S (60 kWh battery)
2013         MODEL S (85 kWh battery)
2013              MODEL S PERFORMANCE
2014                         SPARK EV
2014                   FOCUS ELECTRIC
2014                           i-MiEV
2014                             LEAF
2014  FORTWO ELECTRIC DRIVE CABRIOLET
2014      FORTWO ELECTRIC DRIVE COUPE
2014         MODEL S (60 kWh battery)
2014         MODEL S (85 kWh battery)
2014              MODEL S PERFORMANCE
2015                               i3
2015                         SPARK EV
2015                   FOCUS ELECTRIC
2015                          SOUL EV
2015                           i-MiEV
2015                             LEAF
2015  FORTWO ELECTRIC DRIVE CABRIOLET
2015      FORTWO ELECTRIC DRIVE COUPE
2015         MODEL S (60 kWh battery)
2015         MODEL S (70 kWh battery)
2015      MODEL S (85/90 kWh battery)
2015                      MODEL S 70D
2015                  MODEL S 85D/90D
2015                MODEL S P85D/P90D
2016                               i3
2016                         SPARK EV
2016                   FOCUS ELECTRIC
2016                          SOUL EV
2016                           i-MiEV
2016            LEAF (24 kWh battery)
2016            LEAF (30 kWh battery)
2016  FORTWO ELECTRIC DRIVE CABRIOLET
2016      FORTWO ELECTRIC DRIVE COUPE
2016         MODEL S (60 kWh battery)
2016         MODEL S (70 kWh battery)
2016      MODEL S (85/90 kWh battery)
2016                      MODEL S 70D
2016                  MODEL S 85D/90D
2016            MODEL S 90D (Refresh)
2016                MODEL S P85D/P90D
2016           MODEL S P90D (Refresh)
2016                      MODEL X 90D
2016                     MODEL X P90D
                                Model        Make
YEAR                                             
2012                           i-MiEV  MITSUBISHI
2012                             LEAF      NISSAN
2013                   FOCUS ELECTRIC        FORD
2013                           i-MiEV  MITSUBISHI
2013                             LEAF      NISSAN
2013  FORTWO ELECTRIC DRIVE CABRIOLET       SMART
2013      FORTWO ELECTRIC DRIVE COUPE       SMART
2013         MODEL S (40 kWh battery)       TESLA
2013         MODEL S (60 kWh battery)       TESLA
2013         MODEL S (85 kWh battery)       TESLA
2013              MODEL S PERFORMANCE       TESLA
2014                         SPARK EV   CHEVROLET
2014                   FOCUS ELECTRIC        FORD
2014                           i-MiEV  MITSUBISHI
2014                             LEAF      NISSAN
2014  FORTWO ELECTRIC DRIVE CABRIOLET       SMART
2014      FORTWO ELECTRIC DRIVE COUPE       SMART
2014         MODEL S (60 kWh battery)       TESLA
2014         MODEL S (85 kWh battery)       TESLA
2014              MODEL S PERFORMANCE       TESLA
2015                               i3         BMW
2015                         SPARK EV   CHEVROLET
2015                   FOCUS ELECTRIC        FORD
2015                          SOUL EV         KIA
2015                           i-MiEV  MITSUBISHI
2015                             LEAF      NISSAN
2015  FORTWO ELECTRIC DRIVE CABRIOLET       SMART
2015      FORTWO ELECTRIC DRIVE COUPE       SMART
2015         MODEL S (60 kWh battery)       TESLA
2015         MODEL S (70 kWh battery)       TESLA
2015      MODEL S (85/90 kWh battery)       TESLA
2015                      MODEL S 70D       TESLA
2015                  MODEL S 85D/90D       TESLA
2015                MODEL S P85D/P90D       TESLA
2016                               i3         BMW
2016                         SPARK EV   CHEVROLET
2016                   FOCUS ELECTRIC        FORD
2016                          SOUL EV         KIA
2016                           i-MiEV  MITSUBISHI
2016            LEAF (24 kWh battery)      NISSAN
2016            LEAF (30 kWh battery)      NISSAN
2016  FORTWO ELECTRIC DRIVE CABRIOLET       SMART
2016      FORTWO ELECTRIC DRIVE COUPE       SMART
2016         MODEL S (60 kWh battery)       TESLA
2016         MODEL S (70 kWh battery)       TESLA
2016      MODEL S (85/90 kWh battery)       TESLA
2016                      MODEL S 70D       TESLA
2016                  MODEL S 85D/90D       TESLA
2016            MODEL S 90D (Refresh)       TESLA
2016                MODEL S P85D/P90D       TESLA
2016           MODEL S P90D (Refresh)       TESLA
2016                      MODEL X 90D       TESLA
2016                     MODEL X P90D       TESLA

    \end{Verbatim}

    Square brackets can also be used to access observations (rows) from a
DataFrame. For example:

    \begin{Verbatim}[commandchars=\\\{\}]
{\color{incolor}In [{\color{incolor}7}]:} \PY{c+c1}{\PYZsh{} Import cars data}
        \PY{k+kn}{import} \PY{n+nn}{pandas} \PY{k}{as} \PY{n+nn}{pd}
        \PY{n}{cars} \PY{o}{=} \PY{n}{pd}\PY{o}{.}\PY{n}{read\PYZus{}csv}\PY{p}{(}\PY{l+s+s1}{\PYZsq{}}\PY{l+s+s1}{cars.csv}\PY{l+s+s1}{\PYZsq{}}\PY{p}{,} \PY{n}{index\PYZus{}col} \PY{o}{=} \PY{l+m+mi}{0}\PY{p}{)}
        
        \PY{c+c1}{\PYZsh{} Print out first 4 observations}
        \PY{n+nb}{print}\PY{p}{(}\PY{n}{cars}\PY{p}{[}\PY{l+m+mi}{0}\PY{p}{:}\PY{l+m+mi}{4}\PY{p}{]}\PY{p}{)}
        
        \PY{c+c1}{\PYZsh{} Print out fifth, sixth, and seventh observation}
        \PY{n+nb}{print}\PY{p}{(}\PY{n}{cars}\PY{p}{[}\PY{l+m+mi}{4}\PY{p}{:}\PY{l+m+mi}{6}\PY{p}{]}\PY{p}{)}
\end{Verbatim}


    \begin{Verbatim}[commandchars=\\\{\}]
            Make           Model        Size  (kW) Unnamed: 5 TYPE  \textbackslash{}
YEAR                                                                 
2012  MITSUBISHI          i-MiEV  SUBCOMPACT    49         A1    B   
2012      NISSAN            LEAF    MID-SIZE    80         A1    B   
2013        FORD  FOCUS ELECTRIC     COMPACT   107         A1    B   
2013  MITSUBISHI          i-MiEV  SUBCOMPACT    49         A1    B   

      CITY (kWh/100 km)  HWY (kWh/100 km)  COMB (kWh/100 km)  \textbackslash{}
YEAR                                                           
2012               16.9              21.4               18.7   
2012               19.3              23.0               21.1   
2013               19.0              21.1               20.0   
2013               16.9              21.4               18.7   

      CITY (Le/100 km)  HWY (Le/100 km)  COMB (Le/100 km)  (g/km)  RATING  \textbackslash{}
YEAR                                                                        
2012               1.9              2.4               2.1       0     NaN   
2012               2.2              2.6               2.4       0     NaN   
2013               2.1              2.4               2.2       0     NaN   
2013               1.9              2.4               2.1       0     NaN   

      (km)  TIME (h)  
YEAR                  
2012   100         7  
2012   117         7  
2013   122         4  
2013   100         7  
        Make                            Model        Size  (kW) Unnamed: 5  \textbackslash{}
YEAR                                                                         
2013  NISSAN                             LEAF    MID-SIZE    80         A1   
2013   SMART  FORTWO ELECTRIC DRIVE CABRIOLET  TWO-SEATER    35         A1   

     TYPE  CITY (kWh/100 km)  HWY (kWh/100 km)  COMB (kWh/100 km)  \textbackslash{}
YEAR                                                                
2013    B               19.3              23.0               21.1   
2013    B               17.2              22.5               19.6   

      CITY (Le/100 km)  HWY (Le/100 km)  COMB (Le/100 km)  (g/km)  RATING  \textbackslash{}
YEAR                                                                        
2013               2.2              2.6               2.4       0     NaN   
2013               1.9              2.5               2.2       0     NaN   

      (km)  TIME (h)  
YEAR                  
2013   117         7  
2013   109         8  

    \end{Verbatim}

    Finally, we can also use \texttt{loc} and \texttt{iloc} to perform just
about any data selection operation. \texttt{loc} is label-based, which
means that you have to specify rows and columns based on their row and
column labels. \texttt{iloc} is integer index based, so you have to
specify rows and columns by their integer index like you did in the
previous exercise.

    \begin{Verbatim}[commandchars=\\\{\}]
{\color{incolor}In [{\color{incolor}12}]:} \PY{c+c1}{\PYZsh{} Import cars data}
         \PY{k+kn}{import} \PY{n+nn}{pandas} \PY{k}{as} \PY{n+nn}{pd}
         \PY{n}{cars} \PY{o}{=} \PY{n}{pd}\PY{o}{.}\PY{n}{read\PYZus{}csv}\PY{p}{(}\PY{l+s+s1}{\PYZsq{}}\PY{l+s+s1}{cars.csv}\PY{l+s+s1}{\PYZsq{}}\PY{p}{,} \PY{n}{index\PYZus{}col} \PY{o}{=} \PY{l+m+mi}{0}\PY{p}{)}
         
         \PY{c+c1}{\PYZsh{} Print out observation for Japan}
         \PY{n+nb}{print}\PY{p}{(}\PY{n}{cars}\PY{o}{.}\PY{n}{iloc}\PY{p}{[}\PY{l+m+mi}{2}\PY{p}{]}\PY{p}{)}
         
         \PY{c+c1}{\PYZsh{} Print out observations for 2012 and 2013}
         \PY{n+nb}{print}\PY{p}{(}\PY{n}{cars}\PY{o}{.}\PY{n}{loc}\PY{p}{[}\PY{p}{[}\PY{l+m+mi}{2012}\PY{p}{,} \PY{l+m+mi}{2013}\PY{p}{]}\PY{p}{]}\PY{p}{)}
\end{Verbatim}


    \begin{Verbatim}[commandchars=\\\{\}]
Make                           FORD
Model                FOCUS ELECTRIC
Size                        COMPACT
(kW)                            107
Unnamed: 5                       A1
TYPE                              B
CITY (kWh/100 km)                19
HWY (kWh/100 km)               21.1
COMB (kWh/100 km)                20
CITY (Le/100 km)                2.1
HWY (Le/100 km)                 2.4
COMB (Le/100 km)                2.2
(g/km)                            0
RATING                          NaN
(km)                            122
TIME (h)                          4
Name: 2013, dtype: object
            Make                            Model        Size  (kW)  \textbackslash{}
YEAR                                                                  
2012  MITSUBISHI                           i-MiEV  SUBCOMPACT    49   
2012      NISSAN                             LEAF    MID-SIZE    80   
2013        FORD                   FOCUS ELECTRIC     COMPACT   107   
2013  MITSUBISHI                           i-MiEV  SUBCOMPACT    49   
2013      NISSAN                             LEAF    MID-SIZE    80   
2013       SMART  FORTWO ELECTRIC DRIVE CABRIOLET  TWO-SEATER    35   
2013       SMART      FORTWO ELECTRIC DRIVE COUPE  TWO-SEATER    35   
2013       TESLA         MODEL S (40 kWh battery)   FULL-SIZE   270   
2013       TESLA         MODEL S (60 kWh battery)   FULL-SIZE   270   
2013       TESLA         MODEL S (85 kWh battery)   FULL-SIZE   270   
2013       TESLA              MODEL S PERFORMANCE   FULL-SIZE   310   

     Unnamed: 5 TYPE  CITY (kWh/100 km)  HWY (kWh/100 km)  COMB (kWh/100 km)  \textbackslash{}
YEAR                                                                           
2012         A1    B               16.9              21.4               18.7   
2012         A1    B               19.3              23.0               21.1   
2013         A1    B               19.0              21.1               20.0   
2013         A1    B               16.9              21.4               18.7   
2013         A1    B               19.3              23.0               21.1   
2013         A1    B               17.2              22.5               19.6   
2013         A1    B               17.2              22.5               19.6   
2013         A1    B               22.4              21.9               22.2   
2013         A1    B               22.2              21.7               21.9   
2013         A1    B               23.8              23.2               23.6   
2013         A1    B               23.9              23.2               23.6   

      CITY (Le/100 km)  HWY (Le/100 km)  COMB (Le/100 km)  (g/km)  RATING  \textbackslash{}
YEAR                                                                        
2012               1.9              2.4               2.1       0     NaN   
2012               2.2              2.6               2.4       0     NaN   
2013               2.1              2.4               2.2       0     NaN   
2013               1.9              2.4               2.1       0     NaN   
2013               2.2              2.6               2.4       0     NaN   
2013               1.9              2.5               2.2       0     NaN   
2013               1.9              2.5               2.2       0     NaN   
2013               2.5              2.5               2.5       0     NaN   
2013               2.5              2.4               2.5       0     NaN   
2013               2.7              2.6               2.6       0     NaN   
2013               2.7              2.6               2.6       0     NaN   

      (km)  TIME (h)  
YEAR                  
2012   100         7  
2012   117         7  
2013   122         4  
2013   100         7  
2013   117         7  
2013   109         8  
2013   109         8  
2013   224         6  
2013   335        10  
2013   426        12  
2013   426        12  

    \end{Verbatim}

    \subsection{Bonus Tutorials}\label{bonus-tutorials}

Now that the basics are covered, I won't update this as often. Anything
I add here may be tougher examples and may seem off the path of what we
have covered thus far. So if you're just focused on the basics of Python
and learning how to write some simple code, the above work should be
enough.

On the flip side, if you're looking for more of a challenge, I'm hoping
to find some much more interesting problems and have them saved here as
they come up. Enjoy!

\subsubsection{Generators}\label{generators}

Generators are very easy to implement, but a bit difficult to
understand.

Generators are used to create iterators, but with a different approach.
Generators are simple functions which return an iterable set of items,
one at a time, in a special way.

When an iteration over a set of item starts using the for statement, the
generator is run. Once the generator's function code reaches a "yield"
statement, the generator yields its execution back to the for loop,
returning a new value from the set. The generator function can generate
as many values (possibly infinite) as it wants, yielding each one in its
turn.

Here is a simple example of a generator function which returns 7 random
integers:

    \begin{Verbatim}[commandchars=\\\{\}]
{\color{incolor}In [{\color{incolor}34}]:} \PY{k+kn}{import} \PY{n+nn}{random}
         
         \PY{k}{def} \PY{n+nf}{lottery}\PY{p}{(}\PY{p}{)}\PY{p}{:}
             \PY{c+c1}{\PYZsh{} returns 6 numbers between 1 and 40}
             \PY{k}{for} \PY{n}{i} \PY{o+ow}{in} \PY{n+nb}{range}\PY{p}{(}\PY{l+m+mi}{6}\PY{p}{)}\PY{p}{:}
                 \PY{k}{yield} \PY{n}{random}\PY{o}{.}\PY{n}{randint}\PY{p}{(}\PY{l+m+mi}{1}\PY{p}{,} \PY{l+m+mi}{40}\PY{p}{)}
         
             \PY{c+c1}{\PYZsh{} returns a 7th number between 1 and 15}
             \PY{k}{yield} \PY{n}{random}\PY{o}{.}\PY{n}{randint}\PY{p}{(}\PY{l+m+mi}{1}\PY{p}{,}\PY{l+m+mi}{15}\PY{p}{)}
         
         \PY{k}{for} \PY{n}{random\PYZus{}number} \PY{o+ow}{in} \PY{n}{lottery}\PY{p}{(}\PY{p}{)}\PY{p}{:}
                \PY{n+nb}{print}\PY{p}{(}\PY{l+s+s2}{\PYZdq{}}\PY{l+s+s2}{And the next number is... }\PY{l+s+si}{\PYZpc{}d}\PY{l+s+s2}{!}\PY{l+s+s2}{\PYZdq{}} \PY{o}{\PYZpc{}}\PY{p}{(}\PY{n}{random\PYZus{}number}\PY{p}{)}\PY{p}{)}
\end{Verbatim}


    \begin{Verbatim}[commandchars=\\\{\}]
And the next number is{\ldots} 25!
And the next number is{\ldots} 14!
And the next number is{\ldots} 35!
And the next number is{\ldots} 36!
And the next number is{\ldots} 23!
And the next number is{\ldots} 6!
And the next number is{\ldots} 3!

    \end{Verbatim}

    This function decides how to generate the random numbers on its own, and
executes the yield statements one at a time, pausing in between to yield
execution back to the main for loop.

\paragraph{Exercise}\label{exercise}

Write a generator function which returns the Fibonacci series. They are
calculated using the following formula: The first two numbers of the
series is always equal to 1, and each consecutive number returned is the
sum of the last two numbers. Hint: Can you use only two variables in the
generator function? Remember that assignments can be done
simultaneously.

Solution will be contained in the 2nd code cell below.

Another hint: The code

    \begin{Verbatim}[commandchars=\\\{\}]
{\color{incolor}In [{\color{incolor}43}]:} \PY{n}{a} \PY{o}{=} \PY{l+m+mi}{1}
         \PY{n}{b} \PY{o}{=} \PY{l+m+mi}{2}
         \PY{n}{a}\PY{p}{,} \PY{n}{b} \PY{o}{=} \PY{n}{b}\PY{p}{,} \PY{n}{a}
         \PY{n+nb}{print}\PY{p}{(}\PY{n}{a}\PY{p}{,}\PY{n}{b}\PY{p}{)}
\end{Verbatim}


    \begin{Verbatim}[commandchars=\\\{\}]
2 1

    \end{Verbatim}

    will simultaneously switch the values of a and b.

    \begin{Verbatim}[commandchars=\\\{\}]
{\color{incolor}In [{\color{incolor}28}]:} \PY{c+c1}{\PYZsh{} fill in this function}
         \PY{k}{def} \PY{n+nf}{fib}\PY{p}{(}\PY{p}{)}\PY{p}{:}
             \PY{k}{pass} \PY{c+c1}{\PYZsh{} this is a null statement which does }
                  \PY{c+c1}{\PYZsh{} nothing when executed, useful as a placeholder.}
         
         \PY{c+c1}{\PYZsh{} testing code}
         \PY{k+kn}{import} \PY{n+nn}{types}
         \PY{k}{if} \PY{n+nb}{type}\PY{p}{(}\PY{n}{fib}\PY{p}{(}\PY{p}{)}\PY{p}{)} \PY{o}{==} \PY{n}{types}\PY{o}{.}\PY{n}{GeneratorType}\PY{p}{:}
             \PY{n+nb}{print}\PY{p}{(}\PY{l+s+s2}{\PYZdq{}}\PY{l+s+s2}{Good, The fib function is a generator.}\PY{l+s+s2}{\PYZdq{}}\PY{p}{)}
         
             \PY{n}{counter} \PY{o}{=} \PY{l+m+mi}{0}
             \PY{k}{for} \PY{n}{n} \PY{o+ow}{in} \PY{n}{fib}\PY{p}{(}\PY{p}{)}\PY{p}{:}
                 \PY{n+nb}{print}\PY{p}{(}\PY{n}{n}\PY{p}{)}
                 \PY{n}{counter} \PY{o}{+}\PY{o}{=} \PY{l+m+mi}{1}
                 \PY{k}{if} \PY{n}{counter} \PY{o}{==} \PY{l+m+mi}{10}\PY{p}{:}
                     \PY{k}{break}
\end{Verbatim}


    \begin{Verbatim}[commandchars=\\\{\}]
2 1

    \end{Verbatim}

    \begin{Verbatim}[commandchars=\\\{\}]
{\color{incolor}In [{\color{incolor}45}]:} \PY{c+c1}{\PYZsh{} fill in this function}
         \PY{k}{def} \PY{n+nf}{fib}\PY{p}{(}\PY{p}{)}\PY{p}{:}
             \PY{n}{a} \PY{o}{=} \PY{l+m+mi}{1}
             \PY{n}{b} \PY{o}{=} \PY{l+m+mi}{1}
             
             \PY{k}{while} \PY{l+m+mi}{1}\PY{p}{:}
                 \PY{k}{yield} \PY{n}{a}
                 \PY{n}{a}\PY{p}{,} \PY{n}{b} \PY{o}{=} \PY{n}{b}\PY{p}{,} \PY{n}{a} \PY{o}{+} \PY{n}{b}
         
         \PY{c+c1}{\PYZsh{} testing code}
         \PY{k+kn}{import} \PY{n+nn}{types}
         \PY{k}{if} \PY{n+nb}{type}\PY{p}{(}\PY{n}{fib}\PY{p}{(}\PY{p}{)}\PY{p}{)} \PY{o}{==} \PY{n}{types}\PY{o}{.}\PY{n}{GeneratorType}\PY{p}{:}
             \PY{n+nb}{print}\PY{p}{(}\PY{l+s+s2}{\PYZdq{}}\PY{l+s+s2}{Good, The fib function is a generator.}\PY{l+s+s2}{\PYZdq{}}\PY{p}{)}
         
             \PY{n}{counter} \PY{o}{=} \PY{l+m+mi}{0}
             \PY{k}{for} \PY{n}{n} \PY{o+ow}{in} \PY{n}{fib}\PY{p}{(}\PY{p}{)}\PY{p}{:}
                 \PY{n+nb}{print}\PY{p}{(}\PY{n}{n}\PY{p}{)}
                 \PY{n}{counter} \PY{o}{+}\PY{o}{=} \PY{l+m+mi}{1}
                 \PY{k}{if} \PY{n}{counter} \PY{o}{==} \PY{l+m+mi}{10}\PY{p}{:}
                     \PY{k}{break}
\end{Verbatim}


    \begin{Verbatim}[commandchars=\\\{\}]
Good, The fib function is a generator.
1
1
2
3
5
8
13
21
34
55

    \end{Verbatim}


    % Add a bibliography block to the postdoc
    
    
    
    \end{document}
